\make{manual}
\disable{figurecontents}
\libraryfile{Garnet}
\string{TitleString = `Sample Program'}
\use{Bibliography = `garnet.bib'}
\begin{titlepage}

\begin{titlebox}
\vspace{0.6 inch}
\bg{A Sample Garnet Program}

{\bf Brad A. Myers}
\vspace{0.3 line}
\value{date}
\end{titlebox}
\vspace{0.5 inch}
\begin{center}
{\bf Abstract}\end{center}
\begin{text}
This file contains a sample program written using the Garnet Toolkit.  The
program is a simple graphical editor that allows the user to create boxes
and arrows.

\vspace{0.5 inch}
\begin{center}
Copyright \j{w} 1990 - Carnegie Mellon University\end{center}

This research was sponsored by the Defense Advanced Research Projects
Agency (DOD), ARPA Order No. 4976, Amendment 20, under contract
F33615-87-C-1499,
monitored by the Avionics Laboratory, Air Force Wright Aeronautical
Laboratories, Aeronautical Systems Division (AFSC), Wright-Patterson AFB,
Ohio 45433-6543.

The views and conclusions contained in this document are
those of the authors and should not be interpreted as representing the
official policies, either expressed or implied, of the Defense Advanced
Research Projects Agency or the US Government.

\end{text}
\end{titlepage}


\string{overview = `1'} % \comment{26 pages}
\string{overview-first-page = `3'}
\string{apps = `27'} % \comment{12 pages}
\string{apps-first-page = `29'}
\string{tour = `41'} % \comment{20 pages}
\string{tour-first-page = `43'}
\string{tutorial = `61'} % \comment{42 pages}
\string{tutorial-first-page = `63'}
\string{kr = `101'} % \comment{52 pages}
\string{kr-first-page = `103'}
\string{opal = `151'} % \comment{70 pages}
\string{opal-first-page = `153'}
\string{inter = `219'} % \comment{78 pages}
\string{inter-first-page = `221'}
\string{aggregadgets = `295'} % \comment{54 pages}
\string{aggregadgets-first-page = `297'}
\string{gadgets = `347'} % \comment{116 pages}
\string{gadgets-first-page = `349'}
\string{debug = `461'} % \comment{20 pages}
\string{debug-first-page = `463'}
\string{demos = `481'} % \comment{10 pages}
\string{demos-first-page = `483'}
\string{sampleprog = `491'} % \comment{14 pages}
\string{sampleprog-first-page = `493'}
\string{gilt = `505'} % \comment{20 pages}
\string{gilt-first-page = `507'}
\string{c32 = `525'} % \comment{12 pages}
\string{c32-first-page = `527'}
\string{lapidary = `537'} % \comment{36 pages}
\string{lapidary-first-page = `539'}
\string{hints = `573'} % \comment{8 pages}
\string{hints-first-page = `575'}
\string{GlobalIndex = `580'}

\set{page = sampleprog-first-page}

\artsection{Introduction}
The program in this file is implemented using the Garnet Toolkit, and is
presented as an example of how to write programs using the toolkit.  The
program implements a graphical editor that allows the user to create boxes
with textual labels which can be connected by lines.  The lines have
arrowheads, and go from the center of one box to the center of another.
The boxes can be moved or changed size, and the arrows stay attached
correctly.  The boxes or lines can also be deleted, and the labels can be
edited.

The sample program is in the file \pr{demo-editor.lisp}, and a source and
binary (compiled) version should be available in the \pr{demos}
sub-directory of the Garnet system files.

This graphical editor shows the use of:
\begin{itemize}
\item Constraints: to keep the arrows centered,
to keep the name labels at the tops of boxes, etc.

\item Opal objects: roundtangles, cursor-text, windows.

\item Interactors: to choose which drawing mode (\pr{menu-interactor}), to edit
the text strings (two \pr{text-interactor}s), and to create new objects
(\pr{two-point-interactor}).

\item Toolkit widgets: \pr{Text-button-panel} (a form of menu),
\pr{graphics-selection} (to show
which object is selected and allow it to be moved), and \pr{arrow-line}s.
These widgets have built in Opal objects and Interactors.

\item Aggregadgets: to group the roundtangle and label string.

\item Creating instances from prototypes (creating the new boxes and arrows).
\end{itemize}

This code is about 365 lines long, including comments, and took me two
hours to code and one hour to debug.  I did not use any higher-level Garnet
tools to create it (it was all coded directly in Lisp).

\artsection{Loading the Editor}

After loading \pr{Garnet-loader}, either of the following commands
will load the editor:

\begin{programexample}
(garnet-load `demos:demo-editor')        {\it ; To load the compiled version}
\end{programexample}
or
\begin{programexample}
(garnet-load `demos-src:demo-editor')    {\it ; To load the interpreted version, which}
                                         {\it ; may make experimenting/debugging easier}
\end{programexample}



\artsection{User Interface}

A snap shot of the editor in use is shown in Figure \ref{SampleFig}.
\begin{figure}
\bar{}
\begin{center}
\graphic{Postscript=`sampleprog/toolkitpic.PS', boundingbox=file}\end{center}
\caption{A Sample Garnet Application.  The code for this application is
listed at the end of this technical report.}
\tag{SampleFig}
\bar{}
\end{figure}

The user interface is as follows.  The menu at the top determines the
current mode.  When the roundtangle is outlined, the user can draw new
boxes, and when the arrow is outlined, the user can draw new arrows.
Press with either the left or right mouse buttons over one to change modes.

To create a new roundtangle, press with the {\it right} button in the
workspace window (on the right) and hold down.  Drag to the desired size
and then release.  Next, type in the new name.  Various editing characters
are supported during typing, including:
\begin{description}
\item[] \pr{{\tt\char`\^}h, delete, backspace}: delete previous character.

\item[] \pr{{\tt\char`\^}d}: delete next character.

\item[] \pr{{\tt\char`\^}u}: delete entire string.

\item[] \pr{{\tt\char`\^}b, left-arrow}: go back one character.

\item[] \pr{{\tt\char`\^}f, right-arrow}: go forward one character.

\item[] \pr{{\tt\char`\^}a, home}: go to the beginning of the string.

\item[] \pr{{\tt\char`\^}e, end}: go to the end of the string.

\item[] \pr{{\tt\char`\^}y, insert}: insert the contents of the X cut buffer into the string at the
current point.
\end{description}
When finished typing, press RETURN or any mouse button to stop.

To create a new arrow, when in arrow mode, press with the {\it right} button over
a roundtangle and release over a different roundtangle.  An arrow will be
drawn starting at the center of the first roundtangle and going to the
center of the other one.

Press with the {\it left} button on a roundtangle or an arrow to select it.
Press on the background to cause there to be no selection.  Press and
release on the ``Delete'' button to delete the selected object.

If a roundtangle is selected, you can move it by pressing on a small white
square with the left button and dragging to the new position and releasing.
You can change its size by pressing with the left button on a black square.
While the dotted outline box is displayed, you can abort the operation by
moving outside the window and releasing, or by hitting {\tt\char`\^}G (control-G).

To change the string of a label, press on the label with the left button
and begin typing.  When finished typing, press RETURN or any mouse button
to stop, or press {\tt\char`\^}G to abort the editing and return to the original string.

\artsection{Overview of How the Code Works}

The next section contains the actual code for the demo editor.  This
section presents some the parts of the design and serves as a guide to the
code.

The standard ``Garnet style'' is to \pr{USE-PACKAGE} the \pr{KR} package,
and directly reference all the other Garnet packages, so this is how the
code is written.  Functions such as \pr{create-instance, s-value,
o-formula, formula, gv,} and \pr{gvl} are defined in \pr{KR}.

The first part of the code creates {\it prototypes} of the basic items that
the user will create: arrow lines and labeled boxes.  When the editor is
running, the code will create an {\it instance} of one of these
prototypes to get a new set of objects to be displayed.  The arrow line
object is composed of one \pr{arrow-line} from the Garnet-Gadget set, with
some special {\it constraints} on its end-points.  The arrow-line is
parameterized by the two objects it is connected to.  These two objects are
kept in slots of the arrow-line: \pr{:from-obj} and \pr{:to-obj}.
The constraints on the end-points of the arrow-line are
expressed as {\it formulas} that cause the arrow to go from the center of the
object stored in the \pr{:from-obj} slot of the arrowline, to the center of
the object in the \pr{:to-obj} slot.

The labeled box is more complicated.  It is composed of two parts: a
rounded-rectangle (``roundtangle'') and a label.  An {\it AggreGadget} is
used to compose these together.  The boundaries of the roundtangle are
defined by the values in the \pr{:box} slot, since the standard
\pr{Move-Grow-Interactor} modifies objects by setting this slot.  The label
string is constrained to be centered at the top of the roundtangle.  The
actual string used is stored both at the top level AggreGadget, and in the
text object, so formulas are set up to keep these two \pr{:string} slots
having the same value.

The next function (\pr{create-mode-menu}) creates the top menu that
contains a label object and an arrow object.  A feedback rectangle is also
created to show what the current mode is.  This feedback rectangle has
formulas that keep it over whatever mode object is selected.
An interactor is then created to allow the user to choose the mode.

The main command menu is created using \pr{create-menu}, which simply
creates an instance of a \pr{garnet-gadget:text-button-panel} in the
correct place.  The functions to be executed are \pr{delete-object} and
\pr{do-quit}, and these are defined next.  The only trick here is that if a
labeled box is deleted, the lines to it are also deleted.  For quit,
destroying a window automatically destroys all of its contents.

Creating a new object is fairly straightforward.  The interactor is queried
to find out whether to create a line or a box, and the appropriate kind of
object is then created.  Lines can only start or end in boxes, so the
appropriate boxes are found.  To appear in a window, the newly created
objects must be added to an {\it aggregate} which is attached to the window,
Here, the aggregate found by looking in the \pr{:objs-aggregate} slot of
the interactor.

Another important feature of the \pr{Create-New-Obj} procedure is that if
the object being created is a box, then it starts an interactor to allow
the user to type the text label.

The top-level, exported procedure, \pr{do-go}, starts everything up by
creating a window,
a sub-window to be the work area, and top-level aggregates for both
windows.  Another aggregate will hold the user-created objects.  The
\pr{selection} object will show which object is selected, and also allow
that object to be moved or grown (if it is not a line).  Two text editing
interactors are then created.  One is used when a new object is created to
have the user type in the initial name.  This one is started explicitly
using \pr{Start-interactor} in \pr{Create-New-Object}.  The other
interactor is used when the user presses on the text label of an object to
edit the name.

Finally, the interactor to create new objects is defined.  This one is a
little complex, because it needs to decide whether to use a line or
rectangle feedback based on the current mode.

The last step is to add the top level aggregates to the windows and call
\pr{update} to get the objects to appear.  If you are running Allegro, Lucid,
Lispworks, CMUCL, or MCL CommonLisp, they will begin operating by
themselves, but under other CommonLisps, the \pr{Main-Event-Loop} call
is needed to get the interactors to run.  The
\pr{Exit-Main-Event-Loop} function in \pr{Do-Quit} causes
\pr{Main-Event-Loop} to exit.

\newpage{}
\artsection{The Code}
\begin{programexample}
\include{sampleprog/demo-editor.lisp}
\end{programexample}

