\device{postscript}
\make{manual}
\disable{figurecontents}
\libraryfile{Garnet}
\string{TitleString = `Interactors'}
\use{Bibliography = `garnet.bib'}
\modify{FigureCounter, within unnumbered}
\modify{captionenv, fill, Spaces=Compact, below=0}

\define{programinlist=programexample, LeftMargin +0.5inch, RightMargin 0,
	above 0, below 0}

\begin{titlepage}
\begin{titlebox}
\vspace{0.6 inch}
\bg{Interactors Reference Manual:
Encapsulating Mouse and Keyboard Behaviors}


{\bf Brad A. Myers
James A. Landay
Andrew Mickish}
\vspace{0.3 line}
\value{date}
\end{titlebox}
\vspace{0.5 inch}
\begin{center}
{\bf Abstract}\end{center}
\begin{text}
This document describes a set of objects which encapsulate mouse and
keyboard behaviors.  The motivation is to separate the complexities of
input device handling from the other parts of the user interface.  We have
tried to identify some common mouse and keyboard behaviors and implement them in a
separate place.  There are only a small number of interactor types, but
they are parameterized in a way that will support a wide range of different
interaction techniques.  These interactors form the basis for all
interaction in the Garnet system.

\vspace{0.5 inch}
\begin{center}
Copyright \j{w} 1990 - Carnegie Mellon University\end{center}

This research was sponsored by the Defense Advanced Research Projects
Agency (DOD), ARPA Order No. 4976, Amendment 20, under contract
F33615-87-C-1499,
monitored by the Avionics Laboratory, Air Force Wright Aeronautical
Laboratories, Aeronautical Systems Division (AFSC), Wright-Patterson AFB,
Ohio 45433-6543.

The views and conclusions contained in this document are
those of the authors and should not be interpreted as representing the
official policies, either expressed or implied, of the Defense Advanced
Research Projects Agency or the US Government.

\end{text}
\end{titlepage}


\string{overview = `1'} % \comment{26 pages}
\string{overview-first-page = `3'}
\string{apps = `27'} % \comment{12 pages}
\string{apps-first-page = `29'}
\string{tour = `41'} % \comment{20 pages}
\string{tour-first-page = `43'}
\string{tutorial = `61'} % \comment{42 pages}
\string{tutorial-first-page = `63'}
\string{kr = `101'} % \comment{52 pages}
\string{kr-first-page = `103'}
\string{opal = `151'} % \comment{70 pages}
\string{opal-first-page = `153'}
\string{inter = `219'} % \comment{78 pages}
\string{inter-first-page = `221'}
\string{aggregadgets = `295'} % \comment{54 pages}
\string{aggregadgets-first-page = `297'}
\string{gadgets = `347'} % \comment{116 pages}
\string{gadgets-first-page = `349'}
\string{debug = `461'} % \comment{20 pages}
\string{debug-first-page = `463'}
\string{demos = `481'} % \comment{10 pages}
\string{demos-first-page = `483'}
\string{sampleprog = `491'} % \comment{14 pages}
\string{sampleprog-first-page = `493'}
\string{gilt = `505'} % \comment{20 pages}
\string{gilt-first-page = `507'}
\string{c32 = `525'} % \comment{12 pages}
\string{c32-first-page = `527'}
\string{lapidary = `537'} % \comment{36 pages}
\string{lapidary-first-page = `539'}
\string{hints = `573'} % \comment{8 pages}
\string{hints-first-page = `575'}
\string{GlobalIndex = `580'}

\set{page = inter-first-page}

\chapter{Introduction}

This document is the reference manual for the {\it Interactors} system,
which is part of the Garnet User Interface Development System
\cite{GarnetCHI}.
The Interactors module is responsible for handling all of the input
from the user.  Currently, this includes handling the mouse and
keyboard.

The design of the Interactors is based on the observation that there are
only a few kinds of behaviors that are typically used in graphical user
interfaces.  Examples of these behaviors are selecting one of a set (as in a
menu), moving or growing with the mouse, accepting keyboard typing, etc.
Currently, in Garnet, there are only nine types of interactive behavior,
but these are all that is necessary for the interfaces that Garnet
supports.  These behaviors are provided in Interactor objects.  When the
programmer wants to make a graphical object (created using Opal\dashthe
Garnet graphics package) respond to input, an
interactor object is created and attached to the graphical object.  In
general, the graphics and behavior objects are created and maintained
separately, in order to make the code easier to create and maintain.

This technique of having objects respond to inputs is quite novel, and
different from the normal method in other graphical object systems.  In
others, each type of object is responsible for accepting a stream of mouse
and keyboard events and managing the behavior.  Here, the interactors
handle the events internally, and cause the graphical objects to behave in
the desired way.

The Interactors, like the rest of Garnet, are implemented in CommonLisp for
X/11 and Macintosh QuickDraw.   Interactors are set up to work with the Opal
graphics package and the KR object and constraint systems, which are all
part of Garnet.

The motivation and an overview of the Interactors system is described in
more detail in conference papers \cite{Interactors, InterTOIS}.

Often, interactors will be included in the definition of Aggregadgets.  See
the Aggregadgets manual for a description of how this works.

\section{Advantages of Interactors}

The design for interactors makes creating graphical interfaces
easier.  Other advantages of the interactors are that:
\begin{itemize}
\item They are entirely ``look'' independent; any graphics can be attached to a
particular ``feel.''

\item They allow the details of the behavior of objects to be separated from the
application and from the graphics,
which has long been a goal of user interface software design.

\item They support multiple input devices operating in parallel.

\item They simulate multiple processing.  Different applications can be running
in different windows, and the operations attached to objects in all the
windows will execute whenever the mouse is pressed over them.  The
applications all exist in the same CommonLisp process, but the interactors
insure that the events go to the correct application and that the
correct procedures are called.  If the application is written correctly
(e.g., without global variables), multiple instantiations of the {\it same}
application can exist in the same process.

\item All of the complexities of X and QuickDraw graphics and event handling
are hidden by Opal and the Interactors package.  This makes Garnet
much easier to use than X or QuickDraw, and allows applications written in
Garnet to be run on either Unix or Mac machines without modification.
\end{itemize}


\section{Overview of Interactor Operation}

\index{Inter Package}
The interactors sub-system resides in the \pr{Inter} package.  We
recommend that programmers explicitly reference names from the
\pr{Inter} package, for example: \pr{Inter:Menu-Interactor}, but you
can also get complete access to all exported symbols by doing a
{\obeyspaces \pr{(use-package :INTER)}}.  All of the symbols referenced in this
document are exported.

In a typical mouse-based operation, the end user will press down on a mouse
button to start the operation, move the mouse around with the button
depressed, and then release to confirm the operation.  For example, in a
menu, the user will press down over one menu item to start the operation,
move the mouse to the desired item, and then release.

Consequently, the interactors have two modes: waiting and running.  An
interactor is waiting for its start event (like a mouse button down) and
after that, it is waiting for its stop event, after which it
stops running and goes back to waiting.

In fact, interactors are somewhat more complicated because they can be
aborted at any time and because there are often active regions of the
screen outside of which the interactor does not operate.  The full
description of the operation is presented in section \ref{operation}.

\index{box (slot)}
\index{Selected (slot)}
All the interactors operate by setting specific slots in the graphic
objects.\foot{``Slots'' are the ``instance variables'' of the objects.}
For example, the menu interactor sets a slot called \pr{:selected} to show
which menu item is selected, and the moving and growing interactor sets a
slot called \pr{:box}.  Typically, the objects will contain constraints
that tie appropriate graphical properties to these special slots.  For
example, a movable rectangle would typically contain the following
constraints so it will follow the mouse:
\index{Moving-Rectangle}
\indexsecondary{Primary=`Examples', Secondary=`Moving-Rectangle'}
\indexsecondary{Primary=`Examples', Secondary=`Box (slot}')
\begin{programexample}
(create-instance 'MOVING-RECTANGLE opal:rectangle
   (:box '(80 20 100 150))
   (:left (o-formula (first (gvl :box))))
   (:top (o-formula (second (gvl :box))))
   (:width (o-formula (third (gvl :box))))
   (:height (o-formula (fourth (gvl :box)))))
\end{programexample}
The initial size and position for the rectangle are in the \pr{:box} slot.
When an interactor changed the box slot, the \pr{:left, :top, :width,} and
\pr{:height} slots would change automatically based on constraints.

If the
constraints (formulas) were {\it not} there, the interactor would still
change the \pr{:box} slot, {\it but nothing would change on the screen},
since the rectangle's display is controlled by \pr{:left, :top, :width,} and
\pr{:height}, not by \pr{:box}.
The motivation for setting this extra slot, is to allow application-specific
filtering on the values.  For example, if you do not want the object to
move vertically, you can simply eliminate the formula in the \pr{:top} slot.

\section{Simple Interactor Creation}

\index{create-instance}
\index{Interactor-window}
\index{window}
To use interactors, you need to create {\it interactor-windows} for the
interactors to work in (windows are fully documented in the Opal Manual).
To create an interactor-window, you use the standard KR
\pr{create-instance} function.  For example:
\indexsecondary{Primary=`Examples', Secondary=`Window Creation'}
\indexsecondary{Primary=`Examples', Secondary=`Creating Interactor Window'}
\begin{programexample}
(create-instance 'MYWINDOW inter:interactor-window
   (:left 100)(:top 10)
   (:width 400)(:height 500)
   (:title `My Window'))
(opal:update MYWINDOW)
\end{programexample}

To create interactor objects, you also use the \pr{create-instance}
function.  Each interactor has a large number of optional parameters, which are
described in detail in the rest of this manual.  It must be emphasized,
however, that normally it is not necessary to supply very many of these.
For example, the following code creates an interactor that causes the
MOVING-RECTANGLE (defined above) to move around inside MYWINDOW:
\indexsecondary{Primary=`Examples', Secondary=`Mover for Moving-Rectangle'}
\begin{programexample}
(create-instance 'MYMOVER Inter:Move-Grow-Interactor
   (:start-where (list :in MOVING-RECTANGLE))
   (:window MYWINDOW))
\end{programexample}

This interactor will use the default start and stop events, which are the
left mouse button down and up respectively.  All the other aspects of the
behavior also will use their default values (as described below).

Several implementations of lisp allow interactors to run automatically
(see section \ref{notcmulisp}).  If you are {\it not} running in CMU,
LispWorks, Allegro, Lucid, or MCL Commonlisp, then you need to execute the
following function to make the interactor run:
\begin{programexample}
(inter:main-event-loop)
\end{programexample}
This function does not exit, so you have to type {\tt\char`\^}C (or whatever your
operating system break character is) to the Lisp window when you are
finished (or hit the \pr{F1} key (or whatever your Garnet break key
is\dashsection \ref{Garnetbreakkey})).

As another example, here is a complete, minimal ``Goodbye World'' program, that
creates a window with a button that causes the window to go away (created
from scratch, without using any predefined gadgets).

\label{eventloopexample}
\index{Goodbye World}
\index{Example Program}
\indexsecondary{Primary=`Examples', Secondary=`Goodbye World'}
\indexsecondary{Primary=`Examples', Secondary=`Complete Program'}
\begin{programexample}
;;;{\it using the KR package, but no others, is the `Garnet style'}
(use-package `KR')
;;;{\it first create the graphics; see the Opal manual for explanations}
(create-instance 'MYWINDOW inter:interactor-window
   (:left 100)(:top 10)
   (:width 125)(:height 25)
   (:title `My Window'))
(s-value MYWINDOW :aggregate (create-instance 'MYAGG opal:aggregate))

(create-instance 'MYTEXT opal:text
   (:string `Goodbye World')
   (:left 2)(:top 5))
(opal:add-component MYAGG MYTEXT)
(opal:update MYWINDOW)

;;;{\it now add the interactor}
(create-instance NIL Inter:Button-Interactor
   (:window MYWINDOW)
   (:start-where (list :in MYTEXT))
   (:continuous NIL) ;{\it happen immediately on the downpress}
   (:final-function \#'(lambda (inter final-obj-over)
                        (opal:destroy MYWINDOW)
                        ;; {\it the next line is needed unless you are running CMU Lisp}
			;; {\it or you are running the main-event-loop process in the}
                        ;; {\it background in Allegro, Lucid, or LispWorks}
			\#-or cmu allegro lucid lispworks) (inter:exit-main-event-loop)))
;;;{\it If not CMU Lisp, or if not running the background main-event-loop process in}
;;;{\it Allegro, LispWorks, or Lucid Lisp, then the following is needed to run the interactor:}
\#-(or cmu allegro lucid lispworks) (inter:main-event-loop)
\end{programexample}



\section{Overview of Manual}

This manual is organized as follows.  Section \ref{notcmulisp}
discusses the \pr{main-event-loop}, which allows you to run interactors
while automatically updating the appearance of the windows.
Section \ref{detailsofOper}
describes how interactors work in detail.  Section \ref{accelerators-sec}
describes the definition and operation of global accelerators.  Section
\ref{interslots} lists all the slots that are common to all interactors.
Section \ref{specificinters} describes all the interactors that are provided.
Section \ref{transcripts-sec} describes how to make transcripts of events.
Finally, section \ref{advancedfeatures} describes some advanced features.

Normally, you will not need most of the information in this manual.  To
make an object respond to the mouse, look in section \ref{specificinters}
to find the interactor you need, then check its introduction to see how to
set up the constraints in your graphical objects so that they will respond
to the interactor, and to see what parameters of the interactor you need to
set.  You can usually ignore the advanced customization sections.


\chapter{The Main Event Loop}
\index{CommonLisp}
\index{Lucid}
\index{CMU CommonLisp}
\index{lispworks}
\label{notcmulisp}

CMU CommonLisp \cite{CMUCommonLisp} supports sending events to the
appropriate windows internally.  Therefore, under CMU CommonLisp, the
interactors begin to run immediately when they are created, and run
continuously until they are terminated.  While they are running, you can
still type commands to the Lisp listener (the read-eval-print loop).

To get the same effect on other Lisps, Garnet uses the multiple process
mechanism of Lucid, Allegro, LispWorks, and MCL CommonLisps.  You
usually do not need to worry about the information in this section if you are
using CMU, Allegro, Lucid, or MCL CommonLisp, but you will probably need to go
through an initialization phase for multiprocessing in LispWorks (see the
section `LispWorks' in the Overview Manual).

Note: \pr{Main-Event-Loop} also handles Opal window refreshing, so graphical
objects will not be redrawn automatically in other lisps unless this
function is executing.


\section{Main-Event-Loop}

\index{main-event-loop}
\index{exit-main-event-loop}
\index{Garnet-Break-Key}
\index{*Garnet-Break-Key*}
\index{F1}
\label{Garnetbreakkey}

Under other CommonLisps (like AKCL and CLISP), you need to explicitly
start and stop the main loop that listens for X events.  It is always
OK to call the \pr{main-event-loop} function, because it does nothing
if it is not needed.  Therefore, after
all the objects and interactors have been created, and after the
\pr{opal:update} call has been made, you must call the
\pr{inter:main-event-loop} procedure.  This loops waiting and handling X
events until explicitly stopped by typing \pr{{\tt\char`\^}C} (or whatever is your
operating system break character) to the Lisp listener window, or until you
hit the Garnet break key while the mouse is in a Garnet window.  This
is defined by the global
variable \pr{inter:*Garnet-Break-Key*}, and is bound to \pr{:F1} by
default.  You can simply setf \pr{inter:*Garnet-Break-Key*} to some other
character if you want to use \pr{:F1} for something else.

The other way for a program to exit \pr{Main-Event-Loop} is for it to call
the procedure \pr{inter:exit-main-event-loop}.  Typically,
\pr{inter:main-event-loop} will be called at the end of your set up
routine, and \pr{inter:exit-main-event-loop} will be called from your quit
routine, as in the example of section \ref{eventloopexample}.

\begin{programexample}
inter:Main-Event-Loop \&optional {\it inter-window}\value{function}

inter:Exit-Main-Event-Loop\value{function}
\end{programexample}

The optional window to \pr{Main-Event-Loop} is used to tell which display
to use.  If not supplied, it uses the default Opal display.  You only need
to supply a parameter if you have a single Lisp
process talking to multiple displays.


\section{Main-Event-Loop Process}
By default, Garnet spawns a background process in Allegro, Lucid, and
LispWorks, which will run the interactor's main-event-loop while simultaneously
allowing you to use the ordinary Lisp listener.
This means that you can use the Lisp
listener without having to hit the Garnet break key (usually \pr{:F1}).

Some programs seem to have trouble with this process.  If your system doesn't
work, try killing the main-event-loop process and executing
\pr{(inter:main-event-loop)} explicitly.  In MCL, the background
process is controlled by MCL itself, and cannot be killed.  However,
you might be able to break out of an infinite loop (or otherwise get
MCL's attention) by executing the abort command (Control-comma) or the
reset command (Control-period).


\begin{group}
\subsection{Launching and Killing the Main-Event-Loop-Process}
\index{launch-main-event-loop-process}
\index{kill-main-event-loop-process}

\vspace{1 line}
\begin{programexample}
opal:Launch-Main-Event-Loop-Process \value{function}

opal:Kill-Main-Event-Loop-Process \value{function}
\end{programexample}

These are the top-level functions used for starting and stopping the
main-event-loop process.
You may need to call \pr{launch-@{\tt\char`\|}main-@{\tt\char`\|}event-@{\tt\char`\|}loop-@{\tt\char`\|}process} if the
process is killed explicitly or if the process crashes due to a bug.
\end{group}
\vspace{1 line}

While the main-event-loop background process is running, calling
\pr{(inter:main-event-loop)}, hitting the Garnet break key, and
calling \pr{launch-main-event-loop-process} all have no effect.

You can kill the background main-event-loop process by executing
\pr{kill-@{\tt\char`\|}main-@{\tt\char`\|}event-@{\tt\char`\|}loop-@{\tt\char`\|}process}, but normally you should not
have to, even if you encounter an error and are thrown in the debugger.
If you call it when the main-event-loop process is not running, there
is no effect.

\pr{Launch-main-event-loop-process} and \pr{kill-main-event-loop-process}
belong to the Opal package because \pr{opal:reconnect-garnet} and
\pr{opal:disconnect-garnet} need to call them.


\subsection{Launch-Process-P}
\index{launch-process-p}
In the \pr{garnet-loader}, there is a switch called \pr{user::launch-process-p}
which tells whether or not Garnet should automatically call
\pr{launch-main-event-loop-process} at load time.  You can edit the
\pr{garnet-loader} to change the default value of this variable, or you
can \pr{setf} the variable before loading \pr{garnet-loader}.


\subsection{Main-Event-Loop-Process-Running-P}
\index{main-event-loop-process-running-p}

\begin{programexample}
opal:Main-Event-Loop-Process-Running-P \value{function}
\end{programexample}

This function tells you whether
the parallel main-event-loop process is running, and is not in the debugger.



\chapter{Operation}
\label{detailsofOper}

\section{Creating and Destroying}
For interactors to be used, they must operate on objects that appear in
Garnet windows.  The \pr{inter:interactor-@{\tt\char`\|}window} prototype is described
in the Opal Manual.  To create an interactor window, use:

\begin{programexample}
(create-instance {\it name} inter:interactor-window {\it (slot value)(slot value)...})
\end{programexample}
\index{create-instance}
\index{Interactor-Window}
This creates an interactor window named {\it name} (which will usually be a
quoted symbol like \pr{'MYWINDOW} or NIL).  If {\it name} is NIL, then a
system-supplied name is used.  This returns the new
window.  The \pr{:left}, \pr{:top}, \pr{:width}, and \pr{:height} (and other
parameters) are given just as for all objects.
Note that the window is not visible (``mapped'')
until an \pr{opal:update} call is made on it:
\begin{programexample}
(opal:update {\it an-interactor-window})
\end{programexample}

To create an interactor, use:
\begin{programexample}
(create-instance {\it name} Inter:{\it InteractorType} {\it (slot value) (slot value)...})
\end{programexample}
\index{create-instance}
This creates an interactor named {\it name} (which can be NIL if
a system-supplied name is desired) that is an instance of {\it InteractorType}
(which will be one of the specific types described in section
\ref{specificinters}, such as \pr{button-interactor, menu-interactor,
etc.}  The slots and values are the other parameters to the new
interactor, as described in the rest of this manual.  The create-instance
call returns the interactor.

\vspace{1 line}

\index{Destroy}
\begin{programexample}
opal:Destroy {\it an-interactor} \&optional ({\it erase} T) \value{method}

opal:Destroy {\it an-interactor-window} \value{method}
\end{programexample}
\vspace{1 line}

Invoking this method destroys an interactor or window.  If {\it erase} is T,
then the interactor is aborted and
deallocated.  If {\it erase} is NIL, it is just destroyed.  Use NIL
when the window the interactor is in is going to be destroyed anyway.
Normally, it is not necessary to call this on interactors since they are
destroyed automatically when the window they are associated with is destroyed.

Invoking this method on a window destroys the window, all objects in it,
and all interactors associated with it.


\section{Continuous}
\label{continuous}
\index{Continuous}
Interactors can either be {\it continuous} or not.  A continuous interactor
operates between a start and stop event.  For example, a Move-Grow interactor
might start the object following the mouse when the left button goes down,
and continue to move the object until the button is released.  When the
button is released, the interactor will stop, and the object will stay in
the final place.  Similarly, a menu interactor can be continuous to show
the current selection while the mouse is moving, but only make the final
selection and do the associated action when the button is released.

The programmer might want other interactors to operate only once at the time the
start-event happens.
For example, a non-continuous \pr{Button-Interactor} can be used to
execute some action when the \pr{delete} key is hit on the keyboard.

The \pr{:continuous} slot of an interactor controls whether the interactor
is continuous or not.  The default is T.

Many interactors will do reasonable things for both values of
\pr{:continuous}.  For example, a continuous \pr{button-interactor} would
allow allow the user to press down on the graphical button, and then move
the mouse around.  It would only execute the action if the mouse button is
released over the graphical button.  This is the way Macintosh buttons
work.  A non-continuous button would simply execute as soon as the
mouse-button was hit over the graphical button, and not wait for the release.

\section{Feedback}
\index{Feedback}

When an interactor is continuous, there is usually some feedback to show
the user what is happening.  For example, when an object is being moved
with the mouse, the object usually moves around following the mouse.
Sometimes, it is desirable that the actual object not move, but rather
that a special {\it feedback object} follows the mouse, and then the real
object moves only when the interaction is complete.

\index{Feedback-obj}

The interactors support this through the use of the \pr{:feedback-obj}
slot.  If a graphical object is supplied as the value of this slot, then
the interactor will modify this object while it is running, and only modify
the ``real'' object when the interaction is complete (section \ref{where}
discusses how the interactor finds the ``real'' object).  If no value is
supplied in this slot (or if NIL is specified), then the interactor will
modify the actual object while it is running.  In either case, the
operation can still be aborted, since the interactor saves enough state to
return the objects to their initial configuration if the user requests an
abort.

\index{box (slot)}
Typically, the feedback object will need the same kinds of constraints as
the real object, in order to follow the mouse.  For example, a feedback
object for a \pr{Move-Grow-Interactor} would need formulas to the \pr{:box}
slot.  The sections on the various specific interactors discuss the slots
that the interactors set in the feedback and real objects.

\section{Events}
\label{eventspec}
\index{events}
\index{keyboard keys}
\index{mouse buttons}

An interactor will start running when its {\it start event} occurs and
continue to run until a {\it stop event} occurs.  There may also be an
{\it abort event} that will prematurely cause it to exit and restore the
status as if it had not started.

An ``event'' is usually a transition of a mouse button or keyboard key.
Interactors provide a lot of flexibility as to the kinds of events that
can be used for start, stop and abort.


\subsection{Keyboard and Mouse Events}
\index{leftdown}
\index{middledown}
\index{rightdown}
\index{\#{\tt\char`\\} (character prefix)}
Events can be a mouse button down or up transition, or
any keyboard key.  The names for the mouse buttons are \pr{:leftdown},
\pr{:middledown}, and \pr{:rightdown} (simulating multiple mouse
buttons on the Mac is discussed in section \ref{mac-keys}).  Keyboard
keys are named by their CommonLisp character, such as \pr{\#{\tt\char`\\}g},
\pr{\#{\tt\char`\\}a}, etc.  Note that \pr{\#{\tt\char`\\}g} is lower-case `g' and \pr{\#{\tt\char`\\}G} is
upper case `G' (shift-g).

When specifying shift keys on keyboard events, it is important to be
careful about the `{\tt\char`\\}'.  For example, \pr{:control-g} is {\it upper} case
`G' and \pr{:control-{\tt\char`\\}g} is {\it lower} case `g' (note the extra `{\tt\char`\\}').
You may also use the form \pr{:{\tt\char`\|}CONTROL-g{\tt\char`\|}}, which is equivalent to
\pr{:control-{\tt\char`\\}g} (when using vertical bars, you must put the CONTROL
in upper-case).  It is not legal to use the \pr{shift} modifier with
keyboard keys.

\index{any-leftdown}\index{any-middledown}\index{any-rightdown}
\index{any-leftup}\index{any-middleup}\index{any-rightup}
\index{any-mousedown}\index{any-mouseup}\index{any-keyboard}
Events can also be specified in a more generic manner using
\pr{:any-leftdown}, \pr{:any-middledown}, \pr{:any-rightdown},
\pr{:any-leftup}, \pr{:any-middleup}, \pr{:any-rightup}, \pr{:any-mousedown},
\pr{:any-mouseup}, and \pr{:any-keyboard}.  For these, the event will be
accepted no matter what modifier keys are down.


\subsection{`Middledown' and `Rightdown' on the Mac}
\label{mac-keys}
\index{leftdown}
\index{middledown}
\index{rightdown}
\index{Mac mouse buttons}
To simulate the three-button mouse on the Macintosh, we use keyboard
keys in place of the buttons.  By default, the keys are \pr{F13},
\pr{F14}, and \pr{F15} for the left, middle, and right mouse buttons,
respectively.  The real mouse button is also mapped to \pr{:leftdown},
so you can specify mouse events as usual on the Mac (e.g., \pr{:rightdown}).
The Overview section at the beginning of this manual provides
instructions for customizing the keys that simulate the mouse buttons,
and provides instructions for a small utility that changes the keys to
be used from function keys to arrow keys.


\subsection{Modifiers (Shift, Control, Meta)}
\index{control}
\index{meta}
\index{shift}
\index{command (Mac key)}
Various modifier keys can be specified for the event.  The valid
prefixes are \pr{shift}, \pr{control}, and \pr{meta}.
For example, \pr{:control-meta-leftdown} will only be true
when the left mouse button goes down while both the Control and Meta
keys are held down.  When using a conglomerate keyword like
\pr{:shift-meta-middleup},
the order in which the prefixes are listed matters.  The required order for
the prefixes is: \pr{shift, control, meta}.  For instance,
\pr{:shift-control-leftdown} is legal; \pr{:control-shift-leftdown} is not.

As with MCL itself, the Option key is the `Meta' modifier on the Mac.
There is no way to access the Mac's Command key through Garnet.


\subsection{Window Enter and Leave Events}

\index{Window-Enter event}
\index{Window-Leave event}
Sometimes it is useful to know when the cursor is inside the window.
Garnet has the ability to generate events when the cursor enters and
leaves a window.  To enable this, you must set the
\pr{:want-enter-leave-events} slot of the window to T {\it at window
creation time}.  Changing the value of this slot after the window has
been created will not necessarily work.  If the window has this value as
non-NIL, then when the cursor enters the window, a special event
called \pr{:window-enter} will be generated, and when the cursor
exits, \pr{:window-leave} will be generated.  For example, the
following will change the color of the window to red whenever the
cursor is inside the window:

\begin{programexample}
(create-instance 'MY-WIN inter:interactor-window
  (:want-enter-leave-events T)
  (:aggregate (create-instance NIL opal:aggregate)))
(opal:update MY-WIN)

(create-instance 'SHOW-ENTER-LEAVE inter:button-interactor
  (:start-event '(:window-enter :window-leave))
  (:window MY-WIN)
  (:continuous NIL)
  (:start-where T)
  (:final-function \#'(lambda(inter obj)
		       (declare (ignore obj))
		       (s-value (gv inter :window)
				:background-color
                                {\it ; :start-char is described in section \ref{specialslots}}
				(if (eq :window-enter (gv inter :start-char))
				    opal:red
				    opal:white)))))
\end{programexample}


\subsection{Double-Clicking}
\index{double clicking}
\index{double-click-time}
Garnet also supports double-clicking of the mouse buttons.  When the
variable \pr{inter:*Double-Click-Time*} has a non-NIL value, then it
is the time in milleseconds of how fast clicks must be to be
considered double-clicking.  By default, double clicking is enabled
with a time of 250 milleseconds.
When the user double-clicks, Garnet first reports the first press and
release, and then a \pr{:double-xxx} press and then a {\it regular} release.
For example, the events that will be reported on a double-click of the
left button are: \pr{:leftdown :leftup :double-leftdown :leftup}.
Note that the normal \pr{-up} events are used.  You can use the normal
\pr{:shift, :control,} and \pr{:meta} modifiers in the usual order,
before the \pr{double-}.  For example: \pr{:shift-control-double-middledown}.
If you specify the start-event of a continuous interactor to use a
\pr{:double-} form, then the correct stop event will be generated
automatically.  If you have both single and double click interactors,
then you should be careful that it is OK for the single click one to
run before the double-click one.

\index{triple clicking}
If you want to handle triple-clicks, quadruple-clicks, etc., then you
have to count the clicks yourself.  Garnet will continue to return
\pr{:double-xxx} as long as the clicks are fast enough.  When the user
pauses too long, there will be a regular \pr{:xxxdown} in between.
Therefore, for triple click, the events will be: \pr{:leftdown, :leftup,
:double-leftdown, :leftup, :double-leftdown, :leftup}
whereas for double-click-pause-click, the events will be:
\pr{:leftdown, :leftup, :double-leftdown, :leftup, :leftdown, :leftup}.


\subsection{Function Keys, Arrows Keys, and Others}

The various special keys on the keyboard use special keywords.
For example, \pr{:uparrow}, \pr{:delete}, \pr{:F9}, etc.
The prefixes are added in the same way as for mouse buttons (e.g.,
\pr{:control-F3}).  The arrow keys
are almost always named \pr{:uparrow}, \pr{:downarrow}, \pr{:leftarrow}, and
\pr{:rightarrow} (and so there are no bindings for \pr{:R8} (uparrow),
\pr{:R10} (leftarrow), \pr{:R12} (rightarrow), and \pr{:R14} (downarrow) on
the Sun keyboard).  On the Mac, some users prefer to change their
arrow keys to generate mouse events (see section \ref{mac-keys}).
To see what the Lisp character is for an event, turn on event tracing using
\pr{(Inter:Trace-Inter :event)} and then type the key in some interactor
window, as described in the Garnet Debugging Manual.
If you have keys on your keyboard that are not handled by Garnet, it is
easy to add them.  See the section on ``Keyboard Keys'' in the Overview
Manual, and then please send the bindings to \pr{garnetjcs.cmu.edu}
so we can add them to future versions of Garnet.

\index{*ignore-undefined-keys*}
\index{ignore-undefined-keys}
You can control whether Garnet raises an error when an undefined keyboard
key is hit.  The default for \pr{inter::*ignore-undefined-keys*} is T,
which means that the keys are simply ignored.  If you set this variable to
NIL, then an error will be raised if you hit a key with no definition.


\subsection{Multiple Events}
\index{except}
The event specification can also be a set of events, with an
optional exception list.
In this case, the event descriptor is a list, rather than a single event.
If there are exceptions, these should be at the end of the list after the
keyword \pr{:except}.
For example, the following lists are legal values when an event is called
for (as in the \pr{:start-event} slot):
\indexsecondary{Primary=`Examples', Secondary=`Events'}
\begin{programexample}
   (:any-leftdown :any-rightdown)
   (:any-mousedown \#{\tt\char`\\}RETURN)   ;;{\it any mouse button down or the RETURN key}
   (:any-mousedown :except :leftdown :shift-leftdown)
   (:any-keyboard :any-rightdown :except \#{\tt\char`\\}b \#{\tt\char`\\}a \#{\tt\char`\\}r)
\end{programexample}


\subsection{Special Values T and NIL}

Finally, the event specification can be T or NIL.  T matches any event and
NIL matches no event.  Therefore, if NIL is used for the \pr{:start-event},
then the interactor will never start by itself (which can be useful for
interactors that are explicitly started by a programmer).  If T is used for
the \pr{:start-event}, the interactor will start immediately when it is
created, rather than waiting for an event.  Similarly, if
\pr{stop-event} is NIL, the interactor will never stop by itself.


\section{Values for the ``Where'' slots}
\label{where}
\label{startwhere}

\subsection{Introduction}
\index{where}
\index{backquote}
\index{` (in a ``where'')}
\index{start-where}
In addition to specifying what events cause interactors to start and stop,
you must also specify {\it where} the mouse should be when the interaction
starts using the slot \pr{:start-where}.
The format for the ``where'' arguments is usually a list with a keyword at
the front, and an object afterwards.  For example, \pr{(:in myrect)}.
These lists can be conveniently created either using \pr{list} or
back-quote:
\indexsecondary{Primary=`Examples', Secondary=`Start-Where'}
\indexsecondary{Primary=`Examples', Secondary=`Where'}
\begin{programexample}
(:start-where (list :in MYRECT))
(:start-where `(:in ,MYRECT))
\end{programexample}
For the backquote version, be sure to put a comma before the object names.

The ``where'' specification often serves two purposes: it specifies where
the interaction should start and what object the interaction should work
on.

Unlike some other systems, the Interactors in Garnet will work on any of a
set of objects.  For example, a single menu interactor will handle all the
items of the menu, and a moving interactor will move any of a set of
objects.  Typically, the object to be operated on is chosen by the user
when the start event happens.  For example, the move interactor may move
the object that the mouse is pressed down over.  This one object continues
to move until the mouse is released.

Some of the interactors have an
optional parameter called \pr{:obj-to-change}, where you can specify a
different object to operate on than the one returned by the
\pr{:start-where} specification.

One thing to be careful about is that some slots of the {\it graphical
objects themselves} affect how they are picked, in particular, the
\pr{:hit-threshold}, \pr{:select-outline-only}, and \pr{:pretend-to-be-leaf}
slots.  See section \ref{hitthreshold}.

\subsection{Running-where}
\label{runningwhere}
\index{running-where}

There are actually two ``where'' arguments to each interactor.  One is the
place where the mouse should be for the interaction to start
(\pr{:start-where}).  The other is
the active area for the interaction (\pr{:running-where}).  The default
value for the running-where slot is usually the same as the start-where
slot.  As an example of when you might want them to be different,
with an object that moves with the mouse,
you might want to start moving when the press was over the object itself
(so \pr{:start-where} might be \pr{(:in MY-OBJ)})
but continue moving while the mouse is anywhere over the background
(so \pr{:running-where} might be \pr{(:in MY-BACKGROUND-OBJ)}).

\subsection{Kinds of ``where''}

\index{element (in a ``where'')}
There are a few basic kinds of ``where'' values.
\begin{description}
\item[] {\bf Single object}: These operate on a single object and check if the mouse
is inside of it.

\item[] {\bf Element of an aggregate}: These check if the object is an element of an
aggregate.  Aggregadgets and Aggrelists will also work since they are
subclasses of aggregate.

\item[] {\bf Element of a list}: The list is stored as the value of a slot of some
object.
\end{description}

\index{child vs. leaf}
\index{leaf vs. child}
\index{none}
The last two kinds have a number of varieties:
\begin{description}
\item[] {\bf Immediate child vs. leaf}: Sometimes it is convenient to ask if the
mouse is over a ``leaf'' object.  This is one of the basic types
(rectangle, line, etc.).  This is useful because aggregates often contain
extra white-space (the bounding box of an aggregate includes all of its
children, and all the space in between).  Asking for the mouse to be over a
leaf insures that the mouse is actually over a visible object.

\item[] {\bf Return immediate child or leaf}: If you want the user to have to press
on a leaf object, you may still want the interactor to operate on the top
level object.  Suppose that the movable objects in your system are
aggregates containing a line with an arrowhead and a label.
The user must press on one of the objects directly (so you want leaf), but
the interactor should move the entire aggregate, not just the line.  In
this case, you would use one of the forms that checks the leaf but returns
the element.

\item[] {\bf Or none}.  Sometimes, you might want to know when the user presses over
no objects, for example to turn off selection.  The ``or-none'' option
returns the object normally if you press on it, but if you press on no
object, then it returns the special value \pr{:none}.
\end{description}

Finally, there is a {\bf custom} method that allows you to specify your own
procedure to use.

\subsection{Type Parameter}
\label{leaftype}
\index{Type}
After the specification of the object, an optional \pr{:type} parameter
allows the objects to be further discriminated by type.  For example, you
can look for only the \pr{line}s in an aggregate using
\pr{`(:element-of ,MYAGG :type ,opal:line)}.  Note the comma in front of
\pr{opal:line}.

The type parameter can either be a single type, as shown above, or a list
of types.  In this case, the object must be one of the types listed (the
``or'' of the types).  For example
\begin{programexample}
`(:element-of ,MYAGG :type (,opal:circle ,opal:rectangle))
\end{programexample}
will match any element of \pr{myagg} that is either a circle or a rectangle.

Normally, the \pr{leaf} versions of the functions below only return
primitive (leaf) elements.  However, if the \pr{:type} parameter is given
and it matches an interior (aggregate) object, then that object is checked
and returned instead of a leaf.   For example, if an object is defined as follows:
\indexsecondary{Primary=`Examples', Secondary=`Type in Where'}
\begin{programexample}
(create-instance 'MYAGGTYPE opal:aggregate)

(create-instance 'TOP-AGG opal:aggregate)

(create-instance 'A1 MYAGGTYPE)
(create-instance 'A2 MYAGGTYPE)
(opal:add-components TOP-AGG A1 A2)

;;{\it now add some things to A1 and A2}
\end{programexample}
Then, the description \pr{(:leaf-element-of ,TOP-AGG :type ,MYAGGTYPE)}
will return \pr{A1} or \pr{A2} rather than the leaf elements of \pr{A1} or
\pr{A2}.

\index{pretend-to-be-Leaf}
\index{Leaf Objects}
Another way to prevent the search from going all the way to the actual leaf
objects is to set the \pr{:pretend-to-be-leaf} slot of an intermediate
object.  Note that the
\pr{:pretend-to-be-leaf} slot is set in the Opal objects, not in the
interactor, and it is more fully explained in the Opal manual.

\subsection{Custom}
\index{custom}
\label{customwhere}
The \pr{:custom} option for the \pr{:start-where} field can be used to set up
your own search method.  The format is:
\begin{programexample}
       (list :custom obj \#'function-name arg1 arg2 ...)
\end{programexample}
There can be any number of arguments supplied, even zero.  The function
specified is then called for each event that passes the event
test.  The calling sequence for the function is:
\begin{programexample}
	(lambda (obj an-interactor event arg1 arg2 ...))
\end{programexample}

The arguments are the values in the \pr{-where} list, along with the
interactor itself, and an event.
The \pr{event} is a Garnet event structure, defined in section \ref{Events}.
This function should return NIL if the event does not pass (e.g., if it is
outside the object), or else the object that the interactor should start
over (which will usually be obj itself or some child of obj).  The
implementor of this function should call \pr{opal:point-to-leaf}, or
whatever other method is desired.  The function is also required to check
whether the event occurred in the same window as the object.

For example, if the interactor is in an aggregadget, and we need a
custom checking function which takes the aggregadget and a special parameter
accessed from the aggregadget, the following could be used:
\indexsecondary{Primary=`Examples', Secondary=`Custom (Start-Where}')
\begin{programexample}

;;; First define the testing function
(defun Check-If-Mouse-In-Obj (obj inter event param)
  (if (and (eq (gv obj :window)(inter:event-window event))   {\it ; have to check window}
	   ({\tt\char`\>} (inter:event-x event) (gv obj :left))
	   .....)
      obj  ;{\it then return object}
      NIL))  ;{\it else return NIL}

(create-instance NIL opal:aggregadget
   ... ;{\it various fields}
  (:parameter-val 34)
  (:parts `((....)))
  (:interactors
   `((:start-it ,Inter:Button-Interactor
	 ... ;{\it all the usual fields}
	 (:start-where
	  ,(o-formula (list :custom (gvl :operates-on)
			    \#'Check-If-Mouse-In-Obj
			    (gvl :operates-on :parameter-val))))))))
\end{programexample}


\subsection{Full List of Options for Where}
All of the options for the where fields are concatenated together to form
long keyword names as follows:

\begin{description}
\item[] \pr{T} - anywhere.  This always succeeds.  (The T is not in a list.)  T for
the \pr{:start-where} means the interactor starts whenever the start-event
happens, and T for the \pr{:running-where} means the interactor runs until
the stop event no matter where the mouse goes.

\item[] \pr{NIL} - nowhere.  This never passes the test.  This is useful for
interactors that you want to start explicitly using \pr{Start-Interactor}
(section \ref{startinteractor}).

\item[] \pr{(:in {\tt\char`\<}obj>)}\index{in} - inside {\tt\char`\<}obj{\tt\char`\>}.  Sends the \pr{point-in-gob}
message to the object to ask if it contains the mouse position.

\item[] \pr{(:in-box {\tt\char`\<}obj>)}\index{in-box} - inside the rectangle of {\tt\char`\<}obj{\tt\char`\>}.  This
might be different from \pr{:in} the object since
some objects have special tests for inside.  For example, lines test for
the position to be near the line.  \pr{:In-box} may also be more
efficient than \pr{:in}.

% \begin{comment}
% \pr{(:full-object-in {\tt\char`\<}obj>)}\index{full-object-in} - makes sure the entire
% object being moved is
% inside the object specified here (if a circle is being moved with its
% center connected on the mouse, this will make sure that all of the circle
% is inside the {\tt\char`\<}obj{\tt\char`\>} specified here).  {\bf Not implemented yet.}
% \end{comment}
\item[] \pr{(:in-but-not-on {\tt\char`\<}agg>)}\index{in-but-not-on} - checks if point is
inside the bounding rectangle of {\tt\char`\<}agg{\tt\char`\>}, but not over any of the
children of {\tt\char`\<}agg{\tt\char`\>}.

\item[] \pr{(:element-of {\tt\char`\<}agg> [:type {\tt\char`\<}objtype>])}\index{element-of} - over any
element of the
aggregate {\tt\char`\<}agg{\tt\char`\>}.  If the \pr{:type} keyword is specified, then it searches
the components of {\tt\char`\<}agg{\tt\char`\>} for an element of the specified
type under the mouse.
This uses the Opal message \pr{point-to-component} on the aggregate.

\item[] \pr{(:leaf-element-of {\tt\char`\<}agg> [:type {\tt\char`\<}objtype>])}\index{leaf-element-of} -
over any leaf object of the
aggregate {\tt\char`\<}agg{\tt\char`\>}.  If the \pr{:type} keyword is specified, then it searches
down the hierarchy from {\tt\char`\<}agg{\tt\char`\>} for an element of the specified
type under the mouse.
This uses the Opal message \pr{point-to-leaf} on the aggregate.

\pr{(:element-of-or-none {\tt\char`\<}agg> [:type
\item[] {\tt\char`\<}objtype>])}\index{element-of-or-none} - This returns a non-NIL value whenever
the mouse is over \pr{{\tt\char`\<}agg>}.  If there is an object at the mouse, then it is
returned (as with \pr{:element-of}).  If there is no object, then the special
value \pr{:none} is returned.  If the mouse is not over
the aggregate, then NIL is returned.
This uses the Opal message \pr{point-to-component} on the aggregate.

\item[] \pr{(:leaf-element-of-or-none {\tt\char`\<}agg> [:type {\tt\char`\<}objtype>])}
\index{leaf-element-of-or-none} - Like \pr{:element-of-or-none}, except it
returns leaf children like \pr{:leaf-element-of}.
If there is an object at the mouse, then it is
returned.  If there is no object, then the special
value \pr{:none} is returned.  If the mouse is not over
the aggregate, then NIL is returned.
This uses the Opal message \pr{point-to-leaf} on the aggregate.

\pr{(:list-element-of {\tt\char`\<}obj> {\tt\char`\<}slot> [:type
\item[] {\tt\char`\<}objtype>])}\index{list-element-of} - the contents of the \pr{{\tt\char`\<}slot>} of
\pr{{\tt\char`\<}obj>} should be a list.  Goes through the list to find the object
under the mouse.  Uses \pr{gv} to get the list, so the contents of the
slot can be a formula that computes the list.
If the \pr{:type} keyword is specified, then it searches
the list for an element of the specified type.
This uses the Opal message \pr{point-in-gob} on each element of the list.

\pr{(:list-leaf-element-of {\tt\char`\<}obj> {\tt\char`\<}slot> [:type
\item[] {\tt\char`\<}objtype>])}\index{list-leaf-element-of} - like \pr{:list-element-of}, except if
one of the objects is an aggregate, then returns its leaf element.
The contents of the \pr{{\tt\char`\<}slot>} of \pr{{\tt\char`\<}obj>} should be a list.  Goes
through the list to find the object under the mouse.
Uses \pr{point-in-gob} if the object is {\it not} an aggregate, and uses
\pr{point-to-leaf} if it is an aggregate.

\pr{(:list-element-of-or-none {\tt\char`\<}obj> {\tt\char`\<}slot> [:type
\item[] {\tt\char`\<}objtype>])}\index{list-element-of-or-none} - like \pr{:list-element-of},
except if the event isn't over an object, then returns the special value \pr{:none}.
Note that this never returns NIL.

\pr{(:list-leaf-element-of-or-none {\tt\char`\<}obj> {\tt\char`\<}slot> [:type
\item[] {\tt\char`\<}objtype>])}\index{list-leaf-element-of-or-none} - like \pr{:list-leaf-element-of},
except if the event isn't over an object, then returns the special value \pr{:none}.
Note that this never returns NIL.

\pr{(:check-leaf-but-return-element {\tt\char`\<}agg> [:type
\item[] {\tt\char`\<}objtype>])}\index{check-leaf-but-return-element} -
This is like \pr{:leaf-element-of} except when an object is found, the
immediate component of \pr{{\tt\char`\<}agg>} is returned instead of the leaf element.
If the \pr{:type} keyword is specified, then it searches
the list for an element of the specified type.  This choice is useful, for
example, when the top level aggregate contains aggregates (or aggregadgets)
that mostly contain lines, and the programmer wants the user to have to
select on the lines, but still have the interactor affect the aggregate.

\pr{(:list-check-leaf-but-return-element {\tt\char`\<}obj> {\tt\char`\<}slot> [:type
\item[] {\tt\char`\<}objtype>])}\index{list-check-leaf-but-return-element} - like
\pr{:list-leaf-element-of}, except that it returns the element from the list
itself if a leaf element is hit.

\pr{(:check-leaf-but-return-element-or-none {\tt\char`\<}agg> [:type
\item[] {\tt\char`\<}objtype>])}\index{check-leaf-but-return-element-or-none} -
This is like \pr{:check-leaf-but-return-element} except that if no child is
under the event, but the event is inside the aggregate, then \pr{:none} is
returned.

\pr{(:list-check-leaf-but-return-element-or-none {\tt\char`\<}agg> [:type
\item[] {\tt\char`\<}objtype>])}\index{list-check-leaf-but-return-element-or...} -
This is like \pr{:list-check-leaf-but-return-element} except that if nothing is
found, \pr{:none} is returned instead of NIL.

\item[] \pr{(:custom {\tt\char`\<}obj> 'function-name arg1 arg2 ....)}\index{custom} -
Use a programmer-defined method to search for the object.  See section
\ref{customwhere}.

\end{description}

\subsection{Same Object}
\index{* (in a ``where'')}
A special value for the object can be used when the specification is in the
\pr{:running-where} slot.  Using \pr{*} means ``in the object
that the interactor started over.''  For example, if the start-where is
\pr{(:element-of {\tt\char`\<}agg>)}, a running-where of \pr{'(:in *)} would refer to
whatever object of the \pr{{\tt\char`\<}agg>} the interactor started over.  This \pr{*}
form cannot be used for the \pr{:start-where}.

\subsection{Outside while running}
\label{Outside}
\index{Outside}
\index{Last}
While the interactor is running, the mouse might be moved outside the area
specified by the \pr{:running-where} slot.  The value of the interactor
slot \pr{:outside} determines what happens in this case.  When
\pr{:outside} is NIL, which is the default, the interaction is
temporarily turned off until the mouse moves back inside.
This typically will make the feedback be invisible.  In this case, if the
user gives the stop event while outside, the interactor will be aborted.
For example, for a menu, the \pr{:running-where} will usually be
\pr{(:element-of MENU-AGG)} (same as the \pr{:start-where}).  If the user
moves outside of the menu while the mouse button is depressed, the feedback
will go off, and the mouse button is released outside, then no menu
operation is executed.
This is a convenient way to allow the user to abort an interaction once it
has started.

On the other hand, if you want the interactor to just save the last legal,
inside value, specify \pr{:outside} as \pr{:last}.  In this case, if the user
stops while outside, the last legal value is used.

If you want there to be no area that is outside (so moving everywhere is
legal), then simply set \pr{:running-where} to T, in which case the
\pr{:outside} slot is ignored.

\subsection{Thresholds, Outlines, and Leaves}
\label{hitthreshold}
\index{hit-threshold}
\index{select-outline-only}
\index{pretend-to-be-leaf}
Three slots of Opal objects are useful for controlling the ``where'' for
interactors.  These are \pr{:hit-threshold}, \pr{:select-outline-only},
and \pr{:pretend-to-be-leaf}.  If you set the
\pr{:select-outline-only} slot of an Opal object (note: {\it not} in the
interactor) to T, then all the ``where'' forms (except \pr{:in-box}) will
only notice the object when the mouse is directly over the outline.  The
\pr{:hit-threshold} slot of Opal objects determines how close to the
line or outline you must be (note that you usually have to set the
\pr{:hit-threshold} slot of the aggregate as well as for the
individual objects.)  See the Opal manual for more information on these slots.

An important thing to note is that if you are using one of the \pr{-leaf}
forms, you need to set the \pr{:hit-threshold} slot of {\it all the
aggregates} all the way down to the leaf from the aggregate you put in the
\pr{-where} slot.  This is needed if the object happens to be at the edge
of the aggregate (otherwise, the press will not be considered inside the
aggregate).

The \pr{:pretend-to-be-leaf} slot is used when you want an interactor
to treat an aggregate as a leaf (without it, only the components of an
aggregate are candidates to be leaves).  When you set the
\pr{:pretend-to-be-leaf} slot of an aggregate to T (note: not in the
interactor), then the search
for a leaf will terminate when the aggregate is reached, and the
aggregate will be returned as the current object.


\section{Details of the Operation}
\index{Functions}
\index{States (of interactors)}
\label{operation}

Each interactor runs through a standard set of states as it is running.
First, it starts off {\it waiting} for the start-event to happen over the
start-where.  Once this occurs, the interactor is {\it running} until the
stop-event or abort-event happens, when it goes back to waiting.  While it
is running, the mouse might move {\it outside} the active area (determined by
\pr{:running-where}), and later move {\it back inside}.  Alternatively, the
stop or abort events might happen while the mouse is still outside.
These state changes are implemented as a simple state machine inside each
interactor.

At each state transition, as well as continuously while the interactor is
running, special interactor-specific routines are called to do the actual
work of the interactor.  These routines are supplied with each interactor,
although the programmer is allowed to replace the routines to achieve
customizations that would otherwise not be possible.  The specifics of what
the default routines do, and the parameters if the programmer wants to
override them are discussed in section \ref{specificinters}.

The following table and figure illustrate the working of the state machine
and when the various procedures are called.

\begin{enumerate}
\item If the interactor is not {\it active}, then it waits until a program
explicitly sets the interactor to be active (see section \ref{active}).

\item If active, the interactor waits in the start state for the start-event to
happen while the mouse is over the specified start-where area.

\item When that event happens, if the interactor is {\it not} ``continuous''
(defined in section \ref{continuous}), then it executes the Stop-action
and returns to waiting for the start-event.  If the interactor is
continuous, then it does all of the following steps:
\begin{enumerate}
\item First, the interactor calls the Start-action and goes into the running state.

\item In the running state, it continually calls the running-action routine while
the mouse is in the running-where area.  Typically, the running-action is
called for each incremental mouse movement (so the running-action routine
is not called when the mouse is not moving).

\item If the mouse goes {\it outside} the
running-where area, then outside-action is called once.

\item If the mouse returns from outside running-where to be back inside, then the
back-inside-action is called once.

\item If the abort-event ever happens, then the abort-action is called and the
state changes back to the start state.

\item If the stop-event occurs while the mouse is inside running-where, then the
stop-action is called and the state returns to start.

\item If the stop-event occurs while the mouse is {\it outside}, then if the
\pr{:outside} field has the value \pr{:last}, the the stop-action is called
with the last legal value.  If \pr{:outside} is NIL, then the
abort-action is called.  In either case, the state returns to start.
Note: if \pr{:outside = :last}, and there is no abort-event, then there is
no way to abort an interaction once it has started.
\end{enumerate}
\end{enumerate}

If a program changes the active state to NIL (not active) and the
interactor is running or outside, the interactor is immediately aborted (so
the abort-action is called), and the interactor waits for a program to make
it active again, at which point it is in the start state.  (If the
interactor was in the start state when it became inactive, it simply waits
until it becomes active again.)  This transition is not shown in the
following figure.  Section \ref{active} discusses making an interactor in-active.

\begin{figure}
\bar{}
% \comment{l, bot, l->r,b->t}
\begin{center}
\graphic{Postscript=`inter/interstate.ps',boundingbox=File}\end{center}
\caption{Each Interactor runs the same state machine to control its operation.
The \pr{start-event}, \pr{stop-event} and \pr{abort-event} can be
specified (see section \ref{events}), as can the various
\pr{-action} procedures (section \ref{customroutines}).  Where the mouse
should be for the Interactor to
start (\pr{start-where}), and where it should run (\pr{running-where}) can
also be supplied as parameters (sections \ref{startwhere} and
\ref{runningwhere}).  The \pr{outside-control} parameter
determines whether the interaction is aborted when the user moves outside,
or whether the last legal value is used (section \ref{runningwhere}).
There are default values for all parameters, so the programmer does not
have to specify them.  In addition to the transitions shown, Interactors
can be aborted by the application at any time.}
\bar{}
\tag{StateMachineFig}
\end{figure}


\chapter{Mouse and Keyboard Accelerators}
\label{accelerators-sec}
\index{Accelerators}

The Interactors now have a new mechanism to attach functions to
specific keyboard keys as {\it accelerators}.  These are processed
either before or after interactors, and are either attached to a
particular window, or global to all windows.  If they are {\it after} the
interactors, then the accelerators are only used if no interactor
accepts the event.

(Note: If you are using the \pr{menubar} or \pr{motif-menubar}, then
you can use the slot \pr{:accelerator-windows} of those gadgets to
tell them which windows should have the keyboard accelerators defined
in them.)

By default, a number of {\it global} accelerators are defined:
\begin{programexample}
    :SHIFT-F1 -  raise window
    :SHIFT-F2 -  lower window
    :SHIFT-F3 -  iconify window
    :SHIFT-F4 -  zoom window
    :SHIFT-F5 -  fullzoom window
    :SHIFT-F6 -  refresh window
    :SHIFT-F7 -  destroy window

    :HELP - INSPECTOR object
    :CONTROL-HELP - INSPECTOR next interactor to run
    :SHIFT-HELP - print out object under the mouse (also in inspector.lisp)
\end{programexample}

The last three are processed {\it before} Interactors, and are defined
in the debugging file \pr{inspector.lisp}.
To change these, see the Debugging Reference Manual.
The first 7 are processed {\it after} the interactors.  To change these
bindings, set the variable \pr{*default-global-accelerators*}, which
is initially defined as:
\begin{programexample}
(defvar *default-global-accelerators* '(
    (:SHIFT-F1 .  raise-acc)
    (:SHIFT-F2 .  lower-acc)
    (:SHIFT-F3 .  iconify-acc)
    (:SHIFT-F4 .  zoom-acc)
    (:SHIFT-F5 .  fullzoom-acc)
    (:SHIFT-F6 .  refresh-acc)
    (:SHIFT-F7 .  destroy-acc)))
\end{programexample}

Applications can also set and maintain their own accelerator keys,
using the following functions:
\index{add-global-accelerator}
\index{add-window-accelerator}
\begin{programexample}
inter:Add-Global-Accelerator {\it key fn} \&key {\it replace-existing? first?} \value{function}
inter:Add-Window-Accelerator {\it win key fn} \&key {\it replace-existing? first?} \value{function}
\end{programexample}

Will call the function {\it fn} whenever {\it key} is hit.  If {\it first?}
then the accelerator will be tested before all interactors, otherwise
it will be tested if no interactor uses {\it key}.
{\it Replace-existing}, if non-NIL, will remove any other assignments
for {\it key}.  By using the default NIL value, you can temporarily hide
an accelerator binding.

\vspace{1 line}
\begin{group}
The function {\it fn} is called as:
\begin{programexample}
(lambda (event))
\end{programexample}
where event is the interactor event structure that caused the
accelerator to happen.
\end{group}

\index{remove-global-accelerator}
\index{remove-window-accelerator}
\begin{programexample}
inter:Remove-Global-Accelerator {\it key} \&key {\it remove-all? first?} \value{function}
inter:Remove-Window-Accelerator {\it win key} \&key {\it remove-all? first?} \value{function}
\end{programexample}
Removes the specified accelerator.  If {\it remove-all?} then removes
all the accelerators bound to the {\it key}, otherwise, just removes the
first one.

\index{clear-global-accelerators}
\index{clear-window-accelerators}
\index{default-global-accelerators}
\begin{programexample}
inter:Clear-Global-Accelerators \value{function}
inter:Clear-Window-Accelerators {\it win} \value{function}

inter:Default-Global-Accelerators {\it ;; sets up the default accelerators} \value{function}
\end{programexample}



\chapter{Slots of All Interactors}
\index{Slots (of interactors)}
\label{interslots}

This section lists all the slots common to all interactors.  Most of these
have been explained in the previous sections.  The slots a programmer is most
likely to want to change are listed first.
Some specific interactor types have additional slots, and these are
described in their sections.

The various \pr{-action} procedures are used by the individual interactors
to determine their behavior.  {\it You will rarely need to set these slots.}
See section \ref{customroutines} for how to use the \pr{-action} slots.

The following field {\it must} be supplied:
\begin{description}
\item[] \pr{:start-where}\index{start-where}
- where the mouse should be for this
interactor to start working.  Valid values for \pr{where} are described in
section \ref{where}.
\end{description}

The following fields are optional.  If they are not supplied, then the
default value is used, as described below.  Note that supplying NIL is
{\it not} the same as not supplying a value (since not supplying a value
means to use the default, and NIL often means to not do something).

\begin{description}
\item[] \pr{:window}\index{window}
- the window that the interactor should be connected to.  Usually this is
supplied as a single window, but other options are possible for interactors
that operate on multiple windows.  See section \ref{multiwindow}.

\item[] \pr{:start-event}\index{start-event}
- the event that causes the interactor to start
working.  The default value is \pr{:leftdown}.  NIL means the interactor
never starts by itself (see \ref{startinteractor}).
Using \pr{T} means no event, which
means that the interactor is operating whenever the mouse is over
\pr{:start-where}.  The full syntax for event
specification is described in section \ref{eventspec}.

\item[] \pr{:continuous}\index{continuous}
- if this is T, then the interactor operates continuously
from start-event until stop-event.  If it is NIL, then the interactor
operates exactly once when start-event happens.  The default value
is \pr{T}.  See section \ref{continuous} for more explanation.

\item[] \pr{:stop-event}\index{stop-event}
- This is not used if \pr{:continuous} is NIL.  If \pr{:continuous} is T,
\pr{:stop-event} is the
event that the interaction should stop on.  If not supplied, and the
start-event is a mouse down event (such as \pr{:leftdown}), then
the default \pr{:stop-event} is the corresponding up event (e.g. \pr{:leftup}).
If start-event is a keyboard key, the default stop event is \pr{\#{\tt\char`\\}RETURN}.  If
the \pr{:start-event}
is a list or a special form like \pr{:any-mousedown}, then the default
\pr{:stop-event} is calculated based on the actual start event used.
You only need to define
stop-event if you want some other behavior (e.g. starting on \pr{:leftdown}
and stopping on the next \pr{:leftdown} so you must click twice).  The form
for stop-events is the same as for start-events (see
section \ref{events}).  \pr{T} means no event, so the
interactor never stops (unless it is turned off using \pr{ChangeActive}).

\item[] \pr{:feedback-obj}\index{feedback-obj}
- If supplied, then this is the object to be used to show the feedback while the
interaction is running.  If NIL, then typically the object itself will be
modified.  The default value is NIL.  See the descriptions of the specific
interactors for more information.

\item[] \pr{:running-where}\index{running-where}
- Describes where the interaction should operate if it is continuous.  The
default is usually to use the same value as start-where.  Running-where
will sometimes
need to be different from start-where, however.  For example, with an
object that moves with the mouse,
you might want to start moving when the press was over the object itself.
See section \ref{where} for a complete discussion of this field.

\item[] \pr{:outside}\index{outside}
- Determines what to do when the mouse goes outside of running-where.
Legal values are \pr{:last}, which means to use the last value before the
mouse went outside, or NIL which
means to return to the original value (before the interaction
started).  The default value is NIL.
See section \ref{outside} for more explanation.

\item[] \pr{:abort-event}\index{abort-event}
- This is an event that causes the interaction to terminate prematurely.
If abort-event is NIL, then there is no separate event to cause aborts.
The default value is NIL.  The form for abort-events is the same as for
start-events (see section \ref{events}).

\item[] \pr{:waiting-priority}\index{waiting-priority}
- This determines the priority of the interactor while waiting for the
start event to happen.  See section \ref{priorities} for a description of
priority levels.

\item[] \pr{:running-priority}\index{running-priority}
- This determines the priority of the interactor while it is running
(waiting for the stop event to happen).  See section \ref{priorities} for a
description of priority levels.

\item[] \pr{:final-function}\index{final-function}
- This function is called after the interactor is complete.  The programmer
might supply a function here to cause the application to notice the users
actions.  The particular form for the parameters to this function is
specific to the particular type of the interactor.

\item[] \pr{:stop-action}\index{stop-action}
- This procedure is called once when the \pr{:stop-event} happens,
or if the interactor is {\it not} continuous, then this procedure is called once
when the \pr{:start-event} happens.  The form for the arguments is specific to the
particular interactor sub-class.   Specifying NIL means do no action.
Normally, the \pr{stop-action} procedure (as well as the \pr{start-action,
running-action, etc. below} is {\it not} provided by the programmer, but rather
inherited.  These functions provide the default behavior, such as turning
on and off the feedback object.  In particular the default stop-action
calls the final-function.  See section \ref{customroutines}.

\item[] \pr{:start-action}\index{start-action}
- The action to take place when start-event happens when the mouse is over
start-where
% \comment{** Wrong
% If \pr{:contininuous} is NIL, then this is the only procedure
% that is called.  If \pr{:contininuous} is T (the default), then
% \pr{:running-action}, \pr{:stop-action}, etc. will be called afterwards.
% **}
and continuous is T (if continuous is NIL, then \pr{stop-action}
is called when the start-event happens).
The form for the arguments is specific to the
particular interactor sub-class.
Specifying NIL means do no action.  See section \ref{customroutines}.

\item[] \pr{:running-action}\index{running-action}
- A procedure to be called as the interaction is running.  This is called
repeatedly (typically for each incremental mouse movement) while the mouse
is inside \pr{:running-where} and between when \pr{:start-event} and
\pr{:stop-event} happen.  The form for the arguments is specific to the
particular interactor sub-class.  Specifying NIL means do no action.
See section \ref{customroutines}.

\item[] \pr{:abort-action}\index{abort-action}
- This procedure is called when the interaction is aborted, either by
\pr{:abort-event} or \pr{:stop-event} while outside.
The form for the arguments is specific to the
particular interactor sub-class.  Specifying NIL means do no action.
See section \ref{customroutines}.

\item[] \pr{:outside-action}\index{outside-action}
- This procedure is called once each time the mouse goes from inside
\pr{:running-where} to being outside.  It is {\it not} called repeatedly while
outside (so it is different from \pr{:running-action}).
The form for the arguments is specific to the
particular interactor sub-class.  Specifying NIL means do no action.
See section \ref{customroutines}.

\item[] \pr{:back-inside-action}\index{back-inside-action}
- This is called once each time the mouse goes from outside
\pr{:running-where} to being inside.  Note that
\pr{:running-action} is {\it not} usually called on this point.
The form for the arguments is specific to the
particular interactor sub-class.  Specifying NIL means do no action.
See section \ref{customroutines}.

\item[] \pr{:active}\index{active}
- Normally, an interactor is active (willing to accept its start event)
from the time it is created until it is destroyed.  However, it is
sometimes convenient to make an interactor inactive, so it does not look
for any events, for example, to have different modes in the interface.
This can be achieved by setting the active field of the
interactor.  If the interactor is running, setting \pr{:active} to NIL
causes it to abort, and if the interactor is
not running, then this just keeps it from starting.
This field can be set and changed at any time either using
\pr{s-value} or by having a formula in this slot, but it is safest to use
the \pr{Change-active} procedure, since this guarantees that the interactor
will be aborted immediately if it is running.  Otherwise, if it is running
when the \pr{active} field changes to NIL, then it will abort the next time
there is an event (e.g., when the mouse moves).  See section \ref{active}
for more information.

\item[] \pr{:self-deactivate}\index{self-deactivate}
- Normally, interactors are always active.  If this field is \pr{T}
however, the interactor will become inactive after
it runs once (it will set its own \pr{:active} slot to NIL).
The interactor will then not run again until the \pr{:active}
field is explicitly set to \pr{T}.  If this field is used, it is probably a
bad idea to have a formula in the \pr{:active} slot.

% \begin{comment}
% ************************** NIY **********************************
% \pr{:exception-p}\index{exception}
% - a function to determine whether the current value is illegal.  This
% serves as a temporary way to change what is specified by the \pr{-where}
% parameters.  Illegal
% values add one extra state to the state machine of the figure; the
% interactor goes into the exception state when the mouse goes over an
% illegal item (as determined by this function), and leaves the exception
% state when the mouse goes over a legal item or goes outside.  {\bf Note: Is
% it sufficient to just use the actions for outside and back-inside for
% exceptions, or do we need the following?}
% There are two procedures added for exceptions: \pr{:over-illegal-item} and
% \pr{:leave-illegal-item}, see below.
% Parameters to the exception-p procedure
% are \pr{(an-interactor objUnderMouse)} and it should
% return T or NIL.  Returning \pr{T} means that the value is an exception (is
% illegal), and this is treated as if the mouse went outside.  Returning NIL
% means the value is OK.  The default is for there to be no illegal values.
% The default function for this parameter (use this by supplying T) checks the
% slot \pr{:illegal} in the object under the mouse.
% {\bf This is not implemented yet.}
% 
% \pr{:over-illegal-item}\index{over-illegal-item}
% - Called when go over an illegal item with parameters \pr{(an-interactor
% illegal-obj)}. Default action is to call the \pr{:outside-action} procedure.
% {\bf This is not implemented yet.}
% 
% \pr{:leave-illegal-item}\index{leave-illegal-item}
% - Called when no longer over an illegal item with parameters \pr{(an-interactor
% illegal-obj)}. Default action is to do nothing.
% Note that if the mouse moves from one illegal item to another,
% the \pr{:leave-} and \pr{:over-} procedures will be called.  Also unlike
% back-inside, the \pr{:running-action} procedure is called on the new object
% the mouse goes over.  {\bf This is not implemented yet.}
% 
% \pr{:pop-up}\index{pop-up}
% - If non-NIL, then this interactor controls something that should pop-up
% (become visible) when the interactor starts, and become invisible when the
% interactor completes.
% If NIL, then object that the interactor refers to should be
% visible.  If non-NIL, then the value of pop-up should be a procedure to
% cause the object to be displayed and erased.  It is called with
% \pr{(an-interactor visible-or-erase)}, where \pr{visible-or-erase} is T to
% make it visible, and NIL to erase it.
% The default procedure is to set
% the object's \pr{:visible} slot to T, and to NIL for
% stop-action.  If would be an error to have start-where be the object if it
% was a pop-up, since it would not be visible to press on.  The default value
% for this parameter is NIL.  {\bf This is not implemented yet.}
% ************************** NIY **********************************
% \end{comment}
\end{description}

\string{titstring=`\value{chapter} Specific Interactors'}
\chapter{Specific Interactors}
\label{specificinters}

This section describes the specific interactors that have been defined.
Below is a list of the interactors, and then the following sections
describe them in more detail.  There are also several interactors defined for
the \pr{multifont-text} object.  These are described in the Opal manual.

\begin{description}
\item[] \pr{Inter:Menu-Interactor} \index{Menu-Interactor}
- to handle menu items, where the mouse can choose
among a set of items.  Useful for menus, etc.

\item[] \pr{Inter:Button-Interactor} \index{Button-Interactor}
- to choose a particular button.  The difference
from menus is that when the mouse moves away, the item is deselected,
rather than having a different item selected.  Useful for sets of buttons
like `radio buttons' and `check boxes', and also for single, stand-alone buttons.
This can also be used just to select an object by making
\pr{:continuous} be NIL.

\item[] \pr{Inter:Move-Grow-Interactor} \index{Move-Grow-Interactor}
- move or change the size of
an object or one of a set of objects using the mouse.
There may be feedback to show how the object moves or grows, or the
object itself may
change with the mouse.  If defined over a set of objects, then the
interactor gets the
object to change from where the interaction starts.  Useful for scroll
bars, horizontal and vertical gauges, and for moving and changing
the size of application objects in a graphics editor.  It can change the
bounding box for the objects or the end points for a line.

\item[] \pr{Inter:Two-Point-Interactor} \index{Two-Point-Interactor}
- This is used when there is no original object to
modify, but one or two new points are desired.
A rubber-band feedback object (usually a
rubber-band line or rectangle) will typically be
drawn based on the points specified.

\item[] \pr{Inter:Angle-Interactor} \index{Angle-Interactor}
- Useful for getting the angle the mouse moves from
around some point.  This can be used for circular gauges or for `stirring
motions' for rotating.

\item[] \pr{Inter:Text-Interactor} \index{Text-Interactor}
- Used to input a small edited string of text.  The
text can be one line or multi-line.

\item[] \pr{Inter:Gesture-Interactor} \index{Gesture-Interactor}
- Used to recognize single-path gestures drawn with the mouse.

\item[] \pr{Inter:Animator-Interactor} \index{Animator-Interactor}
- This interactor causes a function to be executed at regular intervals,
allowing rapid updating of graphics for animation.

\end{description}

\vspace{2 lines}
The following interactors are planned but not implemented yet.
\begin{description}
\item[] \pr{Inter:Trace-Interactor} \index{Trace-Interactor}
- This returns all of the points the mouse goes
through between \pr{start-event} and \pr{stop-event}.  This is useful for
inking in a drawing program.  Although this isn't implemented yet, it is
trivial to use a gesture interactor with a \pr{:classifier} of NIL.

\item[] \pr{Inter:Multi-Point-Interactor} \index{Multi-Point-Interactor}
- This is used when there is no original object to
modify, but more than 2 new points are desired.  This is separate from the
\pr{two-point-interactor} because the way the points are stored is usually
different, and the stopping conditions are much more complicated for
multi-points.   {\bf Not implemented yet.  However, there is a gadget in the
gadget set that will do most of this.  See}
\pr{garnet-gadgets:polyline-creator}.

\end{description}


\begin{group}
\section{Menu-Interactor}
\index{Menu-Interactor}
\label{MENUINTER}
\begin{programexample}
(create-instance 'inter:Menu-Interactor inter:interactor
  {\it ;; Slots common to all interactors (see section \ref{interslots})}
  (:start-where NIL)
  (:window NIL)
  (:start-event :leftdown)
  (:continuous T)
  (:stop-event NIL)
  (:running-where NIL)
  (:outside NIL)
  (:abort-event :control-{\tt\char`\\}g)
  (:waiting-priority normal-priority-level)
  (:running-priority running-priority-level)
  (:active T)
  (:self-deactivate NIL)

  {\it ; Slots specific to the menu-interactor (discussed in this section)}
  (:final-function NIL)      {\it ; (lambda (inter final-obj-over))}
  (:how-set :set)            {\it ; How to select new items (toggle selection, etc.)}
  (:feedback-obj NIL)        {\it ; Optional interim feedback object.  The inter will set this object's :obj-over slot.}
  (:final-feedback-obj NIL)  {\it ; The optional object to indicate the final selection}
  (:slots-to-set             {\it ; Names of slots to set in the objects}
   '(:interim-selected       {\it ;   '({\tt\char`\<}interim-selected-slot-name-in-obj>}
     :selected               {\it ;     {\tt\char`\<}selected-slot-name-in-obj>}
     :selected))             {\it ;     {\tt\char`\<}selected-slot-name-in-aggregate>)}
  (:final-feed-inuse NIL)    {\it ; Read-only slot.  A list of final feedback objects (section \ref{menufinalfeedbackobj})}

  {\it ; {\bf Advanced feature:}  Read-only slots.}
  {\it ; See section \ref{specialslots} for details about these slots.}
  (:first-obj-over NIL)     {\it ; Read-only slot.  The object returned from the start-where.}
  (:current-window NIL)     {\it ; Read-only slot.  The window of the last (or current) event.}
  (:start-char NIL)         {\it ; Read-only slot.  The character or keyword of the start event.}

  {\it ; {\bf Advanced feature:} Customizable action routines.}
  {\it ; See sections \ref{interslots} and \ref{menucustomaction} for details about functions in these slots.}
  (:start-action ...)       {\it ; (lambda (inter first-obj-under-mouse))}
  (:running-action ...)     {\it ; (lambda (inter prev-obj-over new-obj-over))}
  (:stop-action ...)        {\it ; (lambda (inter final-obj-over))}
  (:abort-action ...)       {\it ; (lambda (inter last-obj-over))}
  (:outside-action ...)     {\it ; (lambda (inter outside-control prev-obj-over))}
  (:back-inside-action ...) {\it ; (lambda (inter outside-control prev-obj-over new-obj-over))}
...)
\end{programexample}
\end{group}

(Note: If you just want to use a pre-defined menu, it may be sufficient to use
one of the menu objects in the Garnet Gadget Set.)

The menu interactor is used (not surprisingly) mostly for menus.
There is
typically some feedback to show where the mouse is while the interactor is
running.  This is called the {\it interim feedback}.  A separate kind of
feedback might be used to show the final object selected.  This is called
the {\it final feedback}.

Unlike button interactors (see section \ref{buttoninter}), Menu-interactors
allow the user to move from one item to another while the interactor is
running.  For example, the user can press over one menu item, move the
mouse to another menu item, and release, and the second item is the one
that is selected.

There are a number of examples of the use of menu interactors below.  Other
examples can be found in the \pr{menu} gadget in the Garnet
Gadget Set, and in the file \pr{demo-menu.lisp}.

\subsection{Default Operation}

This section describes how the menu interactor works
if the programmer does not remove or override any of the standard
\pr{-action} procedures.  To supply custom action procedures, see section
\ref{Menucustomaction}.

The menu interactor provides many different ways to control how the
feedback graphics are controlled.  In all of these, the interactor sets
special slots in objects, and the graphics must have formulas that depend
on these slots.

\paragraph{Interim Feedback}
\index{Interim Feedback (for menus)}
\label{menuinterimfeedback}
\index{Obj-Over (slot)}
\index{Interim-Selected (slot)}

To signify the object that the mouse is over as {\it interim feedback} (while
the interactor is running), menu-interactors set two different slots.  If
there is a feedback object supplied in the \pr{:feedback-obj} slot of the
interactor, then the \pr{:obj-over} slot of the feedback object is set to
the current menu item
object.  Also, the \pr{:interim-selected} slot of the current menu item is
set to T, and the  \pr{:interim-selected} slots of all other items are set
to NIL.  Note: there is always at most one interim-selected object,
independent of the value of the \pr{:how-set} slot.

This supports two different ways to handle interim feedback:

{\bf A single feedback object.}

\index{feedback-obj}
\index{Obj-Over (slot)}
\index{Visible (slot)}
This object should be supplied in the
\pr{:feedback-obj} slot of the interactor.  The \pr{:obj-over} slot of this
object is set to the menu item that the feedback should appear over, or NIL
if there is no object.  The following is an example of a typical reverse-video black
rectangle as a feedback object:
\index{feedback-rect}
\indexsecondary{Primary=`Examples', Secondary=`Feedback Rectangle'}
\indexsecondary{Primary=`Examples', Secondary=`Obj-Over (slot}')
\begin{programexample}
(create-instance 'FEEDBACK-RECT opal:rectangle
   (:obj-over NIL) ;{\it set by the interactor}
   (:visible (o-formula (gvl :obj-over))) ;{\it this rectangle is visible}
                                          ;{\it only if over something}
   (:left (o-formula (gvl :obj-over :left)))
   (:top (o-formula (gvl :obj-over :top)))
   (:width (o-formula (gvl :obj-over :width)))
   (:height (o-formula (gvl :obj-over :height)))
   (:fast-redraw-p T)
   (:draw-function :xor)
   (:filling-style opal:black-fill)
   (:line-style NIL))
\end{programexample}

The interactor to use it would be something like:
\indexsecondary{Primary=`Examples', Secondary=`Menu Interactor'}
\begin{programexample}
(create-instance 'SELECT-INTER Inter:Menu-Interactor
   (:start-where `(:element-of ,ITEMSAGG))
   (:feedback-obj FEEDBACK-RECT)
   (:window MYWINDOW))
\end{programexample}
The items that can be chosen are elements of an aggregate named ITEMSAGG.

{\bf Multiple feedback objects.}

\index{interim-selected}
\index{Interim-Selected (slot)}
In this case, each item of the menu might have its own feedback object, or
else some property of that menu item object might change as the mouse moves
over it.  Here, you would have formulas that depended on the
\pr{:interim-selected} slot of the menu item.

\vspace{1 line}
\begin{group}
If there are separate objects associated with each menu item that will be
the interim feedback, then their visibility slot can simply be tied to the
\pr{:interim-selected} slot.  An example using an Aggregadget which is
the item-prototype for an AggreList (see the Aggregadgets manual) with an
embedded interactor is:
\index{Aggregadget}\index{Aggrelist}
\indexsecondary{Primary=`Examples', Secondary=`Aggregadget'}
\indexsecondary{Primary=`Examples', Secondary=`Aggrelist'}
\indexsecondary{Primary=`Examples', Secondary=`Menu'}
\begin{programexample}
(create-instance 'MYMENU opal:aggrelist
   (:items '(`One' `Two' `Three'))
   (:item-prototype
    `(opal:aggregadget
      (:width ,(o-formula (gvl :str :width)))
      (:height ,(o-formula (gvl :str :height)))
      (:my-item ,(o-formula (nth (gvl :rank) (gvl :parent :items))))
      (:parts
       `((:str ,opal:text
               (:string ,(o-formula (gvl :parent :my-item)))
               (:left ,(o-formula (gvl :parent :left)))
               (:top ,(o-formula (gvl :parent :top))))
         (:interim-feed ,opal:rectangle
          ;;{\it The next slot causes the feedback to go on at the right time}
                        (:visible ,(o-formula (gvl :parent :interim-selected)))
                        (:left ,(o-formula (gvl :parent :left)))
                        (:top ,(o-formula (gvl :parent :top)))
                        (:width ,(o-formula (gvl :parent :width)))
                        (:height ,(o-formula (gvl :parent :height)))
                        (:fast-redraw-p T)
                        (:draw-function :xor)
                        (:filling-style ,opal:black-fill)
                        (:line-style NIL))))))
   (:interactors
    `((:inter ,Inter:Menu-Interactor
              (:start-where ,(o-formula (list :element-of (gvl :operates-on))))
              (:window ,MYWINDOW)))))
\end{programexample}
\end{group}

\paragraph{Final Feedback}
\index{Final Feedback (for menus)}
\label{menufinalfeedback}

\index{selected}
For some menus, the application just wants to know which item was selected,
and there is no graphics to show the final selection.  In other cases,
there should be {\it final feedback} graphics to show the object the mouse
ends up on.

\index{Selected (slot)}
The Menu-Interactor supplies three ways to have graphics (or applications)
depend on the final selection.  Both the \pr{:selected} slot of the
individual item and the \pr{:selected} slot of the aggregate
the items are in are set.  The item's \pr{:selected} slot is set with T or NIL, as
appropriate, and the aggregate's \pr{:selected} slot is set with the
particular item(s) selected.  The number of items that are
allowed to be selected is controlled by the \pr{:how-set} slot of the
interactor, as described in section \ref{menuhowset}.

Note that the aggregate's \pr{:selected} slot often contains a list of
object names, but the \pr{:selected} slot in the individual items will
always contain T or NIL.
The programmer is responsible for setting up constraints so that the
appropriate final feedback is shown based on the \pr{:selected} field.

If there is no aggregate (because \pr{:start-where} is
something like \pr{(:in xxx)} rather than something
like \pr{(:element-of xxx)}), then the slot of the object is set with T or
NIL.  If the the \pr{:start-where} is one of the ``list'' styles (e.g.
\pr{(:list-element-of obj slot)}, then the
\pr{:selected} slot of the object the list is stored in (here, \pr{obj}) is
set as if that was the aggregate.

The third way to show the final feedback is to use the
\pr{:final-feedback-obj} slot, which is described in the next section.

\paragraph{Final Feedback Objects}
\index{final-feedback-obj}
\label{menufinalfeedbackobj}

The \pr{:feedback-obj} slot can be used for the object to show the
interim-feedback, and the \pr{:final-feedback-obj} slot can be used to hold
the object to show the final feedback.  Garnet will set the \pr{:obj-over} slot
of this object to the object that the interactor finishes on.  If the
\pr{:how-set} field of the interactor is one of the \pr{:list-*} options,
then there might be {\it multiple} final feedback objects needed to show all
the objects selected.  In this case, the interactor creates instances of
object in the \pr{:final-feedback-obj} slot.  Therefore, this object should
{\it not} be an {\it aggregate}; it must be an {\it aggregadget} instead (or
it can be a
single Opal object, such as a rectangle, circle, polyline, etc.).
Furthermore, the
final-feedback object itself should not be a \pr{:part} of an aggregadget,
since you are not allowed to add new objects to an aggregadget with parts.

The \pr{:final-feedback-obj} slot may contain a formula, which might compute
the appropriate feedback
object based on the object selected.  The interactor will
automatically duplicate the appropriate feedback object if more than
one is needed (e.g., if \pr{:how-set} is \pr{:list-toggle}).   One use
of this is to have different kinds of feedback for different kinds of
objects, and another would be to have different feedback objects in
different windows, for an interactor that works across multiple windows.
\index{current-obj-over} \index{current-window}
To aid in this computation, the \pr{:current-obj-over} slot of the
interactor is set with the object the mouse was last over, and the
\pr{:current-window} slot of the interactor is maintained with the
window of the current event.

If the start-where is one of the \pr{...-or-none} forms, then whenever the
user presses in the background, the final feedback objects are all turned off.

For examples of the use of final-feedback-objects, see MENU1 (the month menu)
or MENU2 (the day-of-the-week menu) in \pr{demo-menu.lisp}.

{\bf Useful Functions}

In order to help with final feedback objects, there are a number of
additional, useful functions.  To get the final-feedback objects currently
being displayed by an interactor, you can use:

\index{Return-Final-Selection-Objs}
\begin{programexample}
inter:Return-Final-Selection-Objs {\it inter}\value{function}
\end{programexample}


If you want to reference the current final feedback objects in a
{\it formula}, however,
then you should access the \pr{:final-feed-inuse} slot of the menu
interactor.  This slot contains
a list of the final feedback objects that are in use.  {\it Do not set
this slot}.
This might be useful if you wanted to use the final feedback objects as the
start objects for another interactor (e.g, one to move the object selected
by a final-feedback object):
\index{final-feed-inuse}
\indexsecondary{Primary=`Examples', Secondary=`Final Feedback Objs'}
\begin{programexample}
(create-instance NIL Inter:Move-Grow-Interactor
   ;; {\it start when press on a final-feedback object of SELECT-INTER}
   (:start-where (formula `(list :list-element-of
                                 `,SELECT-INTER :final-feed-inuse)))
   ;;{\it actually move the object which the feedback objects are over.}
   (:obj-to-change (o-formula (gvl :first-obj-over :obj-over)))
   ..... ; {\it all the other slots}
   )
\end{programexample}

\index{SelectObj}
If a program wants to make an object be selected, it can call:
\begin{programexample}
inter:SelectObj {\it inter obj}\value{function}
\end{programexample}
which will cause the object to become selected.  This uses the
\pr{:how-set} slot of the interactor to decide whether to deselect the
other objects (whether single or multiple objects can be selected).  The
\pr{:selected} slots of the object and the aggregate are set, and the
final-feedback objects are handled appropriately.  To de-select an object, use:
\index{DeSelectObj}
\begin{programexample}
inter:DeSelectObj {\it inter obj}\value{function}
\end{programexample}


\paragraph{Items Selected}
\label{menuhowset}
\index{Single selection}
\index{Multiple selection}

The menu interactor will automatically handle control over the {\it number}
of items selected.  A slot of the interactor (\pr{:how-set}) determines
whether a single item can be selected or multiple items.  In addition, this
slot also determines how this interactor will affect the selected items.
For example, if multiple items can be selected, the most common option is
for the interactor to ``toggle'' the selection (so if the item under the
mouse was selected, it becomes de-selected, and if it was not selected,
then it becomes selected).  Another design might use two interactors: one
to select items when the left button is pressed, and another to de-select items
when the right button is pressed.  The \pr{:how-set} slot provides for all
these options.

\index{How-set}
In particular, the legal values for the \pr{:how-set} slot are:
\begin{description}
\item[] \pr{:set} \index{set}\index{Selected (slot)}
- Select the final item.  One item is selectable at a time.
The aggregate's \pr{:selected} slot is set with
this object.  The item's \pr{:selected} slot is set with T.

\item[] \pr{:clear} \index{clear}
- De-select the final item.  At most one item is selectable at a time.
The aggregate's \pr{:selected} slot is set to NIL.
(If some item other than the final item used to be selected, then that
other item becomes de-selected.  I.e., using \pr{:clear} always causes
there to be no selected items.)
The item's \pr{:selected} slot is set to NIL.  (This choice for how-set is
mainly useful when the menu item contains a single item that can be turned
on and off by different
interactors\dashe.g., left button turns it on and right button turns it
off.  With a set of menu items, \pr{:set} is usually more appropriate.)

\item[] \pr{:toggle} \index{toggle}
- Select if not selected, clear if selected.  At most one item
is selectable at a time.  This means that if there are a set of objects and
you select the object that used to be selected, then there becomes no
objects selected.   (This is mainly useful when
there is a single button that can be turned on and off by one
interactor\dasheach press changes the state.  With a set of menu items,
\pr{:list-toggle} or \pr{:set} is usually more appropriate.  However, this
option could be used
with a set of items if you wanted to allow the user to make there be
{\it no} selection.)

\item[] \pr{:list-add} \index{list-add}
- If not in list of selected items, then add it.  Multiple
items are selectable at a time.  The item is added to the aggregate's
\pr{:selected} slot using \pr{pushnew}.
The item's \pr{:selected} slot is set with T.

\item[] \pr{:list-remove} \index{list-remove}
- If in list of selected items, then remove it.  Multiple
items selectable at a time.  The item is removed from the aggregate's
\pr{:selected} slot.  The item's \pr{:selected} slot is set with NIL.

\item[] \pr{:list-toggle} \index{list-toggle}
- If in list of selected items, then remove it, otherwise
add it.  Multiple items are selectable at a time.  The item is removed or
added to the aggregate's \pr{:selected} slot.  The
item's \pr{selected} slot is set with T or NIL.

\item[] {\tt\char`\<}a number{\tt\char`\>} \index{numbers (used in :how-set slot)}
- Increment the \pr{:selected} slot of the item by that amount
(which can be negative).  The aggregate's \pr{:selected} slot is set to
this object.  The value of the item's selected slot should be a number.

\item[] {\tt\char`\<}a list of two numbers{\tt\char`\>}: \pr{(inc mod)} - Increment the
\pr{:selected} slot of the item by the \pr{car} of the list, modulus the
\pr{cadr} of the list.  The aggregate's \pr{:selected} slot is set to
this object.  The value of the item's selected slot should be a number.
\end{description}

The default value for \pr{:how-set} for menus is \pr{:set}, so one item is
selected at a time.

\paragraph{Application Notification}
\label{menufinalfunc}
\index{Selected (slot)}
To have an application notice the effect of the menu-interactor, you can
simply have some slot of some object in the application contain a formula
that depends on the aggregate's \pr{:selected} slot.

Alternatively, the programmer can provide a function to be called when the
interactor is complete by putting the function in the \pr{:final-function}
slot.  This function is called with the following arguments:
\indexsecondary{Primary=`final-function', Secondary=`Menu-Interactor'}
\begin{programexample}
(lambda (an-interactor final-obj-over))
\end{programexample}

\paragraph{Normal Operation}
\index{Selected (slot)}

If the value of \pr{:continuous} is T,
then when the start event happens, the interim feedback is turned on, as
described in section \ref{menuinterimfeedback}.  If the mouse moves to a
different menu item, the interim feedback is changed to that item.  If the
mouse moves outside, the interim feedback is turned off, unless \pr{:outside} is
\pr{:last} (see section \ref{outside}).  If the interactor
aborts, the interim feedback is turned off.  When the stop event happens, the
interim feedback is turned off, and the final \pr{:selected} slots are set
as described in section \ref{menufinalfeedback} based on the value of the
\pr{:how-set} parameter (section \ref{menuhowset}), then the \pr{:obj-over}
field of the final-feedback-obj is set to the final selection (possibly
after creating a new final feedback object, if necessary), as described in
section \ref{menufinalfeedbackobj}.  Then the final-function
(if any) is called (section \ref{menufinalfunc}).

If the interactor is {\it not} continuous, when the start event happens, the
\pr{:selected} slots are set based on the value of the \pr{:how-set} parameter,
the \pr{:obj-over} slot of the final-feedback-obj is set, and then the
final-function is called.


\subsection{Slots-To-Set}
\label{slots-to-set}
\index{slots-to-set}
The button and menu interactors by default set the \pr{:selected}
and \pr{:interim-selected} slots of objects.  This sometimes results
in a conflict when two interactors are attached to the same object.
Therefore, the \pr{:slots-to-set} slot has been provided in which you
may specify what slot names should be used.  Note: it is very
important that once an interactor is started, the slot names for it
should never change.

The \pr{:slots-to-set} slot takes a list of three values:
\begin{programexample}
 ({\tt\char`\<}interim-selected-slot-name-in-obj{\tt\char`\>}
  {\tt\char`\<}selected-slot-name-in-obj{\tt\char`\>}
  {\tt\char`\<}selected-slot-name-in-aggregate{\tt\char`\>} )
\end{programexample}
The default value is \pr{(:interim-selected :selected :selected)}.  If
NIL is supplied for any slot name, then that slot isn't set by the
interactor.

The slots in the object are
set with T or NIL, and the slot in the aggregate is set with the
selected object or a list of the selected objects.


\begin{group}
\section{Button-Interactor}
\index{Button-Interactor}
\label{buttoninter}

\begin{programexample}
(create-instance 'inter:Button-Interactor inter:interactor
  {\it ;; Slots common to all interactors (see section \ref{interslots})}
  (:start-where NIL)
  (:window NIL)
  (:start-event :leftdown)
  (:continuous T)
  (:stop-event NIL)
  (:running-where '(:in *))
  (:outside NIL)
  (:abort-event :control-{\tt\char`\\}g)
  (:waiting-priority normal-priority-level)
  (:running-priority running-priority-level)
  (:active T)
  (:self-deactivate NIL)

  {\it ; Slots common to the menu-interactor and the button-interactor (see section \ref{menuinter})}
  (:final-function NIL)      {\it ; (lambda (inter final-obj-over))}
  (:how-set :list-toggle)    {\it ; How to select new items (toggle selection, etc.)}
  (:feedback-obj NIL)        {\it ; Optional interim feedback object.  The inter will set this object's :obj-over slot.}
  (:final-feedback-obj NIL)  {\it ; The optional object to indicate the final selection}
  (:slots-to-set             {\it ; Names of slots to set in the objects}
   '(:interim-selected       {\it ;   '({\tt\char`\<}interim-selected-slot-name-in-obj>}
     :selected               {\it ;     {\tt\char`\<}selected-slot-name-in-obj>}
     :selected))             {\it ;     {\tt\char`\<}selected-slot-name-in-aggregate>)}
  (:final-feed-inuse NIL)    {\it ; Read-only slot.  A list of final feedback objects (section \ref{menufinalfeedbackobj})}

  {\it ; Slots specific to the button-interactor (discussed in this section)}
  (:timer-repeat-p NIL)        {\it ; when T, then does timer}
  (:timer-initial-wait 0.75)   {\it ; time in seconds}
  (:timer-repeat-wait 0.05)    {\it ; time in seconds}

  {\it ; {\bf Advanced feature:}  Read-only slots.}
  {\it ; See section \ref{specialslots} for details about these slots.}
  (:first-obj-over NIL)     {\it ; Read-only slot.  The object returned from the start-where.}
  (:current-window NIL)     {\it ; Read-only slot.  The window of the last (or current) event.}
  (:start-char NIL)         {\it ; Read-only slot.  The character or keyword of the start event.}

  {\it ; {\bf Advanced feature:} Customizable action routines.}
  {\it ; See sections \ref{interslots} and \ref{buttoncustomaction} for details about functions in these slots.}
  (:start-action ...)       {\it ; (lambda (inter obj-under-mouse))}
  (:stop-action ...)        {\it ; (lambda (inter final-obj-over))}
  (:abort-action ...)       {\it ; (lambda (inter last-obj-over))}
  (:outside-action ...)     {\it ; (lambda (inter last-obj-over))}
  (:back-inside-action ...) {\it ; (lambda (inter new-obj-over))}
...)
\end{programexample}
\end{group}

\vspace{1 line}
(Note: If you
just want to use a pre-defined set of buttons, it may be sufficient to use
the radio buttons or x-button objects from the Garnet Gadget Set.

The button interactor is used (not surprisingly) mostly for buttons.  There is
typically some feedback to show where the mouse is while the interactor is
running.  This is called the {\it interim feedback}.  A separate kind of
feedback might be used to show the final object selected.  This is called
the {\it final feedback}.

Unlike menu interactors (see section \ref{menuinter}), Button-interactors
do not allow the user to move from one item to another while the interactor is
running.  For example, if there are a group of buttons, and the user
presses over one button, moving to a different button in the set does
{\it not} cause the other button to become selected.  Only the first button
that the user presses over can be selected.  This is similar to the way
radio buttons and check boxes work on the Macintosh.

There are a number of examples of the use of button interactors below.  Other
examples can be found in the demos for the \pr{radio-button} and \pr{x-button}
gadgets in the Garnet Gadget Set, and in the file \pr{demo-grow.lisp}.

\subsection{Default Operation}

The button interactor works very similar to the menu interactor (section
\ref{menuinter}).
This section describes how the button interactor works
if the programmer does not remove or override any of the standard
\pr{-action} procedures.  To supply custom action procedures, see section
\ref{buttoncustomaction}.

The button interactor provides the same ways to control the feedback as the
menu interactor.

% \begin{comment}
% *******************************************************************
% \end{comment}
\paragraph{Interim Feedback}
\index{Interim Feedback (for buttons)}
\label{buttoninterimfeedback}

\index{Obj-Over (slot)}
\index{Interim-Selected (slot)}
As with menus, button-interactors set both the \pr{:obj-over} slot of
the object in the \pr{:feedback-obj} slot, and the \pr{:interim-selected}
slot of the current button item.  The \pr{:obj-over} slot is set with the
object that is under the mouse or NIL if none, and the
\pr{:interim-selected} slot is set with T or NIL.  See section
\ref{menuinterimfeedback} for more information.

\paragraph{Final Feedback}
\label{buttonfinalfeedback}
\index{Final Feedback (for buttons)}
\index{Selected (slot)}

\index{selected}
The final feedback for buttons works the same way as for menus:
Both the \pr{:selected} slot of the
individual item and the \pr{:selected} slot of the aggregate
the items are in are set, and the \pr{:obj-over} slot of the object in the
\pr{:final-feedback-obj} slot (if any) is set.
The item's \pr{:selected} slot is set with T or NIL, as
appropriate, and the aggregate's \pr{:selected} slot is set with the
name(s) of the particular item(s) selected.

For more information, see sections \ref{menufinalfeedback} and
\ref{menufinalfeedbackobj}.

\paragraph{Items Selected}
\label{buttonhowset}

As with Menus, the button interactor will automatically handle control over
the {\it number}
of items selected.  A slot of the interactor (\pr{:how-set}) determines
whether a single item can be selected or multiple items.  In addition, this
slot also determines how this interactor will affect the selected items.

The legal values for \pr{:how-set} are exactly the same as for menu (see
section \ref{menuhowset}:
\pr{:set}, \pr{:clear}, \pr{:toggle}, \pr{:list-add}, \pr{:list-remove},
\pr{:list-toggle}, a number, or a list of two numbers).

The default for buttons is \pr{:list-toggle}, however.


\paragraph{Application Notification}
\label{buttonfinalfunc}

\index{Selected (slot)}
As with menus, to have an application notice the effect of the
button-interactor, you can
simply have some slot of some object in the application contain a formula
that depends on the aggregate's \pr{:selected} slot.

Alternatively, the programmer can provide a function to be called when the
interactor is complete by putting the function in the \pr{:final-function}
slot.  This function is called with the following arguments:
\indexsecondary{Primary=`final-function', Secondary=`Button-Interactor'}
\begin{programexample}
(lambda (an-interactor final-obj-over))
\end{programexample}

\paragraph{Normal Operation}

% \begin{comment}
% *******************************************************************
% \end{comment}
\index{Selected (slot)}
If the value of \pr{:continuous} is T,
then when the start event happens, the interim feedback is turned on, as
described in section \ref{buttoninterimfeedback}.  If the mouse moves away
from the item it starts on, the interim feedback goes off.  If the mouse
moves back, the interim feedback goes back on.  If the interactor
aborts, the interim feedback is turned off.  When the stop event happens,
the interim feedback is turned off.
If the mouse is over the item that the interactor started on, the
final \pr{:selected} slots are set
as described in section \ref{buttonfinalfeedback} based on the value of the
\pr{:how-set} parameter (section \ref{buttonhowset}), then the \pr{:obj-over}
field of the final-feedback-obj is set to the final selection (possibly
after creating a new final feedback object, if necessary), as described in
section \ref{menufinalfeedbackobj}.  Then the final-function
(if any) is called (section \ref{buttonfinalfunc}).  Otherwise, when the
stop event happens, the interactor aborts.

The \pr{:last} parameter is ignored by button interactors.

If the interactor is {\it not} continuous, when the start event happens, the
\pr{:selected} slots are set based on the value of the \pr{:how-set} parameter,
the \pr{:obj-over} slot of the final-feedback-obj is set, and then the
final-function is called.

The \pr{:slots-to-set} slot can be used to change the name of the slots
that are set, as described in section \ref{slots-to-set}.



\subsection{Auto-Repeat for Buttons}
\index{Auto-Repeat}

The \pr{button-interactor} can auto-repeat the \pr{:final-function}.
{\it Note: This only works for Allegro, LispWorks, and Lucid lisps
(including Sun and HP CL); {\bf not} for CMU CL, AKCL, etc.}

\index{timer-repeat-p}
\index{timer-initial-wait}
If \pr{:timer-repeat-p} is non-NIL, then after the interactor starts,
if the mouse button is held down more than \pr{:timer-initial-wait}
seconds, then every \pr{:timer-repeat-wait} seconds, the
\pr{:final-function} is called and the appropriate slot (usually
\pr{:selected}) is set into the object the interactor is operating
over (this might be useful, for example, if the \pr{:how-set} was an integer to
cause the value of the \pr{:selected} slot to increment each time).

The various scroll bar and slider gadgets use this feature to cause the
arrows to auto repeat.



\begin{group}
\subsection{Examples}
\paragraph{Single button}
\index{Incrementing Button}
\indexsecondary{Primary=`Examples', Secondary=`Incrementing Button'}
\indexsecondary{Primary=`Examples', Secondary=`Button'}
The button in this example is not continuous, and does not have a final
feedback; it just causes a value to be incremented.
\begin{programexample}
(create-instance 'ARROW-INC opal:aggregadget
   (:parts
    `((:arrow ,opal:polyline
              (:selected 10)
              (:point-list (20 40 20 30 10 30 25 15 40 30 30 30 30 40 20 40)))
      (:label ,opal:text
              (:left 17)(:top 50)
              (:string ,(o-formula (prin1-to-string
				    (gvl :parent :arrow :selected)))))))
   (:interactors
    `((:incrementor ,Inter:Button-Interactor
                    (:continuous NIL)
                    (:start-where ,(o-formula (list :in (gvl :operates-on :arrow))))
                    (:window ,MYWINDOW)
                    (:how-set 3)))))  ; increment by 3
\end{programexample}
\end{group}

\paragraph{Single button with a changing label}
\index{Changing Label Button}
\indexsecondary{Primary=`Examples', Secondary=`Changing Label Button'}
\indexsecondary{Primary=`Examples', Secondary=`Button'}
Here we have an object whose label changes every time the mouse is pressed
over it.  It cycles through a set of labels.
This interactor is not continuous, so the action happens immediately on the
down-press and there is no feedback object.

\begin{programexample}
(create-instance 'CYCLE-STRING opal:aggregadget
  (:parts
   `((:label ,opal:text
             (:left 10)(:top 80)
             (:selected 0)
             (:choices (`USA' `Japan' `Mexico' `Canada'))
             (:string ,(o-formula (nth (gvl :selected) (gvl :choices)))))))
  (:interactors
   `((:incrementor ,Inter:Button-Interactor
	           (:continuous NIL)
		   (:start-where
		     ,(o-formula (list :in (gvl :operates-on :label))))
		   (:window ,MYWINDOW)
		   ;; {\it use a list of 2 numbers and interactor will do MOD}
		   (:how-set
                     ,(o-formula (list 1 (length (gvl :operates-on
                                                      :label :choices)))))))))
\end{programexample}



\begin{group}
\section{Move-Grow-Interactor}
\index{Move-Grow-Interactor}
\label{MOVEGROWINTER}

\begin{programexample}
(create-instance 'inter:Move-Grow-Interactor inter:interactor
  {\it ;; Slots common to all interactors (see section \ref{interslots})}
  (:start-where NIL)
  (:window NIL)
  (:start-event :leftdown)
  (:continuous T)
  (:stop-event NIL)
  (:running-where NIL)
  (:outside NIL)
  (:abort-event :control-{\tt\char`\\}g)
  (:waiting-priority normal-priority-level)
  (:running-priority running-priority-level)
  (:active T)
  (:self-deactivate NIL)

  {\it ; Slots specific to the move-grow-interactor (discussed in this section)}
  (:final-function NIL)      {\it ; (lambda (inter obj-being-changed final-points))}
  (:line-p NIL)              {\it ; If NIL, set :box slot of object.  If T, set :points slot}
  (:grow-p NIL)              {\it ; If T, grow the object instead of move it}
  (:obj-to-change NIL)       {\it ; The object to move or grow (usually this is automatically set to be the object}
                             {\it ; returned from the start-where)}
  (:attach-point :where-hit) {\it ; Where the mouse will attach to the object}
  (:min-width 0)             {\it ; Minimum width for any object being grown}
  (:mih-height 0)            {\it ; Minimum height for any object being grown}
  (:min-length NIL)          {\it ; Minimum length of any line being grown}
  (:feedback-obj NIL)        {\it ; Optional interim feedback object.  The inter will set this object's :obj-over slot}
                             {\it ; and either its :box or :points slot.}
  (:slots-to-set :box)       {\it ; Names of slots to set in the objects. {\bf Note:} :box = :points because of :line-p slot.}
  (:input-filter NIL)        {\it ; Used for gridding}

  {\it ; {\bf Advanced feature:}  Read-only slots.}
  {\it ; See section \ref{specialslots} for details about these slots.}
  (:first-obj-over NIL)     {\it ; Read-only slot.  The object returned from the start-where.}
  (:current-window NIL)     {\it ; Read-only slot.  The window of the last (or current) event.}
  (:start-char NIL)         {\it ; Read-only slot.  The character or keyword of the start event.}

  {\it ; {\bf Advanced feature:} Customizable action routines.}
  {\it ; See sections \ref{interslots} and \ref{movegrowcustomaction} for details about functions in these slots.}
  (:start-action ...)       {\it ; (lambda (inter obj-being-changed first-points))}
  (:running-action ...)     {\it ; (lambda (inter obj-being-changed new-points))}
  (:stop-action ...)        {\it ; (lambda (inter obj-being-changed final-points))}
  (:abort-action ...)       {\it ; (lambda (inter obj-being-changed))}
  (:outside-action ...)     {\it ; (lambda (inter outside-control obj-being-changed))}
  (:back-inside-action ...) {\it ; (lambda (inter outside-control obj-being-changed new-inside-points))}
...)
\end{programexample}
\end{group}

This is used to move or change the size of an object or one of a set
of objects with the mouse.  This is quite a flexible interactor and
will handle many different behaviors including: moving the indicator
in a slider, changing the size of a bar in a thermometer, changing the
size of a rectangle in a graphics editor, changing the position of a
circle, and changing an end-point of a line.

The interactor can either be permanently tied to a particular graphics
object, or it will get the object from where the mouse is when the
interaction starts.  There may be a feedback object to show where the
object will be moved or changed to, or the object itself may change with
the mouse.

There are a number of examples of the use of move-grow-interactors
below.  Other
examples can be found in sections \ref{movegrowexample1},
\ref{movegrowexample2}, and \ref{movegrowexample3}, in
the \pr{graphics-selection} gadget in
the Garnet Gadget Set, and in the files \pr{demo-grow.lisp},
\pr{demo-moveline.lisp}, \pr{demo-scrollbar.lisp} and \pr{demo-manyobjs.lisp}.

\subsection{Default Operation}

This section describes how the \pr{move-grow-interactor} works
if the programmer does not remove or override any of the standard
\pr{-action} procedures.  To supply custom action procedures, see section
\ref{movegrowcustomaction}.

\index{box (slot)}
The feedback object (if any) {\it and}
the object being edited are modified indirectly, by setting slots called
\pr{:box} or \pr{:points}.  The programmer must provide constraints between
these slots and the \pr{:left}, \pr{:top}, \pr{:width}, and
\pr{:height} slots or the \pr{:x1, :y1, :x2,} and \pr{:y2} slots (as
appropriate).  For example, a rectangle that can be moved and changed
size with the mouse might have the following definition:
\label{howobjsdefined}
\index{Moving-Rectangle}
\indexsecondary{Primary=`Examples', Secondary=`Moving-Rectangle'}
\indexsecondary{Primary=`Examples', Secondary=`Box (slot}')
\begin{programexample}
(create-instance 'MOVING-RECTANGLE opal:rectangle
   (:box (list 0 0 10 10))  ;{\it some initial values (x, y, width, height)}
   (:left (o-formula (first (gvl :box))))
   (:top (o-formula (second (gvl :box))))
   (:width (o-formula (third (gvl :box))))
   (:height (o-formula (fourth (gvl :box)))))
\end{programexample}
A movable line could be defined as:
\index{Moving-Line}
\index{Points (slot)}
\index{Box (slot)}
\indexsecondary{Primary=`Examples', Secondary=`Moving-Line'}
\begin{programexample}
(create-instance 'MOVING-LINE opal:line
   (:points (list 0 0 10 10))  ;{\it some initial values (x1 y1 x2 y2)}
   (:x1 (o-formula (first (gvl :points))))
   (:y1 (o-formula (second (gvl :points))))
   (:x2 (o-formula (third (gvl :points))))
   (:y2 (o-formula (fourth (gvl :points)))))
\end{programexample}

The slot \pr{:line-p} tells the interactor whether to change the \pr{:box}
slot or the \pr{:points} slot.
If \pr{:line-p} is NIL (the default), then the interactor changes the
object by setting its
\pr{:box} slot to a list containing the new values for (left, top, width,
height).  If T, then the interactor changes the object by setting its
\pr{:points} slot to a list containing the new values for (x1, y1, x2, y2).
(These are the same slots as used for
\pr{two-point-interactor}\dashsection \ref{twopoint}).

This allows the object to perform any desired filtering on the values
before they are used in the real \pr{:left :top :width :height} or
\pr{:x1 :y1 :x2 :y2} slots.  For example, a scroll bar might be defined as
follows:
\index{Scroll Bar}
\indexsecondary{Primary=`Examples', Secondary=`Scroll Bar'}
\begin{programexample}
(create-instance 'MYSCROLLER opal:aggregadget
   (:parts
    `((:outline ,opal:rectangle
		(:left 100)(:top 10)(:width 20)(:height 200))
      (:indicator ,opal:rectangle
		  (:box (52 12 16 16)) ;;{\it only the second value is used}
		  (:left ,(o-formula (+ 2 (gvl :parent :outline :left))))
		  ;;{\it Clip-And-Map clips the first parameter to keep it}
		  ;;{\it  between the other two parameters, see section \ref{clipandmap}}
		  (:top ,(o-formula
			  (Clip-And-Map (second (gvl :box))
			    12  ;{\it Top of outline + 2}
			    192 ;{\it Bottom of outline - indicator height - 2}
			    )))
		  (:width 16)(:height 16)
		  (:filling-style ,opal:gray-fill)
		  (:line-style NIL)
		  (:fast-redraw-p T)
		  (:draw-function :xor))))
   (:interactors
    `((:move-indicator ,Inter:Move-Grow-Interactor
		       (:start-where
		        ,(o-formula (list :in (gvl :operates-on :indicator))))
		       (:window ,(o-formula (gvl :operates-on :window)))))))
\end{programexample}
		
This interactor will either change the position of the object (if
\pr{:grow-p} is NIL) or the size.  For lines, (if \pr{:line-p} is T),
``growing'' means changing a single end point to follow the mouse while the
other stays fixed, and moving means changing both end points to follow the
mouse so that the line keeps the same length and slope.

Since an object's size can change from the left and
top, in addition to from the right and bottom, and since objects are
defined to by their left, top, width and height, this interactor may have
to change any of the left, top, width and height fields when changing an
object's size.  For example, to
change the size of an object from the left (so that the left moves and the
right side stays fixed), both the \pr{:left} and \pr{:width} fields must be
set.  Therefore, by default, this interactor sets a \pr{:box} field
containing 4 values.  When the interactor is used for moving an object, the
last two values of the \pr{:box} slot are set with the original width and
height of the object.  Similarly, when setting the \pr{:points} slot, all
of the values are set, even though only two of them will change.

When the interaction is running, either the object itself or a separate
{\it feedback} object can follow the mouse.  If a feedback object is used, it
should be specified in the \pr{:feedback-obj} slot of the interactor, and
it will need the same kinds of formulas on \pr{:box} or \pr{:points} as the
actual object.  If
the object itself should change, then \pr{:feedback-obj} should be NIL.
If there is a feedback object, the interactor also sets its
\pr{:obj-over} field to the actual object that is being moved.
This can be used, for example,
to control the visibility of the feedback object or its size.

The object being changed is
either gotten from the \pr{:obj-to-change} slot of the interactor, or if
that is NIL, then from the object returned from \pr{:start-where}.
If the interactor is to work over
multiple objects, then \pr{:obj-to-change} should be NIL, and \pr{:start-where}
will be one of the forms that returns one of a set of objects (e.g.,
\pr{:element-of}).

\paragraph{Attach-Point}
\label{attachpoint}
\index{attach-point}
The \pr{:attach-point} slot of interactors controls where the mouse will
attach to the object.  The legal choices
depend on \pr{:line-p}.

If \pr{:line-p} is T (so the end-point of the line
is changing), and the object is being grown, then legal choices are:
\begin{description}
\item[] \pr{1}: Change the first endpoint of the line (x1, y1).

\item[] \pr{2}: Change the second endpoint of the line (x2, y2).

\item[] \pr{:where-hit}: Change which-ever end point is nearest the
initial press.
\end{description}

If \pr{:line-p} is T and the object is being moved, then legal choices are:
\begin{description}
\item[] \pr{1}: Attach mouse to the first endpoint.

\item[] \pr{2}: Attach mouse to the second endpoint.

\item[] \pr{:center}: Attach mouse to the center of the line.

\item[] \pr{:where-hit}: Attach mouse where pressed on the line.
\end{description}

If \pr{:line-p} is NIL (so the bounding box is changing, either moving or
growing) the choices are:
\begin{description}
\item[] \pr{:N} - Top

\item[] \pr{:S} - Bottom

\item[] \pr{:E} - Right

\item[] \pr{:W} - Left

\item[] \pr{:NE} - Top, right

\item[] \pr{:NW} - Top, left

\item[] \pr{:SE} - Bottom, right

\item[] \pr{:SW} - Bottom, left

\item[] \pr{:center} - Center

\item[] \pr{:where-hit} - The mouse attaches to the object
wherever the mouse was first pressed inside the object.
\end{description}
The default value is \pr{:where-hit} since this works for both \pr{:line-p}
T and NIL.

If growing and \pr{:attach-point} is \pr{:where-hit}, the object
grows from the nearest side or corner (the object is implicitly divided into 9
regions).  If the press is in the center, the object grows from the
\pr{:NW} corner.
% \comment{In the future, we might
% support growing from the center like X10/uwm, where it uses the first
% direction you move in to decide which side or corner of the object to
% change size from.}
The value set into the \pr{:box} slot by this interactor is always the correct value
for the top, left corner, no matter what the value of attach-point (the
interactor does the conversion for you).  Note that the conversion is done
based on the \pr{:left, :top, :width} and \pr{:height} of the actual object being changed; not based on the feedback object.
Therefore, if there is a separate feedback object, either
the feedback object should be the same size as the object being changed, or
\pr{:attach-point} should be \pr{:NW}.
{\bf Possible future enhancement:
allow a list of points, and pick the closest one to the mouse.}

\paragraph{Running where}
Normally, the default value for \pr{:running-where}
is the same as \pr{:start-where}, but for the move-grow-interactor, the
default \pr{:running-where} is T, to allow the mouse to go anywhere.

\paragraph{Extra Parameters}

The extra parameters are:
\begin{description}
\item[] \pr{:line-p}\index{line-p}
- This slot determines whether the object's bounding box or line end points
are set.  If \pr{:line-p} is NIL, then the \pr{:box} slot is set to a list
containing (left top width height) and if \pr{:line-p} is T, then the
\pr{:points} slot is set with a list containing (x1 y1 x2 y2).
The default is NIL.

\item[] \pr{:grow-p}\index{grow-p}
- This slot determines whether the
object moves or changes size.  The
default is NIL, which means to move.  Non-NIL means to change size.

\item[] \pr{:obj-to-change}\index{obj-to-change}
- If an object is supplied as this parameter, then the interactor changes
that object.  Otherwise, the interactor changes the object returned from
\pr{:start-where}.  If the interactor should
change one of a set of objects, then \pr{:obj-to-change} should be NIL and
\pr{:start-where} should be be a form that will return the object to change.
The reason that there may need to be a separate object passed
as the \pr{:obj-to-change}
is that sometimes the interactor cannot get the object to be changed
from the :where fields.  For example, the programmer may want to have a
scroll bar indicator changed whenever the user presses over the background.
The object in the \pr{:obj-to-change} field may be
different from the one in the \pr{:feedback-obj} since the object in the
\pr{:feedback-obj} field is used as the interim feedback.

\item[] \pr{:attach-point}\index{attach-point}
- This tells where the mouse will attach to the object.  Values are
\pr{1, 2, :center} or \pr{:where-hit} if \pr{:line-p} is T, or \pr{:N, :S, :E, :W,
:NW, :NE, :SW, :SE, :center,} or \pr{:where-hit} if \pr{:line-p} is NIL.
The default value is \pr{:where-hit}.
See section \ref{attachpoint} for a full explanation.

\item[] \pr{:min-width}\index{min-width}
- The \pr{:min-width} and \pr{:min-height} fields determine the minimum
legal width and height of the object if \pr{:line-p} is NIL and
\pr{:grow-p} is T.
Default is 0.  If \pr{:min-width} or \pr{:min-height} is NIL, then there is no
minimum width or height.  In this case, the width and height of the object
may become negative values which causes an error (so this is not recommended).
Unlike the \pr{two-point-interactor} (section \ref{twopoint}),
there are no \pr{:flip-if-change-side} or \pr{:abort-if-too-small} slots for
the \pr{move-grow-interactor}.

\item[] \pr{:min-height}\index{min-height}
- See \pr{:min-width}.

\item[] \pr{:min-length}\index{min-length}
- If \pr{:line-p} is T, this specifies the minimum length for lines.  The
default is NIL, for no minimum.  This slot is ignored if \pr{:line-p} is NIL.

\item[] \pr{:input-filter}\index{input-filter}
- Used to support gridding.  See section \ref{gridding}
\end{description}


\paragraph{Application Notification}

Often, it is not necessary to have the application notified of the result
of a move-grow-interactor, if you only want the object to move around.
Otherwise, you can have constraints in the application to the various slots
of the object being changed.

Alternatively, the programmer can provide a function to be called when the
interactor is complete by putting the function in the \pr{:final-function}
slot.  This function is called with the following arguments:
\indexsecondary{Primary=`final-function', Secondary=`Move-Grow-Interactor'}
\begin{programexample}
(lambda (an-interactor object-being-changed final-points))
\end{programexample}
\pr{Final-points} is a list of four values, either the left, top, width and
height if \pr{:line-p} is NIL, or x1, y1, x2, and y2 if \pr{:line-p} is T.

\paragraph{Normal Operation}

\index{Obj-Over (slot)}
If the value of \pr{:continuous} is T,
then when the start event happens, the interactor determines the object to
be changed as
either the value of the \pr{:obj-to-change} slot, or if that is NIL, then
the object returned from the \pr{:start-where}.
The \pr{:obj-over} slot of the object in
the \pr{:feedback-obj} slot of the interactor is set to the object being changed.
Then, for every mouse movement until the stop event happens, the interactor sets
either the \pr{:box} slot or the \pr{:points} slot (depending on the value
of \pr{:line-p}) based on a calculation that depends on the values in the
minimum slots and \pr{:attach-point}.  The object that is modified while
running is either the feedback object if it exists or the object being changed
if there is no feedback object.

\index{visible (slot)}
If the mouse goes outside of \pr{:running-where}, then if \pr{:outside} is
\pr{:last}, nothing happens until the mouse comes back inside or the stop
or abort events happen (the object stays at its last legal inside value).
If \pr{:outside} is NIL, then the feedback
object's \pr{:obj-over} slot is set to NIL (so there should be a formula in
the feedback object's \pr{:visible} slot that depends on \pr{:obj-over}).
If there is no feedback object and the mouse goes outside, then the object
being changed is returned to its original size and position (before the
interactor started).

If the abort event happens, then the feedback
object's \pr{:obj-over} slot is set to NIL, or if there is no feedback
object, then the object being changed is returned to its original size and
position (before the interactor started).

When the stop event happens, the feedback object's \pr{:obj-over} slot is
set to NIL, and the \pr{:box} or \pr{:points} slot of the actual object are
set with the last value, and the final-function
(if any) is called.

If the interactor is {\it not} continuous, when the start event happens, the
\pr{:box} or \pr{:points} slot of the actual object are
set with the initial value, and the final-function
(if any) is called.  This is probably not very useful.

\subsection{Gridding}
\label{gridding}
\index{gridding}\index{gravity}
The \pr{move-grow-interactor} supports arbitrary gridding of the
values.  The slot \pr{:input-filter} can take any of the following
values:
\begin{description}
\item[] \pr{NIL} -  for no filtering.  This is the default.

\item[] a number - grid by that amount in both X and Y with the origin at the
upper left corner of the window.

\item[] a list of four numbers: \pr{(xmod xorigin ymod yorigin)} to allow
non-uniform gridding with a specific origin.

\item[] a function of the form \pr{(lambda(inter x y) ...)} which returns
\pr{(values gridx gridy)}.  This allows arbitrary filtering of the
values, including application-specific gravity to interesting
points of other objects, snap-dragging, etc.
\end{description}

\subsection{Setting Slots}

\index{slots-to-set}\index{box}\index{points}
The \pr{move-grow-interactor} by default sets either the \pr{:box} or
\pr{:points} slots of objects (depending on whether it was a rectangle
or line-type object).  We discovered that there were a large number of
formulas that simply copied the values out of these lists.  Therefore,
in the current version, you can ask the \pr{move-grow-interactor} to
directly set the slots of objects, if you don't need any filtering
on the values.  If you want to use \pr{Clip-and-Map} or other
filtering, you should still use the \pr{:box} slot.  The slot
\pr{:slots-to-set} can be supplied to determine which slots to set.
The values can be:
\begin{description}
\item[] \pr{:box} - if line-p object, then sets
the \pr{:points} slot, otherwise sets the \pr{:box} slot.

\item[] \pr{:points} - same as \pr{:box}.  Note that the interactor ignores the
actual value put in \pr{:slots-to-set} and decides which to use based
on the value of the \pr{:line-p} slot of the object.

\item[] a list of four T's and NILs (representing \pr{(:left :top :width :height)}
or \pr{(:x1 :y1 :x2 :y2)}) - In this case,
the interactor sets the slots of the
object that have T's and doesn't set the slots that are NIL.  For
example, if \pr{:slots-to-set} is \pr{(T T NIL NIL)}, then the
interactor will set the \pr{:x1} and \pr{:y1} slots of objects that
are \pr{:line-p}, and the \pr{:left} and \pr{:top} slots of all other
objects.

\item[] a list of four slot names or NILs - In this case, the values are set
into the specified slots of the object.  Any NILs mean that slot isn't
set.  The specified slots are used whether the object is \pr{:line-p} or
not.  This can be used to map the four values into new slots.
\end{description}



\subsection{Useful Function: Clip-And-Map}
\index{Clip-And-Map}
\label{clipandmap}
It is often useful to take the value returned by the mouse and clip it
within a range.  The function \pr{Clip-And-Map} is provided by the
interactors package to help with this:
\begin{programexample}
inter:Clip-And-Map {\it val  val-1  val-2} \&optional {\it target-val-1  target-val-2}\value{function}
\end{programexample}

If {\it target-val-1} or {\it target-val-2} is NIL or not supplied, then this
function just clips {\it val} to be between {\it val-1} and {\it val-2} (inclusive).

If {\it target-val-1} and {\it target-val-2} are supplied, then this function
clips {\it val} to be in the range {\it val-1} to {\it val-2}, and then
then scales and translates the value (using linear-interpolation) to be between
{\it target-val-1} and {\it target-val-2}.

{\it Target-val-1} and {\it target-val-2} should be integers, but {\it val},
{\it val-1} and {\it val-2} can be any kind of numbers.  {\it Val-1} can
either be less or greater than {\it val-2} and {\it target-val-1} can be less
or greater than {\it target-val-2}.

Examples:
\indexsecondary{Primary=`Examples', Secondary=`Clip-And-Map'}
\begin{programexample}
(clip-and-map 5 0 10) ={\tt\char`\>} 5
(clip-and-map 5 10 0) ={\tt\char`\>} 5
(clip-and-map -5 0 10) ={\tt\char`\>} 0
(clip-and-map 40 0 10) ={\tt\char`\>} 10
(clip-and-map 5 0 10 100 200) ={\tt\char`\>} 150
(clip-and-map -5 0 10 100 200) ={\tt\char`\>} 100
(clip-and-map 0.3 0.0 1.0 0 100) ={\tt\char`\>} 30
(clip-and-map 5 20 0 100 200) ={\tt\char`\>} 175

{\it ;; Formula to put in the :percent slot of a moving scroll bar indicator.
;; Clip the moving indicator position to be between the top and bottom of
;; the slider-shell (minus the height of the indicator to keep it inside),
;; and then map the value to be between 0 and 100.}
(formula `(Clip-and-Map (second (gvl :box))
			(gv ',SLIDER-SHELL :top)
			(- (gv-bottom ',SLIDER-SHELL) (gvl :height) 2)
			0 100))
\end{programexample}


\begin{group}
\section{Two-Point-Interactor}
\index{Two-Point-Interactor}
\label{twopoint}

\begin{programexample}
(create-instance 'inter:Two-Point-Interactor inter:interactor
  {\it ;; Slots common to all interactors (see section \ref{interslots})}
  (:start-where NIL)
  (:window NIL)
  (:start-event :leftdown)
  (:continuous T)
  (:stop-event NIL)
  (:running-where NIL)
  (:outside NIL)
  (:abort-event :control-{\tt\char`\\}g)
  (:waiting-priority normal-priority-level)
  (:running-priority running-priority-level)
  (:active T)
  (:self-deactivate NIL)

  {\it ; Slots specific to the two-point-interactor (discussed in this section)}
  (:final-function NIL)      {\it ; (lambda (inter final-point-list))}
  (:line-p NIL)              {\it ; Whether to set the :box or :points slot of the feedback-obj}
  (:min-width 0)             {\it ; Minimum width for new rectangular region}
  (:mih-height 0)            {\it ; Minimum height for new rectangular region}
  (:min-length NIL)          {\it ; Minimum length for new line}
  (:abort-if-too-small NIL)  {\it ; Whether to draw feedback and execute final function when the selected region}
                             {\it ; is smaller than the minimum}
  (:feedback-obj NIL)        {\it ; Optional interim feedback object.  The inter will set this object's :visible slot}
                             {\it and its :points or :box slot.}
  (:flip-if-change-side T)   {\it ; Whether to flip origin of rectangle when appropriate}
  (:input-filter NIL)        {\it ; Used for gridding (see section \ref{gridding})}

  {\it ; {\bf Advanced feature:}  Read-only slots.}
  {\it ; See section \ref{specialslots} for details about these slots.}
  (:first-obj-over NIL)     {\it ; Read-only slot.  The object returned from the start-where.}
  (:current-window NIL)     {\it ; Read-only slot.  The window of the last (or current) event.}
  (:start-char NIL)         {\it ; Read-only slot.  The character or keyword of the start event.}

  {\it ; {\bf Advanced feature:} Customizable action routines.}
  {\it ; See sections \ref{interslots} and \ref{twopcustomaction} for details about functions in these slots.}
  (:start-action ...)       {\it ; (lambda (inter first-points))}
  (:running-action ...)     {\it ; (lambda (inter new-points))}
  (:stop-action ...)        {\it ; (lambda (inter final-points))}
  (:abort-action ...)       {\it ; (lambda (inter))}
  (:outside-action ...)     {\it ; (lambda (inter outside-control))}
  (:back-inside-action ...) {\it ; (lambda (inter outside-control new-inside-points))}
...)
\end{programexample}
\end{group}

The Two-Point-interactor is used to enter one or two new points, when there
is no existing object to change.  For example, this interactor might be
used when creating a new rectangle or line.  If the new object needs to be
defined by more than two points (for example for polygons), then you would
probably use the \pr{multi-point-interactor} instead, except that
it is not implemented yet.

Since lines and rectangles are defined differently, there are two modes for
this interactor, determined by the \pr{:line-p} slot.  If \pr{:line-p} is NIL, then
rectangle mode is used, so the new object is defined by its left, top, width,
and height.  If \pr{:line-p} is T, then the object is defined by two
points: x1, y1, and x2, y2.  Both of these are stored as a list of four values.

As a convenience, this interactor will handle clipping of the values.  A
minimum size can be supplied, and the object will not be smaller than this.

While the interactor is running, a feedback object, supplied in the
\pr{:feedback-obj} slot is usually modified to show where the new object
will be.  When the interaction is complete, however,
there is no existing object to modify, so this interactor cannot just
set an object field with the final value, like most other interactors.
Therefore, the \pr{final-function} (section \ref{twopapplnotif}) will
usually need to be used for this interactor.

There are a number of examples of the use of two-point-interactors below,
and another in section \ref{twopselectexample}.  Other
examples can be found in the file \pr{demo-twop.lisp}.

\subsection{Default Operation}

This section describes how the two-point interactor works
if the programmer does not remove or override any of the standard
\pr{-action} procedures.  To supply custom action procedures, see section
\ref{twopcustomaction}.

Just as for move-grow-interactors (section \ref{movegrowinter}),
the feedback object (if any) is modified indirectly, by setting slots called
\pr{:box} or \pr{:points}.  The programmer must provide constraints between
the \pr{:left, :top, :width,} and \pr{:height} slots or the \pr{:x1, :y1,
:x2,} and \pr{:y2} slots (as appropriate).  The examples in section
\ref{movegrowinter} show how to define constraints for the feedback object.

The slot \pr{:line-p} tells the interactor whether to change the \pr{:box}
slot or the \pr{:points} slot in the feedback object.
If \pr{:line-p} is NIL (the default), then the interactor changes the
object by setting its
\pr{:box} slot to a list containing the new values for (left, top, width,
height).  If T, then the interactor changes the object by setting its
\pr{:points} slot to a list containing the new values for (x1, y1, x2, y2).
(These are the same slots as used for \pr{move-grow-interactor}).


\paragraph{Minimum sizes}
\label{Minimumsizes}
The two-point interactor will automatically keep objects the same or bigger than a
specified size.  There are two different mechanisms: one if \pr{:line-p} is
NIL (so the object is defined by its \pr{:box}), and another if
\pr{:line-p} is T.

In both modes, the slot \pr{:abort-if-too-small} determines what happens if
the size is smaller than the defined minimum.  The default is NIL, which
means to create the object with the minimum size.  If
\pr{:abort-if-too-small} is T, however, then the feedback object will
disappear if the size is too small, and if the mouse is released, the
final-function will be called with an error value (NIL) so the application will
know not to create the object.

If \pr{:line-p} is NIL, the slots \pr{:min-width} and \pr{:min-height}
define the minimum size of the object.  If both of these are not set, zero is used
as the minimum size (the two-point-interactor will not let the width or
height get to be less than zero).  If the user moves the mouse to the left
or above of the original point, the parameter \pr{:flip-if-change-side}
determines what happens.  If \pr{:flip-if-change-side} is T (the default),
then the box will still be drawn from the initial point to the current
mouse position, and the box will be flipped.  The values put into the
\pr{:box} slot will always be the correct left, top, width and height.
If \pr{:flip-if-change-side} is NIL, then the box will peg at its minimum
value.

If \pr{:line-p} is T, the slot \pr{:min-length} determines the minimum
length.  This length is the actual distance along the line, and the line
will extend from its start point through the current mouse position for the
minimum length.  If not
supplied, then the minimum will be zero. The \pr{:min-width},
\pr{:min-height} and \pr{:flip-if-change-side} slots are ignored for lines.

\paragraph{Extra Parameters}

The extra parameters are:
\begin{description}
\item[] \pr{:line-p}\index{line-p}
- If T, the \pr{:points} slot of the feedback object is set with the list
\pr{(x1 y1 x2 y2)}.  If NIL, the \pr{:box} slot of the feedback object is
set with the list \pr{(left top width height)}.  The values in the list passed to the
final-function is also determined by \pr{:line-p}.
The default is NIL (rectangle mode).

\item[] \pr{:min-width}\index{min-width}
- The \pr{:min-width} and \pr{:min-height} fields determine the minimum
legal width and height of the rectangle or other object if \pr{:line-p} is NIL.
Default is NIL, which means use 0.  Both
\pr{min-width} and \pr{min-height}
must be non-NIL for this to take effect.  \pr{:min-width} and
\pr{:min-height} are ignored if \pr{:line-p} is non-{\sc NIL} (see \pr{:min-length}).

\item[] \pr{:min-height}\index{min-height}
- See \pr{:min-width}.

\item[] \pr{:min-length}\index{min-length}
- If \pr{:line-p} is non-{\sc NIL}, then \pr{:min-width} and
\pr{:min-height} are ignored, and the \pr{:min-length} slot is used
instead.  This slot determines the minimum allowable length for a line (in
pixels).  If NIL (the default), then there is no minimum length.

\item[] \pr{:abort-if-too-small}\index{abort-if-too-small}
- If this is NIL (the default), then if the size is smaller than the
minimum, then the size is made bigger to be the minimum (this works for
both \pr{:line-p} T and NIL).  If
\pr{:abort-if-too-small} is T, then instead, no object is created and no
feedback is shown if the size is smaller than \pr{:min-width} and
\pr{:min-height} or \pr{:min-length}.

\item[] \pr{:flip-if-change-side}\index{flip-if-change-side}
- This only applies if \pr{:line-p} is NIL (rectangle mode).
If \pr{:flip-if-change-side} is T (the default), then if the user moves to
the top or left of the original point, the rectangle will be ``flipped'' so
its top or left is the new point and the width and height is based on the
original point.  If \pr{:flip-if-change-side} is NIL, then the original
point is always the top-left, and if the mouse goes above or to the left of
that, then the minimum legal width or height is used.

\item[] \pr{:input-filter}\index{input-filter}
- Used to support gridding.  See section \ref{gridding}.
\end{description}

\paragraph{Application Notification}
\label{twopapplnotif}

Unlike with other interactors, it is usually necessary to have an
application function called with the result of the two-point-interactor.
The function is put into the \pr{:final-function} slot of the interactor,
and is called with the following arguments:
\indexsecondary{Primary=`final-function', Secondary=`Two-Point-Interactor'}
\begin{programexample}
(lambda (an-interactor final-point-list))
\end{programexample}
The \pr{final-point-list} will either be a list of the
left top width, and height or the x and y of two points, depending on the
setting of the \pr{:line-p} slot.  If the \pr{:abort-if-too-small} slot is
set (section \ref{Minimumsizes}), then the \pr{final-point-list} will be NIL
if the user tries to create an object that is too small.

Therefore, the function
should check to see if \pr{final-point-list} is NIL, and if so, not create
the object.  If you want to access the points anyway, the original point is
available as the \pr{:first-x} and \pr{:first-y} slots of the interactor,
and the final point is available in the \pr{*Current-Event*} as described
in section \ref{twopselectexample}.

{\bf IMPORTANT NOTE: When creating an object using \pr{final-point-list}, the elements
of the list should be accessed individually (e.g, \pr{(first
final-point-list) (second
final-point-list) etc.}) or else the list should be
copied \pr{(copy-list final-point-list)} before they are used in any object
slots, since to avoid consing, the interactor reuses the same list.  Examples:}
\begin{programexample}
(defun Create-New-Object1 (an-interactor points-list)
  (when points-list
    (create-instance NIL opal:rectangle
       (:left (first points-list)) ;{\it access the values in}
       (:top (second points-list)) ; {\it the list individually}
       (:width (third points-list))
       (:height (fourth points-list)))))
{\bf OR}
(defun Create-New-Object2 (an-interactor points-list)
  (when points-list
    (create-instance NIL opal:rectangle
       (:box (copy-list points-list)) ;{\it copy the list}
       (:left (first box))
       (:top (second box))
       (:width (third box))
       (:height (fourth box)))))
\end{programexample}


\paragraph{Normal Operation}
\index{box (slot)}
\index{points (slot)}
\index{visible (slot)}
If the value of \pr{:continuous} is T,
then when the start event happens,
if \pr{:abort-if-too-small} is non-NIL, then nothing happens until the
mouse moves so that the size is big enough.  Otherwise,
if \pr{:line-p} is NIL, then the \pr{:visible} slot of the
\pr{:feedback-obj} is set to T, and its \pr{:box} or \pr{:points} slot is
set with the correct values for the minimum size rectangle or line.  As the
mouse moves, the \pr{:box} or \pr{:points} slot is set with the current
size (or minimum size).  If the size gets to be less than the minimum and
\pr{:abort-if-too-small} is non-NIL, then the \pr{:visible} field of the
feedback object is set to NIL, and it is set to T again when the size gets
equal or bigger than the minimum.

If the mouse goes outside of \pr{:running-where}, then if \pr{:outside} is
\pr{:last}, nothing happens until the mouse comes back inside or the stop
or abort events happen (the object stays at its last legal inside value).
If \pr{:outside} is NIL, then the feedback
object's \pr{:visible} slot is set to NIL.

If the abort event happens, then the feedback
object's \pr{:visible} slot is set to NIL.

When the stop event happens, the feedback
object's \pr{:visible} slot is set to NIL and the final-function is called.

If the value of \pr{:continuous} is NIL, then the final-function is
called immediately on the start event with the \pr{final-point-list}
parameter as NIL if \pr{:abort-if-too-small} is non-NIL, or else a list
calculated based on the minimum size.


\begin{group}
\subsection{Examples}

\paragraph{Creating New Objects}
Create a rectangle when the middle button is pressed down, and a line when
the right button is pressed.
\index{Creating new objects}
\indexsecondary{Primary=`Examples', Secondary=`Creating new objects'}
\indexsecondary{Primary=`Examples', Secondary=`Two-point-interactor'}
\begin{programexample}
(defun Create-New-Object (an-interactor points-list)
  (when points-list
    (let (obj)
      (if (gv an-interactor :line-p)
	  ;;{\it then create a line}
	  (setq obj (create-instance NIL opal:line
                       (:x1 (first points-list))
                       (:y1 (second points-list))
                       (:x2 (third points-list))
                       (:y2 (fourth points-list))))
	  ;;{\it else create a rectangle}
	  (setq obj (create-instance NIL opal:rectangle
                       (:left (first points-list))
                       (:top (second points-list))
                       (:width (third points-list))
                       (:height (fourth points-list)))))
      (opal:add-components MYAGG obj)
      obj)))

(create-instance 'CREATERECT Inter:Two-Point-Interactor
   (:window MYWINDOW)
   (:start-event :middledown)
   (:start-where T)
   (:final-function \#'Create-New-Object)
   (:feedback-obj MOVING-RECTANGLE) ;{\it section \ref{howobjsdefined}}
   (:min-width 20)
   (:min-height 20))
(create-instance 'CREATELINE Inter:Two-Point-Interactor
   (:window MYWINDOW)
   (:start-event :rightdown)
   (:start-where T)
   (:final-function \#'Create-New-Object)
   (:feedback-obj MOVING-LINE) ;{\it section \ref{howobjsdefined}}
   (:line-p T)
   (:min-length 20))
\end{programexample}
\end{group}


\begin{group}
\section{Angle-Interactor}
\index{Angle-Interactor}

\begin{programexample}
(create-instance 'inter:Angle-Interactor inter:interactor
  {\it ;; Slots common to all interactors (see section \ref{interslots})}
  (:start-where NIL)
  (:window NIL)
  (:start-event :leftdown)
  (:continuous T)
  (:stop-event NIL)
  (:running-where NIL)
  (:outside NIL)
  (:abort-event :control-{\tt\char`\\}g)
  (:waiting-priority normal-priority-level)
  (:running-priority running-priority-level)
  (:active T)
  (:self-deactivate NIL)

  {\it ; Slots specific to the move-grow-interactor (discussed in this section)}
  (:final-function NIL)      {\it ; (lambda (inter obj-being-rotated final-angle))}
  (:obj-to-change NIL)       {\it ; The object to change the :angle slot of (if NIL, then the interactor will change}
                             {\it ; the object returned from the start-where)}
  (:feedback-obj NIL)        {\it ; Optional interim feedback object.  The inter will set this object's :obj-over slot}
                             {\it ; and its :angle slot during interim selection}
  (:center-of-rotation NIL)  {\it ; A list (x y) indicating a coordinate around which the objects will be rotated.}
                             {\it ; If NIL, the center of the object is used}

  {\it ; {\bf Advanced feature:}  Read-only slots.}
  {\it ; See section \ref{specialslots} for details about these slots.}
  (:first-obj-over NIL)     {\it ; Read-only slot.  The object returned from the start-where.}
  (:current-window NIL)     {\it ; Read-only slot.  The window of the last (or current) event.}
  (:start-char NIL)         {\it ; Read-only slot.  The character or keyword of the start event.}

  {\it ; {\bf Advanced feature:} Customizable action routines.}
  {\it ; See sections \ref{interslots} and \ref{anglecustomaction} for details about functions in these slots.}
  (:start-action ...)       {\it ; (lambda (inter obj-being-rotated first-angle))}
  (:running-action ...)     {\it ; (lambda (inter obj-being-rotated new-angle angle-delta))}
  (:stop-action ...)        {\it ; (lambda (inter obj-being-rotated final-angle angle-delta))}
  (:abort-action ...)       {\it ; (lambda (inter obj-being-rotated))}
  (:outside-action ...)     {\it ; (lambda (inter outside-control obj-being-rotated))}
  (:back-inside-action ...) {\it ; (lambda (inter outside-control obj-being-rotated new-angle))}
...)
\end{programexample}
\end{group}

This is used to measure the angle the mouse moves around a point.  It can
be used for circular gauges, for rotating objects, or for ``stirring
motions'' for objects. % \comment{ \cite{StirringMotions} }
It operates very much like the \pr{move-grow-interactor} and has
interim and final feedback that work much the same way.

The interactor can either be permanently tied to a particular graphics
object, or it will get the object from where the mouse is when the
interaction starts.  There may be a feedback object to show where the
object will be moved or changed to, or the object itself may change with
the mouse.

There is an example of the use of the angle-interactor below.  Other
examples can be found in
the \pr{Gauge} gadget in
the Garnet Gadget Set, and in the files \pr{demo-angle.lisp} and
\pr{demo-clock.lisp}.

\subsection{Default Operation}

This section describes how the angle interactor works
if the programmer does not remove or override any of the standard
\pr{-action} procedures.  To supply custom action procedures, see section
\ref{anglecustomaction}.

\index{angle (slot)}
The feedback object (if any) {\it and}
the object being edited are modified indirectly, by setting a slot called
\pr{:angle}.  The programmer must provide constraints to this slot.
If there is a feedback object, the interactor also sets its
\pr{:obj-over} field to the actual object that is being moved.
This can be used, for example,
to control the visibility of the feedback object or its size.

The angle slot is set with a value in radians measured counter-clockwise
from the far right.  Therefore, straight up is \pr{(/ PI 2.0)}, straight left
is \pr{PI}, and straight down is \pr{(* PI 1.5)}.

The object being changed is
either gotten from the \pr{:obj-to-change} slot of the interactor, or if
that is NIL, then from the object returned from \pr{:start-where}.

The interactor needs to be told where the center of rotation should be.
The slot \pr{:center-of-rotation} can contain a point as a list of
\pr{(x y)}.  If  \pr{:center-of-rotation} is NIL (the default), then the
center of the object being rotated is used.

For example, a line that can be rotated around an endpoint might have
the following definition:
\index{Rotating Line}
\indexsecondary{Primary=`Examples', Secondary=`Rotating Line'}
\begin{programexample}
(create-instance 'ROTATING-LINE opal:line
   (:angle (/ PI 4))  ;{\it initial value = 45 degrees up}
   (:line-length 50)
   (:x1 70)
   (:y1 170)
   (:x2 (o-formula (+ (gvl :x1)
                      (round (* (gvl :line-length)
				(cos (gvl :angle)))))))
   (:y2 (o-formula (- (gvl :y1)
		      (round (* (gvl :line-length)
				(sin (gvl :angle))))))))

(create-instance 'MYROTATOR Inter:Angle-Interactor
   (:start-where T)
   (:obj-to-change ROTATING-LINE)
   (:center-of-rotation (o-formula (list (gvl :obj-to-change :x1)
					 (gvl :obj-to-change :y1))))
   (:window MYWINDOW))
\end{programexample}



\paragraph{Extra Parameters}

The extra parameters are:
\begin{description}
\item[] \pr{:obj-to-change}\index{obj-to-change}
- If an object is supplied here, then the interactor modifies the
\pr{:angle} slot of that object.  If \pr{:obj-to-change} is NIL, then the
interactor operates on whatever is returned from \pr{:start-where}.
The default value is NIL.

\item[] \pr{:center-of-rotation}\index{center-of-rotation}
- This is the center of rotation for the interaction.  It should be a list
of \pr{(x y)}.  If NIL, then the center of the real object being rotated
(note: {\it not} the feedback object) is used.  The default value is NIL.
\end{description}

\paragraph{Application Notification}

\index{angle (slot)}
Often, it is not necessary to have the application notified of the result
of a angle-interactor, if you only want the object to rotate around.
Otherwise, you can have constraints in the application to the \pr{:angle} slot.

Alternatively, the programmer can provide a function to be called when the
interactor is complete by putting the function in the \pr{:final-function}
slot.  This function is called with the following arguments:
\indexsecondary{Primary=`final-function', Secondary=`Angle-Interactor'}
\begin{programexample}
(lambda (an-interactor object-being-rotated final-angle))
\end{programexample}

\paragraph{Normal Operation}

\index{Obj-Over (slot)}
\index{angle (slot)}
If the value of \pr{:continuous} is T,
then when the start event happens, the interactor determines the object to
be changed as
either the value of the \pr{:obj-to-change} slot, or if that is NIL, then
the object returned from the \pr{:start-where}.
The \pr{:obj-over} slot of the object in
the \pr{:feedback-obj} slot of the interactor is set to the object being changed.
Then, for every mouse movement until the stop event happens, the interactor sets
the \pr{:angle} slot.  The object that is modified while
running is either the feedback object if it exists or the object being changed
if there is no feedback object.

If the mouse goes outside of \pr{:running-where}, then if \pr{:outside} is
\pr{:last}, nothing happens until the mouse comes back inside or the stop
or abort events happen (the object stays at its last legal inside value).
If \pr{:outside} is NIL, then the feedback
object's \pr{:obj-over} slot is set to NIL.
If there is no feedback object and the mouse goes outside, then the object
being changed is returned to its original angle (before the
interactor started).

If the abort event happens, then the feedback
object's \pr{:obj-over} slot is set to NIL, or if there is no feedback
object, then the object being rotated is returned to its original angle
(before the interactor started).

When the stop event happens, the feedback object's \pr{:obj-over} slot is
set to NIL, and the \pr{:angle} slot of the actual object is
set with the last value, and the final-function
(if any) is called.

If the interactor is {\it not} continuous, when the start event happens, the
\pr{:angle} slot of the actual object is
set with the initial value, and the final-function
(if any) is called.

\begin{group}
\section{Text-interactor}
\index{Text-interactor}

\begin{programexample}
(create-instance 'inter:Text-Interactor inter:interactor
  {\it ;; Slots common to all interactors (see section \ref{interslots})}
  (:start-where NIL)
  (:window NIL)
  (:start-event :leftdown)
  (:continuous T)
  (:stop-event NIL)
  (:running-where T)
  (:outside NIL)
  (:abort-event '(:control-{\tt\char`\\}g :control-g))
  (:waiting-priority normal-priority-level)
  (:running-priority running-priority-level)
  (:active T)
  (:self-deactivate NIL)

  {\it ; Slots specific to the text-interactor (discussed in this section)}
  (:final-function NIL)      {\it ; (lambda (inter obj-being-edited final-event final-string x y))}
  (:obj-to-change NIL)       {\it ; The object to change the :string slot of (if NIL, then the interactor will change}
                             {\it ; the object returned from the start-where)}

  (:feedback-obj NIL)        {\it ; Optional interim feedback object.  The inter will set this object's :string, :cursor-index,}
                             {\it ; :obj-over, and :box slots}
  (:cursor-where-press T)    {\it ; Whether to position the cursor under the mouse or at the end of the string}
  (:input-filter NIL)        {\it ; Used for gridding (see section \ref{gridding})}
  (:button-outside-stop? T)  {\it ; Whether a click outside the string should stop editing (see section \ref{extra-text-parameters})}

  {\it ; {\bf Advanced feature:}  Read-only slots.}
  {\it ; See section \ref{specialslots} for details about these slots.}
  (:first-obj-over NIL)     {\it ; Read-only slot.  The object returned from the start-where.}
  (:current-window NIL)     {\it ; Read-only slot.  The window of the last (or current) event.}
  (:start-char NIL)         {\it ; Read-only slot.  The character or keyword of the start event.}

  {\it ; {\bf Advanced feature:} Customizable action routines.}
  {\it ; See sections \ref{interslots} and \ref{textcustomaction} for details about functions in these slots.}
  (:start-action ...)       {\it ; (lambda (inter new-obj-over start-event))}
  (:running-action ...)     {\it ; (lambda (inter obj-over event))}
  (:stop-action ...)        {\it ; (lambda (inter obj-over stop-event))}
  (:abort-action ...)       {\it ; (lambda (inter obj-over abort-event))}
  (:outside-action ...)     {\it ; (lambda (inter obj-over))}
  (:back-inside-action ...) {\it ; (lambda (inter obj-over event))}
...)
\end{programexample}
\end{group}


If you want to use multi-font, multi-line text objects, you will probably
want to use the special interactors defined for them, which are described
in the Opal manual.

The \pr{text-interactor} will input a one-line or multi-line string of text,
while allowing editing on the string.  A fairly complete set of editing
operations is supported, and the programmer or user can add new ones or
change the bindings of the default operations.
The intention is that this be used for string entry in text forms, for file
names, object names, numbers, labels for pictures, etc.  The strings can be
in any font, but the entire string must be in the same font.  More complex editing
capabilities are clearly possible, but not implemented here.

\index{Text}
Text-interactors work on \pr{opal:text} objects.
The interactor can either be permanently tied to a particular text
object, or it will get the object from where the mouse is when the
interaction starts.  There may be a feedback object to show the edits, with
the final object changed only when the editing is complete, or else the
object itself can be edited.  (Feedback objects are actually not very
useful for text-interactors.)
Both the feedback and the main object should be an \pr{opal:text} object.

There is an example of the use of the text-interactor below.  Other
examples can be found in
the top type-in area in the \pr{v-slider} gadget in
the Garnet Gadget Set, and in the file \pr{demo-text.lisp}.

\subsection{Default Editing Commands}
\label{defaulteditingcommands}
\index{Text Editing Commands}
\index{Editing Commands}
There is a default set of editing commands provided with text interactors.
These are based on the EMACS command set.  The programmer change this and
can create his own mappings and functions (see section \ref{keytranslations}).
In the following, keys like `\pr{insert}' and `\pr{home}' are the
specially labeled keys on the IBM/RT or Sun keyboard.  If your keyboard has
some keys you would like to work as editing commands, see section \ref{eventspec}.

\begin{description}
\item[] \pr{{\tt\char`\^}h, delete, backspace}: delete previous character.

\item[] \pr{{\tt\char`\^}w, {\tt\char`\^}backspace, {\tt\char`\^}delete}: delete previous word.

\item[] \pr{{\tt\char`\^}d}: delete next character.

\item[] \pr{{\tt\char`\^}u}: delete entire string.

\item[] \pr{{\tt\char`\^}k}: delete to end of line.

\item[] \pr{{\tt\char`\^}b, left-arrow}: go back one character.

\item[] \pr{{\tt\char`\^}f, right-arrow}: go forward one character.

\item[] \pr{{\tt\char`\^}n, down-arrow}: go vertically down one line (for multi-line strings).

\item[] \pr{{\tt\char`\^}p, up-arrow}: go vertically up one line (for multi-line strings).

\item[] \pr{{\tt\char`\^}{\tt\char`\<}, {\tt\char`\^}comma, home}: go to the beginning of the string.

\item[] \pr{{\tt\char`\^}>, {\tt\char`\^}period, end}: go to the end of the string.

\item[] \pr{{\tt\char`\^}a}: go to beginning of the current line (different from \pr{{\tt\char`\^}{\tt\char`\<}} for
multi-line strings).

\item[] \pr{{\tt\char`\^}e}: go to end of the current line (different from \pr{{\tt\char`\^}>} for
multi-line strings).

\item[] \pr{{\tt\char`\^}y, insert}: insert the contents of the cut buffer into the string at the
current point.

\item[] \pr{{\tt\char`\^}c}: copy the current string to the cut buffer.

\item[] \pr{enter, return, {\tt\char`\^}j, {\tt\char`\^}J}: Add a new line.

\item[] \pr{{\tt\char`\^}o}: Insert a new line after the cursor.

\item[] \pr{any mouse button down inside the string}: move the cursor to the
specified point.  This only works if the \pr{:cursor-where-press} slot of
the interactor is non-NIL.
\end{description}

In addition, the numeric keypad is mapped to normal numbers and symbols.

Note: if you manage to get an illegal character into the string, the string
will only be displayed up to the first illegal character.  The rest will be
invisible (but still in the \pr{:string} slot).

The interactor's \pr{:stop-event} and \pr{:abort-event} override the above
operations.  For example, if the \pr{:stop-event} is \pr{:any-mousedown},
then when you press in the string, editing will stop rather than causing
the cursor to move.  Similarly, if \pr{\#{\tt\char`\\}RETURN} is the
\pr{:stop-event}, then it
cannot be inserted into the string for a multi-line string, and if
\pr{:control-{\tt\char`\\}c} is the \pr{:abort-event}, it cannot be used to copy to the
X cut buffer.  Therefore, you need to pick the \pr{:stop-event} and
\pr{:abort-event} appropriately, or change the bindings (see section
\ref{keytranslations})

\subsection{Default Operation}

This section describes how the text interactor works
if the programmer does not remove or override any of the standard
\pr{-action} procedures.  To supply custom action procedures, see section
\ref{textcustomaction}.

\index{string (slot)}
\index{cursor-index (slot)}
Unlike other interactors, the feedback object (if any) and
the object being edited are modified directly, by setting the \pr{:string}
and \pr{:cursor-index} fields (that control the value displayed and the
position of the cursor in the string).  If there is a feedback object, the
interactor also sets the first two values of its \pr{:box} field to be the
position where the start event happened.  This might be used to put the
feedback object at the mouse position when the user presses to start a new string.

In general, feedback objects are mainly useful when you want to create new
strings as a result of the event.

The object being changed is
either gotten from the \pr{:obj-to-change} slot of the interactor, or if
that is NIL, then from the object returned from \pr{:start-where}.

\paragraph{Multi-line text strings}
The default stop event for text interactors is \pr{\#{\tt\char`\\}RETURN}, which is fine
for one-line strings, but does not work for multi-line strings.  For those,
you probably want to specify a stop event as something like
\pr{:any-mousedown} so
that \pr{\#{\tt\char`\\}RETURN}s can be typed into the string (actually, the character
in the string that makes it go to the next line is \#{\tt\char`\\}NEWLINE; the
interactor maps the return key to \#{\tt\char`\\}NEWLINE).  Also \pr{:any-mousedown}
would be a bad choice for the stop event if you wanted to allow the mouse
to be used for changing the text insert cursor position.

Note that the stop event is {\it not} edited into the string.

The \pr{:outside} slot is ignored.

The default \pr{:running-where}
is T for text-interactors.

\paragraph{Extra Parameters}
\label{extra-text-parameters}

The extra parameters are:
\begin{description}
\item[] \pr{:obj-to-change}\index{obj-to-change}
- If an object is supplied here, then the interactor modifies the
\pr{:string} and \pr{:cursor-index} slots of that object.  If
\pr{:obj-to-change} is NIL, then the
interactor operates on whatever is returned from \pr{:start-where}.  The
default value is NIL.

\item[] \pr{:cursor-where-press}\index{cursor-where-press}
- If this slot is non-NIL, then the initial position of the
text editing cursor is underneath the mouse cursor (i.e, the user begins
editing the string on the character under where the mouse was pressed).
This is the default.  If \pr{:cursor-where-press} is specified as NIL,
however, the cursor always starts at the end of the string.
This slot also controls whether the mouse is allowed to move the cursor
while the string is being edited.  If \pr{:cursor-where-press} is NIL, then
mouse presses are ignored while editing (unless they are the \pr{:stop-} or
\pr{:abort-} events), otherwise, presses can be used to move the cursor.

\item[] \pr{:input-filter}
- Used to support gridding.  See section \ref{gridding}.

\item[] \pr{:button-outside-stop?}\index{button-outside-stop?}\index{outside stop}
- Whether a mouse click {\it outside} the string should stop editing,
but still do the action it would have
done if text wasn't being edited.  This means, for example, that you
typically won't have to type RETURN before hitting the OK
button, since the down press will stop editing {\it and} still operate the
OK button.  By default this feature is enabled, but you can turn
it off by setting the \pr{:button-outside-stop?} parameter to NIL.

\end{description}



\paragraph{Application Notification}

Often, it is not necessary to have the application notified of the result
of a text-interactor, if you only want the string object to be changed, it
will happen automatically.

Alternatively, the programmer can provide a function to be called when the
interactor is complete by putting the function in the \pr{:final-function}
slot.  This function is called with the following arguments:
\indexsecondary{Primary=`final-function', Secondary=`Text-Interactor'}
\begin{programexample}
(lambda (an-interactor obj-being-edited final-event final-string x y))
\end{programexample}
The definition of the type for \pr{final-event} is in section \ref{Events}.
(It is a Lisp structure containing the particular key hit.)
The \pr{final-string} is the final value for the entire string.  {\it It is
important that you copy the string (with \pr{copy-seq}) before using it,
since it will be shared with the feedback object.}
The \pr{x} and \pr{y} parameters are the {\it initial} positions put into the
feedback object's \pr{:box} slot (which might be used as the position of the
new object).  The values of \pr{x} and \pr{y} are {\it filtered} values computed
via the \pr{:input-filter} given to the interactor (see section
\ref{gridding}).


\paragraph{Normal Operation}
\index{box (slot)}
\index{cursor-index (slot)}
\index{visible (slot)}
If the value of \pr{:continuous} is T,
then when the start event happens, if there is a feedback object, then its
\pr{:box} slot is set to the position of the start-event, and its
\pr{:obj-over} slot is set to \pr{:obj-to-change} or the result of the
\pr{:start-where}.  Its \pr{:cursor-index} is set to the
position of the start-event (if \pr{:cursor-where-press} is T) or to the
end of the string (so the cursor becomes visible).  If there is no
\pr{:feedback-obj}, then the \pr{:obj-to-change} or if that is NIL, then
the object returned from \pr{:start-where} has its cursor turned on at the
appropriate place.  If the start event was a keyboard character, it is then
edited into the string.  Therefore, you can have a text interactor start on
\pr{:any-keyboard} and have the first character typed entered into the string.

Then, for every subsequent keyboard down-press, the key is either entered
into the string, or if it is an editing command, then it is performed.

If the mouse goes outside of \pr{:running-where}, then the cursor is turned
off, and it is turned back on when the mouse goes back inside.  Events
other than the stop event and the abort event are ignored until the mouse
goes back inside.  Note: this
is usually not used because \pr{:running-where} is usually T for
text-interactors.  If
it is desirable to only edit while the mouse is
over the object, then \pr{:running-where} can be specified as \pr{'(:in *)}
which means that the interactor will work only when the mouse is over the
object it started over.

\index{string (slot)}
\index{visible (slot)}
\index{cursor-index (slot)}
If the abort event happens, then the feedback object's \pr{:string} is set with
its initial value, its \pr{:cursor-index} is set to NIL, and its
\pr{:obj-over} is set to NIL.  If there is no feedback object, then the main
object's \pr{:string} is set to its original value and its
\pr{:cursor-index} is set to NIL.

When the stop event happens, if there is a feedback object, then its
\pr{:visible} slot is set with NIL, the main object is set with
feedback object's \pr{:string}, and the \pr{:cursor-index} is set to NIL.
If there is no feedback object, then the \pr{:cursor-index} of the main
object is set to NIL.  Note that the stop event is {\it not} edited into the
string.  Finally, the final-function (if any) is called.

If the interactor is {\it not} continuous, when the start event happens, the
actions described above for the stop event are done.


\subsection{Useful Functions}

\index{Insert-Text-Into-String}
\begin{programexample}
inter:Insert-Text-Into-String {\it text-obj} {\it new-string} \&optional {\it (move-back-cursor 0)}\value{function}
\end{programexample}
The function \pr{Insert-Text-Into-String} inserts a string {\it new-string} into
an \pr{opal:text} object {\it text-obj} at the current cursor point.  This can
be used even while the text-interactor is
running.  For example, an on-screen button might insert some text into a
string.  After the text is inserted, the cursor is moved to the end of the
new text, minus the optional offset {\it move-back-cursor} (which should be
a non-negative integer).  For example, to insert the string \pr{`(+ foo
5)'} and leave the cursor between the \pr{`5'} and the \pr{`)'}, you could call:
\begin{programexample}
(Insert-Text-Into-String MYTEXT `(+ foo 5)' 1)
\end{programexample}


\subsection{Examples}

\paragraph{Editing a particular string}
\index{String}
\index{Editable String}
\indexsecondary{Primary=`Examples', Secondary=`Aggregadget'}
\indexsecondary{Primary=`Examples', Secondary=`Editable String'}
This creates an aggregadget containing a single-line text object and an
interactor to edit it when the right mouse button is pressed.
\begin{programexample}
(create-instance 'EDITABLE-STRING opal:aggregadget
   (:left 10)
   (:top 200)
   (:parts
    `((:txt ,opal:text
	    (:left ,(o-formula (gvl :parent :left)))
	    (:top ,(o-formula (gvl :parent :top)))
	    (:string `Hello World')))) ;{\it default initial value}
   (:interactors
    `((:editor ,Inter:Text-Interactor
	       (:start-where ,(o-formula (list :in (gvl :operates-on :txt))))
	       (:window ,(o-formula (gvl :operates-on :window)))
	       (:stop-event (:any-mousedown \#{\tt\char`\\}RETURN)) ;{\it either}
	       (:start-event :rightdown)))))
\end{programexample}

\paragraph{Editing an existing or new string}
\label{editstringexample}
Here, the right button will create a new multi-line string object when the
user presses on
the background, and it will edit an existing object if the user presses on
top of it, as in Macintosh MacDraw.

Note: This uses a formula in the \pr{:feedback-obj} slot that depends on
the \pr{:first-obj-over} slot of the interactor.  This slot, which holds the
object the interactor starts over, is explained in section \ref{specialslots}.
\indexsecondary{Primary=`Examples', Secondary=`Feedback'}
\indexsecondary{Primary=`Examples', Secondary=`Text'}
\indexsecondary{Primary=`Examples', Secondary=`Create or edit string'}
\begin{programexample}
(create-instance 'THE-FEEDBACK-OBJ opal:text
   (:string `')
   (:visible NIL)
   (:left (formula '(first (gvl :box))))
   (:top (formula '(second (gvl :box)))))

;;;{\it Assume there is a top level aggregate in the window called} top-agg.
;;;{\it Create an aggregate to hold all the strings.  This aggregate must have a fixed}
;;;{\it    size so user can press inside even when it does not contain any objects.}
(create-instance 'OBJECT-AGG opal:aggregate
   (:left 0)(:top 0)
   (:width (o-formula (gvl :window :width)))
   (:height (o-formula (gvl :window :height))))

(opal:add-components TOP-AGG THE-FEEDBACK-OBJ OBJECT-AGG)
(opal:update MYWINDOW)

(create-instance 'CREATE-OR-EDIT Inter:Text-Interactor
    (:feedback-obj (o-formula
		    (if (eq :none (gvl :first-obj-over))
			;{\it then create a new object, so use feedback-obj}
			THE-FEEDBACK-OBJ
			;{\it else use object returned by mouse}
			NIL)))
    (:start-where `(:element-of-or-none ,OBJECT-AGG))
    (:window MYWINDOW)
    (:start-event :any-rightdown)
    (:stop-event '(:any-mousedown :control-{\tt\char`\\}j)) ;{\it either one stops}
    (:final-function
     \#'(lambda (an-interactor obj-being-edited stop-event final-string x y)
	 (declare (ignore an-interactor stop-event))
	 (when (eq :none obj-being-edited)
	   ;;{\it then create a new string and add to aggregate.}
	   ;;{\it Note that it is important to copy the string.}
	   (let ((new-str (create-instance NIL opal:text
			     (:string (copy-seq final-string))
			     (:left x)(:top y))))
	     (opal:add-component OBJECT-AGG new-str)
	     (s-value THE-FEEDBACK-OBJ :string `') ;{\it so starts empty next time}
	     )))))
\end{programexample}

\subsection{Key Translation Tables}
\index{Key Translation Tables}
\index{Key Bindings}
\label{keytranslations}

The programmer can change the bindings of keyboard keys to editing
operations, and even add new editing operations in a straightforward
manner.

Each text interactor can have its own {\it key translation table}.  The default
table is stored in the top-level \pr{Text-Interactor} object, and so
text-interactor instances will inherit it automatically.  If you want to
change the bindings, you need to use \pr{Bind-key}, \pr{Unbind-key},
\pr{Unbind-All-Keys}, or \pr{Set-Default-Key-Translations} (these functions
are defined below).

If you want to change the binding for all of your text interactors, you can
edit the bindings of the top-level \pr{Text-Interactor} object.  If you
want a binding to be different for a particular interactor instance, just
modify the table for that instance.  What happens in this case is that the
inherited table is copied first, and then modified.  That way, other
interactors that also inherit from the default table will not be affected.
This copy is performed automatically by the first call to one of these
functions.

Bindings can be changed while the interactor is running, and they will take
effect immediately.

\index{Bind-Key}
\begin{programexample}
inter:Bind-Key {\it key} {\it val} {\it an-interactor}\value{function}
\end{programexample}
\pr{Bind-Key} sets the binding for {\it key} to be {\it val} for
{\it an-interactor}.   The {\it key} can either be a Lisp character (like
\pr{:control-{\tt\char`\\}t}) or a special keyword
that is returned when a key is hit (like \pr{:uparrow}).   If the key used
to have some other binding, the old binding is removed.  It is fine to bind
multiple keys to the same value, however (e.g., both \pr{{\tt\char`\^}p} and
\pr{:uparrow} are bound to \pr{:upline}).

The second parameter ({\it val}) can be any one of the following four
forms:
\begin{enumerate}
\item A character to map into.  This allows special keys to map to regular keys.
So, for example, you can have \pr{:super-4} map to \pr{\#{\tt\char`\\}4}.

\item A string.  This allows the key to act like an abbreviation and expand into a
string.  For example, \pr{(inter:bind-key :F2 `long string' MYINTER)} will
insert `long string' whenever F2 is hit.
Unfortunately, the string must be constant and cannot, for example, be
computed by a formula.

\item 
One of the built-in editing operations which are keywords.  These are
implemented internally to the text interactor, but the user can decide which
key(s) causes them to happen.  The keywords that are available are:
\begin{itemize}
\item \pr{:prev-char} - move cursor to previous character.

\item \pr{:next-char} - move cursor to next character.

\item \pr{:up-line} - move cursor up one line.

\item \pr{:down-line} - move cursor down one line.

\item \pr{:delete-prev-char} - delete character to left of cursor.
		
\pr{:delete-prev-word} - delete word to left of cursor.
		
\pr{:delete-next-char} - delete character to right of cursor.

\item \pr{:kill-line} - delete to end of line.

\item \pr{:insert-lf-after} - add new line after cursor.
		
\pr{:delete-string} - delete entire string.

\item \pr{:beginning-of-string} - move cursor to beginning of string.

\item \pr{:beginning-of-line} - move cursor to beginning of line.
		
\pr{:end-of-string} - move cursor to end of string.

\item \pr{:end-of-line} - move cursor to end of line.

\item \pr{:copy-to-X-cut-buffer} - copy entire string to cut buffer.

\item \pr{:copy-from-X-cut-buffer} - insert cut buffer at current cursor.
\end{itemize}
For example, \pr{(inter:bind-key :F4 :upline MYINTER)} will make the F4 key
move the cursor up one line.


\item 
A function that performs an edit.  The function should be of the following
form:
\begin{programexample}
(lambda (an-interactor text-obj event))
\end{programexample}
The interactor will be the text-interactor, the text object is the one
being edited, and the event is an Interactor event structure (see section
\ref{events}). Note: {\it not} a lisp character; the character is a field in
the event.  This function can do arbitrary manipulations of the \pr{:string} slot
and the \pr{:cursor-index} slot of the \pr{text-obj}.  For example,
the following code could be used to implement the ``swap previous two
character'' operation from EMACS:
\indexsecondary{Primary=`Examples', Secondary=`Binding Keys'}
\begin{programexample}
;;{\it first define the function}
(defun flip (inter str event) ;{\it swap the two characters to the left of the cursor}
  (let ((index (gv str :cursor-index)) ;{\it get the old values}
	(s (gv str :string)))
    (when ({\tt\char`\>} index 1)  ;{\it make sure there are 2 chars to the left of the cursor}
      (let ((oldsecondchar (elt s (1- index)))) ;{\it do the swap in place in the str}
	(setf (elt s (1- index)) (elt s (- index 2)))
	(setf (elt s (- index 2)) oldsecondchar)
	(mark-as-changed str :string ))))) ;{\it since we modified the string value}
				; {\it of the object in place, we have to let KR know}
				; {\it it has been modified.}
;;{\it now bind it to {\tt\char`\^}t for a particular text-interactor called} my-text-inter.
(inter:bind-key :control-{\tt\char`\\}t \#'flip MY-TEXT-INTER) ; lower case t
\end{programexample}

\end{enumerate}




\index{Unbind-Key}
The function \pr{Unbind-Key} removes the binding of {\it key} for
{\it an-interactor}.  All keys that
are not bound to something either insert themselves into the string (if
they are printable characters), or else cause the interactor to beep when typed.

\index{Unbind-All-Keys}
\begin{programexample}
inter:Unbind-Key {\it key} {\it an-interactor}\value{function}

inter:Unbind-All-Keys {\it an-interactor}\value{function}
\end{programexample}
\pr{Unbind-All-Keys} unbinds all keys for {\it an-interactor}.  This
would usually be followed by binding some of the keys in a different way.

\index{Set-Default-Key-Translations}
\begin{programexample}
inter:Set-Default-Key-Translations {\it an-interactor}\value{function}
\end{programexample}
This sets up {\it an-interactor} with the default key bindings presented in
section \ref{defaulteditingcommands}.  This might be useful to restore an
interactor after the other functions above were used to change the bindings.

\subsection{Editing Function}
\index{Edit-Func}

If you need even more flexibility than changing the key translations
offers, then you can override the entire editing function, which is
implemented as a method.  Simply set the \pr{:edit-func} slot of the text
interactor with a function as follows:
\begin{programexample}
lambda (an-interactor string-object event)
\end{programexample}
It is expected to perform the modifications of the string-object based on
the \pr{event}, which is a Garnet event structure (section \ref{events}).


\begin{group}
\section{Gesture-Interactor}
\index{gesture-interactor}
\label{gesture}

\begin{programexample}
(create-instance 'inter:Gesture-Interactor inter:interactor
  {\it ;; Slots common to all interactors (see section \ref{interslots})}
  (:start-where NIL)
  (:window NIL)
  (:start-event :leftdown)
  (:continuous T)                {\it ; Must be T for gesture-interactor}
  (:stop-event NIL)
  (:running-where T)
  (:outside NIL)
  (:abort-event '(:control-{\tt\char`\\}g :control-g))
  (:waiting-priority normal-priority-level)
  (:running-priority running-priority-level)
  (:active T)
  (:self-deactivate NIL)

  {\it ; Slots specific to the gesture-interactor (discussed in this section)}
  (:final-function NIL)          {\it ; (lambda (inter first-obj-over gesture-name attribs points-array nap dist))}
  (:classifier NIL)              {\it ; classifier to use}
  (:show-trace T)                {\it ; show trace of gesture?}
  (:min-non-ambig-prob nil)      {\it ; non-ambiguity probability}
  (:max-dist-to-mean nil)        {\it ; distance to class mean}
  (:went-outside NIL)            {\it ; Read-only slot.  Set in outside action function}

  {\it ; {\bf Advanced feature:}  Read-only slots.}
  {\it ; See section \ref{specialslots} for details about these slots.}
  (:first-obj-over NIL)     {\it ; Read-only slot.  The object returned from the start-where.}
  (:current-window NIL)     {\it ; Read-only slot.  The window of the last (or current) event.}
  (:start-char NIL)         {\it ; Read-only slot.  The character or keyword of the start event.}

  {\it ; {\bf Advanced feature:} Customizable action routines.}
  {\it ; See sections \ref{interslots} and \ref{gestcustomaction} for details about functions in these slots.}
  (:start-action ...)       {\it ; (lambda (inter obj-under-mouse point))}
  (:running-action ...)     {\it ; (lambda (inter new-obj-over point))}
  (:stop-action ...)        {\it ; (lambda (inter final-obj-over point))}
  (:abort-action ...)       {\it ; (lambda (inter))}
  (:outside-action ...)     {\it ; (lambda (inter prev-obj-over))}
  (:back-inside-action ...) {\it ; (lambda (inter new-obj-over))}
...)
\end{programexample}
\end{group}


The Gesture-interactor is used to recognize single-path gestures that are
drawn with the mouse.  For example, this interactor might be used to allow
the user to create circles and rectangles by drawing an ellipse for a
circle and an ``L'' shape for a rectangle with the mouse.  A
{\it classifier} will be created for these two gestures.  A
``classifier'' is a data structure that holds the information the
gesture interactor needs to differentiate the gestures.  Classifiers
are created by using a special training program to give several
examples of each kind of gesture that will be recognized.  For
instance, you might use Agate (section \ref{agate}), the Garnet
gesture trainer, to give 15 examples of the ellipses and 15 of the ``L''
shape.
Each gesture is named with a keyword (here, \pr{:circle} and \pr{:rectangle}
might be used).
Then, the classifier will be written to a file.  The gesture
interactor will then read this file and know how to recognize the
specified gestures.

The classification algorithm is based on Rubine's gesture
recognition algorithm \cite{rubine, rubinesiggraph}.  It uses a
statistical technique.

There is one example of the gesture-interactor below. Other
examples can be found in the files \pr{demo-arith.lisp} and
\pr{demo-gesture.lisp}.

Unlike other interactors, Gestures are not automatically loaded when
you load Garnet.  To load gestures, use:
\begin{programexample}
(load Garnet-Gesture-Loader)
\end{programexample}


\subsection{Default Operation}

This section describes how the gesture-interactor works
if the programmer does not remove or override any of the standard
\pr{-action} procedures.  To supply custom action procedures, see section
\ref{gestcustomaction}.

The interactor is used by specifying a classifier to use and a
\pr{final-function} (\ref{gestapplnotif}) to call with the result of the
classification.

The \pr{:classifier} slot should be set to the value of a gesture
classifier.  Classifiers trained and saved by Agate can be read with
\pr{inter:gest-classifier-read}.  The \pr{:final-function}
slot should be set to a function to call with the result of the gesture
classification.

Since the programmer may or not want the trace of the gesture to be
shown, there are two drawing modes for the interactor, determined by the
\pr{:show-trace} slot.  If \pr{:show-trace} is non-NIL (the default), then the
points making up the gesture will be displayed as the gesture is drawn
and erased when it is finished.

\paragraph{Rejecting Gestures}
\label{rejecting-gestures}
If the gesture-interactor is unable to classify the gesture, it will
call the \pr{final-function} with a value of NIL for the classified
gesture name. Often,
the gesture will be ambiguous, in that it is similar to more than one
known gesture.  By setting the \pr{:min-non-ambig-prob} slot, the
programmer can specify the minimum non-ambiguous probability below which
gestures will be rejected. Empirically, a value of .95 has been found to
be a reasonable value for a number of gesture sets \cite{rubine}.

Also, the gesture may be an outlier, different from any of the expected
gestures. An approximation of the Mahalanobis distance from the features
of the given gesture to the features of the gesture it was classified as
gives a good indication of this.  By setting the \pr{:max-dist-to-mean}
slot, the programmer can specify the maximum distance above which
gestures will be rejected. Rubine shows that a value of 60 (for our
feature set) is a good compromise between accepting obvious outliers and
rejecting reasonable gestures.

NIL for either parameter means that that kind of checking is not performed.

\paragraph{Extra Parameters}

The extra parameters are:
\begin{description}
\item[] \pr{:classifier}\index{classifier}
 - This field determines which classifier to use when recognizing
gestures.  If NIL (the default), the gesture-interactor will call the
\pr{final-function} with a result of NIL.

\item[] \pr{:show-trace}\index{show-trace}
 - If non-NIL (the default), the points making up the gesture are displayed in
the supplied interactor window as the gesture is drawn.  If NIL, no points
are displayed.

\item[] \pr{:min-non-ambig-prob}\index{min-non-ambig-prob}
 - This field determines the minimum non-ambiguous probability below which
gestures will be rejected.  The default value of NIL causes the
interactor to not make this calculation and pass NIL as the \pr{nap}
parameter to \pr{final-function}.

\item[] \pr{:max-dist-to-mean}\index{max-dist-to-mean}
 - This field determines the maximum distance to the classified
gesture from the given gesture.  Any gesture with a distance above
this value will be rejected.
The default value of NIL causes the interactor to not make this
calculation and pass NIL as the \pr{dist} parameter to \pr{final-function}.
\end{description}

\paragraph{Application Notification}
\label{gestapplnotif}

Like the two-point-interactor, it is always necessary to have an
application function called with the result of the gesture-interactor.
The function is put into the \pr{:final-function} slot of the interactor,
and is called with the following arguments:
\indexsecondary{Primary=`final-function', Secondary=`Gesture-Interactor'}
\begin{programexample}
(lambda (an-interactor first-obj-over gesture-name attribs points-array nap dist))
\end{programexample}

The \pr{gesture-name} will be set to the name the drawn gesture was
recognized as.  These names are stored in the classifier as keyword
parameters (e.g., \pr{:circle}).  If the gesture could not be recognized
this will be set to NIL.

The \pr{attribs} will be set to a structure of gesture attributes that may
be useful to the application.  For example, the bounding box of the gesture
is one of these attributes.  The following function calls can be used to
access these attributes:
\index{gest-attributes-startx}
\index{gest-attributes-starty}
\index{gest-attributes-initial-sin}
\index{gest-attributes-initial-cos}
\index{gest-attributes-dx2}
\index{gest-attributes-dy2}
\index{gest-attributes-magsq2}
\index{gest-attributes-endx}
\index{gest-attributes-endy}
\index{gest-attributes-minx}
\index{gest-attributes-maxx}
\index{gest-attributes-miny}
\index{gest-attributes-maxy}
\index{gest-attributes-path-r}
\index{gest-attributes-path-th}
\index{gest-attributes-abs-th}
\index{gest-attributes-sharpness}
\begin{programexample}
(gest-attributes-startx attribs)        ;{\it first point}
(gest-attributes-starty attribs)

(gest-attributes-initial-sin attribs)   ;{\it initial angle to the x axis}
(gest-attributes-initial-cos attribs)

(gest-attributes-dx2 attribs)           ;{\it differences: endx - prevx}
(gest-attributes-dy2 attribs)           ;{\it                    endy - prevy}
(gest-attributes-magsq2 attribs)        ;{\it (dx2 * dx2) + (dy2 * dy2)}

(gest-attributes-endx attribs)          ;{\it last point}
(gest-attributes-endy attribs)

(gest-attributes-minx attribs)          ;{\it bounding box}
(gest-attributes-maxx attribs)
(gest-attributes-miny attribs)
(gest-attributes-maxy attribs)

(gest-attributes-path-r attribs)        ;{\it total path length (in rads)}
(gest-attributes-path-th attribs)       ;{\it total rotation (in rads)}
(gest-attributes-abs-th attribs)        ;{\it sum of absolute values of path angles}
(gest-attributes-sharpness attribs)     ;{\it sum of non-linear function of absolute values}
                                        ;{\it of path angles counting acute angles heavier}
\end{programexample}

The \pr{points-array} will be set to an array (of the form
[x1 y1 x2 y2...]) containing the points in the gesture. This array can
be used along with a NIL classifier to use the gesture-interactor as a
trace-interactor.  A trace-interactor returns all the points the
mouse goes through between the \pr{start-event} and the \pr{stop-event}.
This is useful for inking in a drawing program.

{\bf IMPORTANT NOTE:} The elements of the \pr{attribs} structure and the
\pr{points-array} should be accessed individually
(e.g., \pr{(gest-attributes-minx attribs) (aref points-array 0) etc.})
or else they should be copied
(e.g., \pr{(copy-gest-attributes attribs) (copy-seq points-array)})
before they are used in any object slots.  This is necessary because
the interactor reuses the \pr{attribs} structure and the \pr{points-array}
in order to avoid extra memory allocation.

If \pr{:min-non-ambig-prob} is not NIL, the \pr{nap} parameter will be
set to the calculated non-ambiguous probability of the entered gesture.

If \pr{:max-dist-to-mean} is not NIL, the \pr{dist} parameter will be
set to the calculated distance of the entered gesture from the
classification.

\paragraph{Normal Operation}
\index{show-trace}
\index{classifier (slot)}
When the start event happens, if \pr{:show-trace} is non-NIL, a trace
following the mouse pointer will be displayed.  If \pr{:show-trace}
is NIL, no trace will be seen.

If the mouse goes outside of \pr{:running-where}, then the system will
beep and if \pr{:show-trace} is non-NIL, the trace will be erased.

If the abort event happens and if \pr{:show-trace} is non-NIL, the trace
will be erased.

When the stop event happens, if \pr{:show-trace} is non-NIL, the trace
will be erased. Then, the final-function is called with the result of
classifying the given gesture with the classifier supplied in the
\pr{:classifier} slot.

An error will be generated if the \pr{:continuous} slot is anything
other than T, the default.

\begin{group}
\subsection{Example - Creating new Objects}
Create a rectangle when an ``L'' shape is drawn and create a circle when
a circle is drawn.
\vspace{1 line}
\index{creating new objects}
\indexsecondary{Primary=`examples', Secondary=`creating new objects'}
\indexsecondary{Primary=`examples', Secondary=`Gesture-Interactor'}
\begin{programexample}
; {\it load the gesture interactor, unless already loaded (Garnet does NOT load the gesture-interactor by default)}
(defvar DEMO-GESTURE-INIT
    (load Garnet-Gesture-Loader))

; {\it handle-gesture is called by the gesture interactor after it classifies a gesture}
(defun Handle-Gesture (inter first-obj-over gesture-name attribs
                       points-array nap dist)
    (declare (ignore inter first-obj-over points-array nap dist))
    (case gesture-name
        (:CIRCLE
            ; {\it create a circle with the same `radius' as the gesture and with the same upper left of the gesture}
            (opal:add-components SHAPE-AGG
                (create-instance NIL opal:circle
                   (:left (inter:gest-attributes-minx attribs))
                   (:top (inter:gest-attributes-miny attribs))
                   (:width (- (inter:gest-attributes-maxx attribs)
                              (inter:gest-attributes-minx attribs)))
                   (:height (- (inter:gest-attributes-maxx attribs)
                               (inter:gest-attributes-minx attribs)))))
        )
        (:RECTANGLE
            ; {\it create a rectangle with the same height and width as the gesture and with the same upper left of the gesture}
            (opal:add-components SHAPE-AGG
                (create-instance NIL opal:rectangle
                   (:left (inter:gest-attributes-minx attribs))
                   (:top (inter:gest-attributes-miny attribs))
                   (:width (- (inter:gest-attributes-maxx attribs)
                              (inter:gest-attributes-minx attribs)))
                   (:height (- (inter:gest-attributes-maxy attribs)
                               (inter:gest-attributes-miny attribs)))))
        )
        (otherwise
            (format T `Can not handle this gesture ...{\tt\char`\~}\%{\tt\char`\~}\%')
        )
    )
    (opal:update TOP-WIN)
)

; {\it create top-level window}
(create-instance 'TOP-WIN inter:interactor-window
   (:left 750) (:top 80) (:width 520) (:height 400)
)

; {\it create the top level aggregate in the window}
(s-value TOP-WIN :aggregate (create-instance 'TOP-AGG opal:aggregate))

; {\it create an aggregate to hold the shapes we will create}
(create-instance 'SHAPE-AGG opal:aggregate)
(opal:add-components TOP-AGG SHAPE-AGG)
(opal:update TOP-WIN)

; {\it create a gesture interactor that will allow us to create circles and rectangles}
(create-instance 'GESTURE-INTER inter:gesture-interactor
   (:window TOP-WIN)
   (:start-where (list :in TOP-WIN))
   (:running-where (list :in TOP-WIN))
   (:start-event :any-mousedown)
   (:classifier (inter:gest-classifier-read
                    (merge-pathnames `demo-gesture.classifier'
                        \#+cmu `gesture-data:'
                        \#-cmu user::Garnet-Gesture-Data-Pathname)))
   (:final-function \#'Handle-Gesture)
   (:min-non-ambig-prob .95)
   (:max-dist-to-mean 60)
)
\end{programexample}
\end{group}

\subsection{Agate}
\label{agate}
\index{agate}
\index{training gestures}
\indexsecondary{Primary=`gestures', Secondary=`training new gestures'}

Agate is a Garnet application that is used to train gestures for use
with the gesture interactor.  Agate stands for {\it A}
{\it G}esture-recognizer {\it A}nd {\it T}rainer by {\it E}xample.
Agate is in the \pr{gesture} subdirectory, and can be loaded using
\pr{(garnet-load `gestures:agate')}.  Then type \pr{(agate:do-go)} to begin.

\begin{figure}
\bar{}
\begin{center}
\graphic{Postscript=`inter/agate-pix.ps',boundingbox=file,magnify=.75}\end{center}
\caption{An example session with the Agate gesture trainer}
\tag{agate-pix}
\bar{}
\end{figure}

\begin{group}
\paragraph{End-User Interface}

To train a gesture classifier, the user first types a gesture name into
the \pr{Gesture Class Name} field and then demonstrates approximately
15 examples of the gesture by drawing on the \pr{Canvas} window with
one of the mouse buttons pressed.  To train another gesture class the
user can press on the \pr{New Class} button, type in the new gesture
name, and give some examples of the gesture. This is done repeatedly for
each of the gestures that the user would like the classifier to
recognize.
\end{group}
\vspace{1 line}

At any point, the user can try out the gestures trained so far by
switching to \pr{Recognize} mode by clicking on the \pr{Recognize}
toggle button.  After demonstrating a gesture in \pr{Recognize} mode,
Agate will print the name of the gesture in the \pr{Gesture Class
Name} field, along with numbers that represent the non-ambiguity
probability and distance of the example from the mean (see section
\ref{rejecting-gestures}).

When the gesture classifier performs as desired, it can be saved to a
file by clicking on the \pr{Save Classifier} button.  Existing
classifiers can be modified by first loading them into Agate by
clicking on the \pr{Load Classifier} button.  Then the user can add
more examples to existing gestures or add entirely new gestures to the
classifier.

A gesture example can be deleted by first selecting the example (a
full-sized version of the gesture will be displayed on the Canvas) and
then clicking on the \pr{Delete Example} button.  Similarly, an entire
class can be deleted by selecting the class (all of the examples
will be displayed in the \pr{Examples} window) and then clicking on
the \pr{Delete Class} button.  A gesture class can be renamed by
selecting the class and then editing the name in the \pr{Gesture Class
Name} field.

The current gesture classifier can be cleared out by clicking on
the \pr{New Classifier}.  The user will be prompted to save the
classifier if it has not been previously saved.

\paragraph{Programming Interface}

Agate V2.0 is a self-contained interface tool that can be integrated within
another Garnet application.  A designer can call \pr{agate:do-go} with
parameters for an initial classifier, an initial name to be
displayed in the \pr{Gesture Class Name} field, and a final function
to call when the user quits Agate.

\begin{programexample}
agate:Do-Go \&key {\it dont-enter-main-event-loop  double-buffered-p}\value{function}
                 {\it initial-classifier  initial-examples  initial-gesture-name final-function}
\end{programexample}

\pr{Do-go} creates the necessary windows and Garnet objects, and
then starts the application.  The parameters to \pr{do-go} are as follows:

\begin{description}
\item[] {\it dont-enter-main-event-loop} - if T, don't enter the main event loop

\item[] {\it double-buffered-p} - if T, use double buffered windows

\item[] {\it initial-classifier} - initial classifier to use

\item[] {\it initial-examples} - initial examples to display

\item[] {\it initial-gesture-name} - name to fill in gesture class name field

\item[] {\it final-function} - function to call on quit
\end{description}

The final function takes five parameters:

\begin{description}
\item[] {\it last-saved-filename} - the last filename saved to

\item[] {\it cur-classifier} - the current classifier (as of last training)

\item[] {\it cur-examples} - the current examples (if untrained, will not
necessarily correspond to the cur-classifier)

\item[] {\it saved-p} - has the current classifier been saved?

\item[] {\it trained-p} - has the current classifier been trained?
\end{description}


\subsection{Gesture Demos}

There are two demos that show how gestures can be used in an application.
Demo-gesture allows you to draw rough approximations of circles and
rectangles, which become perfect shapes in the window.  Demo-unidraw is
a gesture-based text editor which uses a gesture shorthand for entering
characters.  Both of these demos are discussed in the Demos section of
this manual, starting on page \value{demos-first-page}.


\begin{group}
\section{Animator-Interactor}
\index{animator-interactor}
\label{animation}

\begin{programexample}
(create-instance 'inter:Animator-Interactor inter:interactor
  {\it ;; Slots common to all interactors (see section \ref{interslots})}
  (:window NIL)
  (:active T)

  {\it ; Slots specific to the button-interactor (discussed in this section)}
  (:timer-handler NIL)         {\it ; (lambda (inter))  ;; function to execute}
  (:timer-repeat-wait 0.2)     {\it ; time in seconds}
...)
\end{programexample}
\end{group}


The \pr{animator-interactor} has been implemented using the
multiple process mechanism of Allegro, LispWorks, Lucid (also Sun and
HP) Common Lisp.  {\bf It does not
work under CMU Common Lisp, AKCL, CLISP, etc.; sorry.}

The \pr{animator-interactor} works quite differently from other
interactors.  In particular, it is more procedural.
You provide a function to be called at a fixed rate in the \pr{:timer-handler}
slot, and a time interval in the slot \pr{:timer-repeat-wait} at which this
function will be executed.
\index{timer-handler slot (animation)}
The \pr{:timer-handler} function takes as a parameter the animation
interactor and should update the appropriate graphics.

Unlike other interactors, the animation interactor does {\it not} start
immediately when created.  You must explicitly start it operating with
\pr{inter:Start-Animator} and stop it with \pr{inter:Stop-Animator}:
\index{Start-Animator}
\index{Stop-Animator}
\begin{programexample}
inter:Start-Animator {\it animator-inter}\value{function}

inter:Stop-Animator {\it animator-inter}\value{function}
\end{programexample}
After starting, the interactor will call the \pr{:timer-handler} every
\pr{:timer-repeat-wait} seconds until you explicitly stop the
interactor.  It is OK for the \pr{:timer-handler} itself to call
\pr{stop-animator}.

Two special-purpose animator interactors have been supplied that have
built-in timer functions (so you don't have to supply the
\pr{:timer-handler} for these):
\index{animator-bounce}
\index{animator-wrap}
\index{timer functions}
\begin{programexample}
(create-instance 'inter:Animator-Bounce inter:animator-interactor
  (:x-inc 2)
  (:y-inc 2)
  (:timer-repeat-wait 0.2)  {\it ; seconds}
  (:obj-to-change NIL)      {\it ; fill this in}
  ...)

(create-instance 'inter:Animator-Wrap inter:animator-interactor
  (:x-inc 2)
  (:y-inc 2)
  (:timer-repeat-wait 0.2) {\it ; seconds}
  (:obj-to-change NIL)     {\it ; fill this in}
  ...)
\end{programexample}

\pr{Animator-bounce} will move the object supplied in the
\pr{:obj-to-change} by \pr{:x-inc} pixels in the x direction and
\pr{:y-inc} pixels in the y direction every \pr{:timer-repeat-wait}
seconds.  The object is modified by directly setting its \pr{:left}
and \pr{:top}. (Note: {\it not} its \pr{:box} slot.)
When the object comes to the edge of its window, it will
bounce and change direction.

\pr{Animator-wrap} moves an object the same way except that when it
gets to an edge, it re-appears at the opposite edge of the window.

See the demo \pr{demo-animator} for examples.






% \begin{comment}
% +++++++++++++++++++++++++++++++++++++++++++++++++++++++++++++++++++++++++++++++
%           All the interactors that are not yet implemented
% +++++++++++++++++++++++++++++++++++++++++++++++++++++++++++++++++++++++++++++++
% 
% \section{Button-Trill-Interactor}
% {\it Performs the action once when the interactor starts, then after a
% specified delay, does the action repeatedly at a particular rate until
% stop-event}.
% 
% {\it This section not yet written or implemented.}
% 
% \section{Trace-Interactor}
% {\it This section not yet written or implemented.}
% 
% \section{Multi-Point-Interactor}
% 
% Stopping conditions: One of a set of other interactors wants to run, or two
% points in the same place, or last point same as first point, or some other
% special event, or event while outside.
% 
% {\it This section not yet written or implemented.}
% 
% +++++++++++++++++++++++++++++++++++++++++++++++++++++++++++++++++++++++++++++++
% \end{comment}
\chapter{Transcripts}
\label{transcripts-sec}
\index{Transcripts}
\index{Recording}
\index{Playback}

Garnet will create a transcript of all mouse and keyboard events in a file, and
allow the file to be replayed later as if the user had executed
all events again.  This can be used for demonstrations, human factors testing,
and/or debugging.  Using the transcript mechanism is very easy.  The
procedure to start saving events is:
\index{Transcript-Events-To-File}
\begin{programexample}
inter:Transcript-Events-To-File {\it filename  window-list}\value{function}
				\&key ({\it motion} T) ({\it if-exists} :supersede)
                                     ({\it wait-elapsed-time} T) ({\it verbose} T)
\end{programexample}
Events are then written to file {\it filename}.  The {\it window-list} is a
list of windows that events should be saved for.  It is also allowed to be
a single window.  (Note: subwindows of windows on the window list are also
handled automatically, and do {\it not} have to be specified.)
If the \pr{:motion} parameter is specified as NIL, then
mouse movement events are not saved to the file, which can significantly
decrease the file size.  The \pr{:if-exists} parameter is used in the Lisp
\pr{open} command when opening a file, and takes the same values (see
the Common Lisp book).  If specified as \pr{:append}, then the new events
are appended to the end of an existing file.  The transcript is a textual
file, where each event has its own line.

When you are finished making the transcript, call
\index{Close-Transcript}
\begin{programexample}
inter:Close-Transcript\value{function}
\end{programexample}

To replay a transcript, use:
\index{Transcript-Events-From-File}
\begin{programexample}
inter:Transcript-Events-From-File {\it filename window-list} \&key ({\it wait-elapsed-time} T)\value{function}
\end{programexample}
The {\it filename} is the file to read from.  {\it Wait-elapsed-time}
determines if the replay should wait for the correct time so the replay
goes about the same speed as the original user went, or else (if NIL)
whether the replay should just go as fast as possible.  (Each event in the
transcript has a timestamp in it.)

It is important that the {\it window-list} passed to
\pr{Transcript-Events-From-File} be windows that are the same type
and in the same order as the windows passed to the
\pr{Transcript-Events-To-File} call that made
the transcript.  Garnet maps each event from the transcript into the
corresponding window in the specified window-list.  The windows do not have
to be in the same places (all events are window-relative), however.

A typical example would be to create a bunch of windows, call
\pr{Transcript-Events-To-File}, do some operations, call
\pr{Close-Transcript}, then sometime later, create new windows the
same way, then call \pr{Transcript-Events-From-File}.

\index{Garnet-Break-Key}
\index{*Garnet-Break-Key*}
During playback, all mouse and keyboard events are ignored, except
\pr{inter:*Garnet-Break-Key*}, which is normally bound to \pr{:F1}.
This aborts
the transcript playback.  Window refresh events are handled while
replaying, however.


\chapter{Advanced Features}
\label{advancedfeatures}

This chapter describes a number of special features that will help
experienced Interactor users achieve some necessary effects.  The features
described in this chapter are:
\begin{description}
\item[] {\it Priorities}:  Interactors can be put at different priority levels, to help
control which ones start and stop with events.

\item[] {\it Modes}:  The priority levels and the \pr{:active} slots can be used for
local or global modes.

\item[] {\it Events}:  The \pr{event} structure that describes the user's event
can be useful.

\item[] {\it Start-interactor} and {\it Abort-Interactor}:  These functions can be used
to explicitly start and stop an
interactor without waiting for its events.

\item[] {\it Special slots of interactors}:  There are a number of slots of
interactors that are maintained by the system that can be used by
programmers in formulas or custom action routines.

\item[] {\it Multiple windows}:  Interactors can be made to work over multiple
windows.

\item[] {\it Waiting for interaction to complete}:  To support synchronous interfaces.

\item[] {\it Custom Action Routines}:  Some advice about how to write your own action
routines, when necessary.
\end{description}

\section{Priority Levels}
\index{Priorities}
\label{priorities}
Normally, when events arrive from the user, they are processed by {\it all}
the interactors that are waiting for events.  This means that if two
interactors are waiting for the same event (e.g. \pr{:leftdown}) they may
both start if the mouse location passes both of their \pr{:start-where}s.

The interactors do not know about object covering, so that even if an
object is covered by some other object, the mouse can still be in that
object.  For example, you might have an interactor that starts when you
press over the indicator of a scroll bar, and a different interactor that
starts when you press on the background of the scroll bar.  However, if
these interactors both start with the same event, they will both start when
the user presses on the indicator, because it is also inside the
background.  Priority levels can be used to solve this problem.
The higher-priority interactors get to process events and run first, and if
they accept the event, then lower-priority interactors can be set up so
they do not run.  Garnet normally uses three priority levels, but you can
but you can add more priority levels for your interactors as you need them
(see below).

By default, interactors wait at ``normal'' priority for their
start event to happen, and then are elevated to a higher priority while
they are running.  This means that the stop event for the running
interactor will not be seen by other interactors.  The programmer has full
control over the priorities of interactors, however.  There are two slots of
interactors that control this:
\begin{description}
\item[] \pr{:waiting-priority}\index{waiting-priority}
- the priority of the interactor while waiting for its start event.
The default value is \pr{inter:normal-priority-level}.

\item[] \pr{:running-priority}\index{running-priority}
- the priority of the interactor while running (waiting for the stop
event).
The default value is \pr{inter:running-priority-level}.
\end{description}

There are a list of priority levels, each of which contains a list of
interactors.  The events from the user are first processed by all the
interactors in the highest priority level.  All the interactors at this level
are given the event.  After they are finished, then lower level priorities
may be given the event (controlled by the \pr{:stop-when} slot of the
priority level that has just finished running, see below).
Thus, all the interactors at the same priority level get to process the
events that any of them get.

\index{priority-level-list}
\index{priority-level}
There is a list of priorities stored in the variable
\pr{inter:priority-level-list}.  The first element of this list has the
highest priority, and the second element has the second priority, etc.
This list is exported so programs can use the standard list manipulation
routines to modify it.

The elements of this list must be
instances of \pr{inter:priority-level}, which is a KR schema with the
following slots:
\begin{description}
\item[] \pr{:interactors}\index{interactors (slot of priority-level)}
- List of interactors at this priority level.  This slot is maintained
automatically based on the values in the interactor's
\pr{:waiting-priority} and \pr{:running-priority} slots.  {\it Do not set
or modify this slot directly.}

\item[] \pr{:active}\index{active (slot of priority-level)}
- Determines whether this priority level and all the interactors in it are
active.  The default value is T.
For an interactor to be usable, both the interactor's \pr{:active}
slot and the priority-level's \pr{:active} slot must be non-NIL.
If this slot is NIL, then this level is totally ignored, including its
\pr{:stop-when} field (see below).  The value of the \pr{:active} slot can
be a formula, but
if it changes to be NIL, the interactors will not be automatically aborted.  Use
the \pr{change-active} function to get the priority level and all its
interactors to be aborted immediately (see section \ref{active}).
Note: It is a really bad idea to make
the \pr{:active} slot of any {\it running}-priority levels be NIL, since
interactors will start but never complete.

\item[] \pr{:stop-when}\index{stop-when (slot of priority-level)}
- This slot controls what happens after the event has been processed by the
interactors at this priority level.  This slot can take one of three
values:
\begin{description}
\item[] \pr{:always}\index{always}
- Always stop after handling this level.  This means that the event is
never seen by interactors at lower levels.  Pushing a new priority level
with \pr{:stop-when} as \pr{:always} on the front of \pr{:priority-level-list}
is a convenient way to set up a special mode where the interactors in the
new priority level are processed and all other ones are ignored.  The
priority level can be popped or de-activated (by setting its \pr{:active}
slot to NIL) to turn this mode off.

\item[] \pr{:if-any}\index{if-any}
- If any of the interactors at this level accept the event, then do not
pass the event down to lower levels.  If no interactors at this level want
the event, then {\it do} pass it through to lower levels.  This is used, for
example, for the \pr{:stop-when} of the default
\pr{running-priority-level} to keep the stop-event of a running interactor
from starting a different interactor.

\item[] \pr{NIL}
- If \pr{:stop-when} is NIL, then the events are always passed through.
This might be useful if you want to control the order of interactors
running, or if you want to set the \pr{:active} slots of the priority
levels independently.
\end{description}

\item[] \pr{:sorted-interactors} - See section \ref{sorted-sec}.
\end{description}

Three priority levels are supplied by default.  These are:
\begin{description}
\item[] \pr{inter:running-priority-level}\index{running-priority-level}
- The highest default priority is for interactors that are running.   It is
defined with \pr{:stop-when} as \pr{:if-any}.

\item[] \pr{inter:high-priority-level}\index{high-priority-level}
- A high-priority level for use by programs.  It is
defined with \pr{:stop-when} as \pr{:if-any}.

\item[] \pr{inter:normal-priority-level}\index{normal-priority-level}
- The normal priority for use by interactors that are waiting to run.
\pr{:Stop-when} is NIL.
\end{description}

The initial value of \pr{priority-level-list} is:
\begin{programexample}
(list running-priority-level high-priority-level normal-priority-level)
\end{programexample}

The programmer can create new priority levels (using
\pr{(create-instance NIL inter:priority-level ...)} and add them to this
list (using the standard CommonLisp list manipulation routines).
The new priorities can be at any level.  Priorities can also be removed at
any time, but {\it do not remove the three default priority levels}.
There is nothing special about the pre-defined priorities.  They are just
used as the defaults for interactors that do not define a waiting and
running priority.  For example, it is acceptable to use the pre-defined
\pr{inter:running-priority-level} as the \pr{:waiting-priority} for an
interactor, or to use \pr{inter:high-priority-level} as the
\pr{:running-priority} of another interactor.

It is acceptable for an interactor to use the same priority level for its
\pr{:waiting-priority} and \pr{:running-priority}, but it is a bad idea for
the  \pr{:running-priority} to be {\it lower} than the \pr{:waiting-priority}.
Therefore, if you create a new priority level above the
\pr{running-priority-level}
and use it as the \pr{:waiting-priority} of an interactor, be sure to
create an even higher priority level for use as the \pr{:running-priority}
of the interactor (or use the same priority level as both the waiting and
running priorities).

\subsection{Example}
\label{movegrowexample1}

Consider the scroll bar.  The interactor
that moves the indicator might have higher priority than the one that
operates on the background.
\index{scroll bar}
\indexsecondary{Primary=`Examples', Secondary=`Scroll Bar'}
\indexsecondary{Primary=`Examples', Secondary=`Priority Levels'}
\begin{programexample}
(create-instance NIL Inter:Move-Grow-Interactor
   (:window MYWINDOW)
   (:start-where (list :in-box INDICATOR))
   (:running-where (list :in-box SLIDER-SHELL))
   (:outside :last)
   (:attach-point :center)
   (:waiting-priority inter:high-priority-level))

(create-instance NIL Inter:Move-Grow-Interactor
   (:continuous NIL)
   (:window MYWINDOW)
   (:start-event :leftdown)
   (:start-where (list :in-box SLIDER-SHELL))
   (:obj-to-change indicator)
   (:attach-point :center))
\end{programexample}


\subsection{Sorted-Order Priority Levels}
\label{sorted-sec}
\index{priority levels}

As an experiment, and to support the Marquise tool which is in
progress, there is an alternative way to control which interactors run.
You can mark an interactor priority level as having
\pr{:sorted-interactors}.  When this slot of a priority level is
non-NIL, then the interactors in that level run in sorted order by the
number in the \pr{:sort-order} slot of each interactor (which can be an
integer or float, negative or positive).  The lowest numbered
interactor runs first.  Then, if that interactor has a value in its
\pr{:exclusivity-value} slot, then no other interactor with the same value
in that slot will be run, but interactors with a different value in
that slot will be run in their sorted order.  Interactors with NIL in
their \pr{:sort-order} and/or \pr{:exclusivity-value} slot will run after all
other interactors are run.  Note that multiple interactors with the
same number in the \pr{:sort-order} slot will run in an indeterminate order
(or if they have the same \pr{:exclusivity-value}, then only one of them
will run, but no telling which one).  The \pr{:stop-when} slot of the
priority-level works as always to determine what happens when the
interactors in that level are finished.



\section{Modes and Change-Active}
\label{active}
\index{Modes}

In order to implement ``Modes'' in a user interface, you need to have
interactors turn off sometimes.  This can be done in several ways.  Section
\ref{modal-p} below discusses how to restrict all interactor input to a
single window (like a dialog box) while suspending the interactors in all
other windows.  Section \ref{change-active} below discusses how to turn off
particular interactors or groups of interactors.


\subsection{Modal Windows}
\label{modal-p}
\index{modal windows}

When the \pr{:modal-p} slot of an \pr{interactor-window} is T,
then interaction in all other Garnet windows
will be suspended until the window goes away (e.g., the user clicks an `OK'
button).  Any input directed to a non-modal window will
cause a beep.  If more than one modal window is visible at the same time,
then input can be directed at any of them (this allows stacking of modal
windows).  The \pr{:modal-p} slot can be calculated by a formula.
Typically, however, the \pr{:modal-p} slot will stay T, and you will
simply set the window to be visible or invisible.

The \pr{:modal-p} slot is often used in conjunction with
\pr{wait-interaction-complete}, a function which suspends all lisp activity
until \pr{interaction-complete} is called.  An example application would make
a modal window visible, then call \pr{wait-interaction-complete}.  The user
would be unable to interact with the rest of the interface until the modal
window was addressed.  Then, when the user clicks on the `OK' button in the
modal window, the window becomes invisible and \pr{interaction-complete} is
called.  Interaction then resumes as usual in the interface.
See section \ref{w-i-c} for a discussion of \pr{wait-interaction-complete}.

The \pr{error-gadget} and \pr{query-gadget} dialog boxes use this feature
exactly as in the example above.  They
ensure that the user responds to the error message before continuing any
action in the rest of the interface.  The property sheet gadget display
routines and the \pr{gilt:show-in-window} routine have an optional modal
parameter which uses this feature.  You may be able to implement your design
using these gadgets and routines, rather than using the \pr{:modal-p} slot
explicitly.



\subsection{Change-Active}
\index{Change-Active}
\index{active}
\label{change-active}
Interactors can either be turned on and off individually using
the \pr{:active} slot in each interactor, or you can put a group of
interactors together in a priority level (see section \ref{priorities})
and turn on and off the entire group using the priority level's \pr{:active}
slot.

The \pr{:active} slot of an interactor may be \pr{s-value}'d explicitly,
causing the interactor to abort immediately.  But to change the activity of
a priority level, you should use the function \pr{Change-Active}:

\begin{programexample}
inter:Change-Active {\it an-interactor-or-priority-level new-value}\value{function}
\end{programexample}
\index{Change-Active}
This makes the interactor or priority-level be active (if {\it new-value} is T)
or inactive (if {\it new-value} is NIL).  When \pr{change-active} makes a
priority level not active, then all interactors on the priority level will
abort immediately.  Interactors are not guaranteed to abort immediately
if their priority level's \pr{:active} slot is simply set to NIL.



\section{Events}
\index{Events}
\label{Events}

Some functions, such as \pr{Start-Interactor} (see section
\ref{startinteractor}) take an ``event'' as a
parameter.  You might also want to look at an event to provide extra
features.

\pr{Inter:Event} is an interactor-defined structure (a regular Lisp
structure, not a KR schema), and is not the same as
the events created by the X window manager or Mac QuickDraw.
Normally, programs do not need to ever look at the event structure,
but it is exported from interactors in case you need it.

\pr{Inter:Event} has the following fields:
\begin{description}
\item[] \pr{Window}\index{window}\index{event-window}
- The Interactor window that the event occurred in.

\item[] \pr{Char}\index{Char}\index{event-Char}
- The Lisp character that the event corresponds to.  If this is a mouse
event, then the \pr{Char} field will actually hold a keyword like \pr{:leftdown}.

\item[] \pr{Code}\index{Code}\index{event-Code}
- The X/11 or MCL internal code for the event.

\item[] \pr{Mousep}\index{Mousep}\index{event-Mousep}
- Whether the event is a mouse event or not.

\item[] \pr{Downp}\index{Downp}\index{event-Downp}
- If a mouse event, whether it is a down-transition or not.

\item[] \pr{X}\index{X}\index{event-X}
- The X position of the mouse when the event happened.

\item[] \pr{Y}\index{Y}\index{event-Y}
- The Y position of the mouse when the event happened.

\item[] \pr{Timestamp}\index{Timestamp}\index{event-Timestamp}
- The X/11 or MCL timestamp for the event.
\end{description}

Each of the fields has a corresponding accessor and setf function:
\begin{programexample}
(event-window event)     (setf (event-window event) w)
(event-char event)       (setf (event-char event) c)
(event-code event)       (setf (event-code event) c)
(event-mousep event)     (setf (event-mousep event) T)
(event-downp event)      (setf (event-downp event) T)
(event-x event)          (setf (event-x event) 0)
(event-y event)          (setf (event-y event) 0)
(event-timestamp event)  (setf (event-timestamp event) 0)
\end{programexample}


\index{Make-Event}
You can create new events (for example, to pass to the
\pr{Start-Interactor} function), using the standard structure creation
function \pr{Make-Event}.
\begin{programexample}
inter:Make-Event \&key ({\it window} NIL) ({\it char} :leftdown) ({\it code} 1) ({\it mousep} T) \value{function}
                        ({\it downp} T) ({\it x} 0) ({\it y} 0) ({\it timestamp} 0))
\end{programexample}

\index{*Current-event*}\index{Current-event}
The last event that was processed by the interactors system is stored in
the variable \pr{Inter:*Current-Event*}.  This is often useful for
functions that need to know where the mouse is or what actual mouse or
keyboard key was hit.  Note that two of the fields of this event (window and
char) are copied into the slots of the interactor (see section
\ref{specialslots}) and can be more easily accessed from there.

\subsection{Example of using an event}
\label{twopselectexample}
\index{selecting in a rectangle}
The two-point interactor calls the final-function with a NIL parameter if
the rectangle is smaller than a specified size (see section
\ref{twopapplnotif}).  This feature can be used to allow the end user to
pick an object under the mouse if the user presses and releases, but to
select everything inside a rectangle if the user presses and moves (in this
case, moves more than 5 pixels).

Assume the objects to be selected are stored in the aggregate \pr{all-obj-agg}.
\index{Select objects inside a box}
\indexsecondary{Primary=`Examples', Secondary=`Select objects inside a box'}
\indexsecondary{Primary=`Examples', Secondary=`Events'}
\indexsecondary{Primary=`Examples', Secondary=`Two-Point-Interactor'}
\begin{programexample}
(create-instance 'SELECT-POINT-OR-BOX Inter:Two-Point-Interactor
  (:start-where T)
  (:start-event :leftdown)
  (:abort-if-too-small T)
  (:min-width 5)
  (:min-height 5)
  (:line-p NIL)
  (:flip-if-change-sides T)
  (:final-function
   \#'(lambda (an-interactor final-point-list)
       (if (null final-point-list)
	   ; {\it then select object at point.  Get point from}
	   ; {\it  the *Current-event* structure, and use it in the}
	   ; {\it  standard point-to-component routine.}
	   (setf selected-object
		 (opal:point-to-component ALL-OBJ-AGG
					  (inter:event-x inter:*Current-event*)
					  (inter:event-y inter:*Current-event*)))
	   ; {\it else we have to find all objects inside the rectangle.}
	   ; {\it  There is no standard function to do this.}
	   (setf selected-object
		 (My-Find-Objs-In-Rectangle ALL-OBJ-AGG final-point-list))))))
\end{programexample}


\section{Starting and Stopping Interactors Explicitly}
\index{Start-interactor}
\label{startinteractor}

Normally an interactor will start operating (go into the ``running'' state)
after its start-event happens over its start-where.  However, sometimes it
is useful to explicitly start an interactor without waiting for its start
event.  You can do this using the function \pr{Start-interactor}.
For example, if a menu selection should cause a sub-menu to start
operating, or if after creating a new rectangle you want to immediately
start editing a text string that is the label for that rectangle.

\begin{programexample}
inter:Start-Interactor {\it an-interactor} \&optional ({\it event} T)\value{function}
\end{programexample}

This function does nothing if the interactor is already running or if it is
not active.
If an event is passed in, then this is used as the x and y location to
start with.  This may be important for selecting which object the
interactor operates on, for example if the \pr{:start-where} of the
interactor is \pr{(:element-of {\tt\char`\<}agg>)}, the choice of which element is made
based on the value of x and y in the event.
(See section \ref{events} for a description of the event).  If
the event parameter is T (the default), then the last event that
was processed is re-used.  The event is also used to calculate
the appropriate default stop event (needed if the start-event is
a list or something like \pr{:any-mousedown} and the stop-event is not supplied).
If the event is specified as NIL or the x and y in the event do not pass
\pr{:start-where}, the interactor is still started, but the initial object
will be NIL, which might be a problem (especially for button-interactors,
for example).  NOTE: If you want the interactor to never start by itself,
then its \pr{:start-where} or \pr{:start-event} can be set to NIL.

Examples of using \pr{start-interactor} are in the file \pr{demo-sequence.lisp}.

\index{Abort-interactor}
Similarly, it is sometimes useful to abort an interactor explicitly.  This
can be done with the function:
\begin{programexample}
inter:Abort-Interactor {\it an-interactor}\value{function}
\end{programexample}
If the interactor is running, it is aborted (as if the abort event had occurred).

\index{Stop-Interactor}
\pr{Stop-Interactor} can be called to stop an interactor as if the stop
event had happened.
\begin{programexample}
inter:Stop-Interactor {\it an-interactor}\value{function}
\end{programexample}
It reuses the last object the interactor was operating on, and the current event is
ignored.  This function is useful if you want to have the interactor
stopped due to some other external action.
For example, to stop a text-interactor when the user chooses a menu item,
simply call stop-interactor on the text-interactor from the final-function
of the menu.

\section{Special slots of interactors}
\label{specialslots}

There are a number of slots of
interactors that are maintained by the system that can be used by
programmers in formulas or custom action routines.  These are:
\begin{description}
\item[] \pr{:first-obj-over} - this is set to the object that is returned from
\pr{:start-where}.  This might be useful if you want a formula in the
\pr{:obj-to-change} slot that will depend on which object is pressed on
(see the examples below and in section \ref{editstringexample}).  Note that
if the \pr{:start-where} is T, then
\pr{:first-obj-over} will be T, rather than an object.  The value in
\pr{:first-obj-over} does not change as the interactor is running (it is
only set once at the beginning).

\item[] \pr{:current-obj-over} - this slot is set with the object that the mouse
was last over (see section \ref{menufinalfeedbackobj}).

\item[] \pr{:current-window} - this is set with the actual window of the last (or
current) input event. This might be useful for multi-window interactors (see
section \ref{multiwindow}).  The \pr{:current-window} slot is set
repeatedly while the interactor is running.

\item[] \pr{:start-char} - The Lisp character (or keyword if a mouse event) of the
actual start event.  This
might be useful, for example, if the start event can be one of a set of
things, and some parameter of the interactor depends on which one.  See the
example below.  The value in
\pr{:start-char} does not change as the interactor is running (it is
only set once at the beginning).

\end{description}

\subsection{Example of using the special slots}

This example uses two slots of the interactor in formulas.  A formula in the
\pr{:grow-p} slot determines whether to move or grow an object based on
whether the user starts with a left or right mouse button
(\pr{:start-char}).  A formula in
the \pr{:line-p} slot decides whether to change this object as a line or a
rectangle based on
whether the object started on (\pr{:first-obj-over}) is a line or not.
Similarly, a formula in the feedback slot chooses the correct type of
object (line or rectangle).

The application creates a set of objects and stores them in an
aggregate called \pr{all-object-agg}.

\label{movegrowexample2}
\indexsecondary{Primary=`Examples', Secondary=`Move or Change Size'}
\indexsecondary{Primary=`Examples', Secondary=`Special Slots'}
\begin{programexample}
(create-instance 'MOVE-OR-GROWER Inter:Move-Grow-Interactor
   (:start-event '(:leftdown :rightdown))  ;{\it either left or right}
   (:grow-p (o-formula (eq :rightdown (gvl :start-char)))) ;{\it grow if right button}
   (:line-p (o-formula (is-a-p (gvl :first-obj-over) opal:line)))
   (:feedback-obj (o-formula
		   (if (gvl :line-p)
		       MY-LINE-FEEDBACK-OBJ
		       MY-RECTANGLE-FEEDBACK-OBJ)))
   (:start-where `(:element-of ,ALL-OBJECT-AGG))
   (:window MYWINDOW))
\end{programexample}

\section{Multiple Windows}
\index{Multiple Windows}
\label{multiwindow}
\index{window}
Interactors can be made to work over multiple windows.  The \pr{:window}
slot of an interactor can contain a single window (the normal case), a list
of windows, or T which means all Interactor windows (this is rather
inefficient).  If one of the last
two options is used, then the interactor will operate over all the
specified windows.   This means that as the interactor is running, mouse
movement events are processed for all windows that are referenced.  Also,
when the last of the windows referenced is deleted, then the interactor is
automatically destroyed.

This is mainly useful if you want to have an object move among various
windows.  If you want an object to track the mouse as it changes windows,
however, you have to explicitly change the aggregate that the object is in
as it follows the mouse, since each window has a single top-level aggregate
and aggregates cannot be connected to multiple windows.  You will probably
need a custom \pr{:running-action} routine to do this (see section
\ref{customroutines}).  This is true of the feedback object as well as the
main object.

You can look at the demonstration program
\pr{demo-multiwin.lisp} to see how this might be done.


\section{Wait-Interaction-Complete}
\label{w-i-c}

Interactors supplies a pair of functions which can be used to suspend
Lisp processing while waiting for the user to complete an action.
It is a little complicated to do this at the Interactors level,
but there is a convenient function for Gilt-created dialog boxes called
\pr{gilt:Show-In-Window-And-Wait} (see the Gilt manual).  Also,
\pr{garnet-gadgets:display-error-and-wait} and
\pr{garnet-gadgets:@{\tt\char`\|}display-query-and-wait} can be used to pop up
message windows and wait for the user's response (see the
\pr{error-gadget} in the Gadgets manual).

\index{wait-interaction-complete}
For other applications, you can call:
\begin{programexample}
inter:Wait-Interaction-Complete \&optional {\it window-to-raise}\value{function}
\end{programexample}
which does not return until an interactor executes:
\index{interaction-complete}
\begin{programexample}
inter:Interaction-Complete \&optional {\it val}\value{function}
\end{programexample}
If a {\it val} is supplied, then this is returned as the value of
\pr{Inter:Wait-Interaction-Complete}.  The {\it window-to-raise} parameter is
provided to avoid a race condition that occurs when you call update on a
window and
immediately call \pr{wait-interaction-complete}.  If you have problems with
this function, then try supplying your window as the optional argument.
\pr{Wait-interaction-complete} will then raise your window to the top and
update it for you.


Typically, \pr{Inter:Interaction-Complete} will be called in the
final-function of the interactor (or the selection-function of the gadget)
that should cause a value to be returned, such as a value associated with
the `OK' button of a dialog box.  Note that you must use some other mechanism
of interactors to make sure that only the interactors you care about are
executable; \pr{Wait-Interaction-Complete} allows {\it all} interactors in
{\it all} windows to run.




\section{Useful Procedures}

\index{bell}\index{beep}
The text interactor beeps (makes a sound) when you hit an illegal
character.  The function
to cause the sound is exported as
\begin{programexample}
inter:Beep\value{function}
\end{programexample}
which can be used anywhere in application code also.

\index{warp-pointer}
The Interactors package exports the function
\begin{programexample}
inter:Warp-Pointer {\it window x y}\value{function}
\end{programexample}
which moves the position of the mouse cursor to the specified point
in the specified window.  The result is the same as if the user had moved
the mouse to position {\tt\char`\<}x,y{\tt\char`\>}.



\section{Custom Action Routines}
\index{Custom Action Routines}
\index{Action Routines}
\label{customroutines}

We have found that the interactors supply sufficient flexibility to support
almost all kinds of interactive behaviors.  There are many parameters that
you can set in each kind of interactor, and you can use formulas
to determine values for these dynamically.  The \pr{final-function} can be used for
application notification if necessary.

However, sometimes a programmer may find that special actions are required
for one or more of the action routines.  In this case, it is easy to
override the default behavior and supply your own functions.  As described
in section \ref{interslots}, the action routines are:
\begin{description}
\item[] \pr{:stop-action}

\item[] \pr{:start-action}

\item[] \pr{:running-action}

\item[] \pr{:abort-action}

\item[] \pr{:outside-action}

\item[] \pr{:back-inside-action}
\end{description}

Each of the interactor types has its own functions supplied in each of
these slots.

If you want the default behavior {\it in addition to} your own custom behavior, then
you can use the KR function \pr{Call-Prototype-Method} to call the standard
function from your function.  The parameters are the same as for your function.

For example, the \pr{:running-action} for Move-Grow interactors is defined
(in section \ref{movegrowcustomaction}) as:
\begin{programexample}
(lambda (an-interactor object-being-changed new-points))
\end{programexample}
so to create an interactor with a custom action as well as the default
action, you might do:
\label{movegrowexample3}
\indexsecondary{Primary=`Examples', Secondary=`Running-action'}
\begin{programexample}
(create-instance NIL Inter:Move-Grow-Interactor
   ... {\it the other usual slots}
   (:running-action
    \#'(lambda (an-interactor object-being-changed new-points)
	(call-prototype-method an-interactor object-being-changed new-points)
	({\it Do-My-Custom-Stuff}))))
\end{programexample}

The parameters to all the action procedures for all the interactor types
are defined in the following sections.

\subsection{Menu Action Routines}
\indexsecondary{Primary=`Action Routines', Secondary=`Menu'}
\index{Menu action routines}
\label{Menucustomaction}
\index{Menu action routines}

The parameters to the action routines of menu interactors are:

\begin{description}

\pr{:Start-action} -
\begin{programinlist}
(lambda (an-interactor first-object-under-mouse))
\end{programinlist}
Note that \pr{:running-action} is not called until the
mouse is moved to a different object (it is not called on this first object
which is passed as \pr{first-object-under-mouse}).



\pr{:Running-action} -
\begin{programinlist}
(lambda (an-interactor prev-obj-over new-obj-over))
\end{programinlist}
This is called once each time the object under the mouse changes (not each
time the mouse moves).



\pr{:Outside-action} -
\begin{programinlist}
(lambda (an-interactor outside-control prev-obj-over))
\end{programinlist}
This is called when the mouse moves out of the entire menu.
% \comment{ (or over one of the exception items)}
\pr{Outside-Control} is simply the value of the \pr{:outside} slot.



\pr{:Back-inside-action} -
\begin{programinlist}
(lambda (an-interactor outside-control prev-obj-over new-obj-over))
\end{programinlist}
Called when the mouse was outside all items and then moved back inside.
\pr{Prev-obj-over} is the last object the mouse was over before it went
outside.  This is used to remove feedback from it if \pr{:outside} is
\pr{:last}.



\pr{:Stop-action} -
\begin{programinlist}
(lambda (an-interactor final-obj-over))
\end{programinlist}
The interactor guarantees that \pr{:running-action} {\it has} been called on
\pr{final-obj-over} before the \pr{:stop-action} procedure is called.



\pr{:Abort-action} -
\begin{programinlist}
(lambda (an-interactor last-obj-over))
\end{programinlist}

\end{description}

\subsection{Button Action Routines}
\indexsecondary{Primary=`Action Routines', Secondary=`Button'}
\index{Button action routines}
\label{buttoncustomaction}

The parameters to the action routines of button interactors are:

\begin{description}

\pr{:Start-action} -
\begin{programinlist}
(lambda (an-interactor object-under-mouse))
\end{programinlist}
Note that \pr{back-inside-action} is not called this first time.



\pr{:Running-action} -
This is not used by this interactor.
\pr{:Back-inside-action} and \pr{:Outside-action} are used instead.



\pr{:Back-inside-action} -
\begin{programinlist}
(lambda (an-interactor new-obj-over))
\end{programinlist}
This is called each time the mouse comes back to the original object.



\pr{:Outside-action} -
\begin{programinlist}
(lambda (an-interactor last-obj-over))
\end{programinlist}
This is called if the mouse moves outside of
\pr{:running-where} before stop-event.  The default \pr{:running-where} is
\pr{'(:in *)} which means in the object that the interactor started on.



\pr{:Stop-action} -
\begin{programinlist}
(lambda (an-interactor final-obj-over))
\end{programinlist}



\pr{:Abort-action} -
\begin{programinlist}
(lambda (an-interactor obj-over))
\end{programinlist}
Obj-over will be the object originally pressed on, or NIL if outside
when aborted.


\end{description}


\subsection{Move-Grow Action Routines}
\indexsecondary{Primary=`Action Routines', Secondary=`Move-Grow'}
\index{Move-Grow action routines}
\label{movegrowcustomaction}

The parameters to the action routines of move-grow interactors are:

\begin{description}

\pr{:Start-action} -
\begin{programinlist}
(lambda (an-interactor object-being-changed first-points))
\end{programinlist}
\pr{First-points} is a list of the original left, top, width and
height for the object, or the original X1, Y1, X2, Y2, depending on the
setting of \pr{:line-p}.  The \pr{object-being-changed} is the actual object
to change, not the feedback object.  Note that \pr{:running-action} is not
called on this first point; it will not be called
until the mouse moves to a new point.



\pr{:Running-action} -
\begin{programinlist}
(lambda (an-interactor object-being-changed new-points))
\end{programinlist}
The \pr{object-being-changed} is the actual object
to change, not the feedback object.



\pr{:Outside-action} -
\begin{programinlist}
(lambda (an-interactor outside-control object-being-changed))
\end{programinlist}
The \pr{object-being-changed} is the actual object
to change, not the feedback object.  \pr{Outside-control} is set with the
value of \pr{:outside}.



\pr{:Back-inside-action} -
\begin{programinlist}
(lambda (an-interactor outside-control object-being-changed new-inside-points))
\end{programinlist}
The \pr{object-being-changed} is the actual object
to change, not the feedback object.  Note that
the running-action procedure is not called on the point passed to this
procedure.



\pr{:Stop-action} -
\begin{programinlist}
(lambda (an-interactor object-being-changed final-points))
\end{programinlist}
The \pr{object-being-changed} is the actual object
to change, not the feedback object.  \pr{:Running-action} was not
necessarily called on the point passed to this procedure.



\pr{:Abort-action} -
\begin{programinlist}
(lambda (an-interactor object-being-changed))
\end{programinlist}
The \pr{object-being-changed} is the actual object
to change, not the feedback object.


\end{description}


\subsection{Two-Point Action Routines}
\indexsecondary{Primary=`Action Routines', Secondary=`Two-Point'}
\index{Two-Point action routines}
\label{twopcustomaction}

The parameters to the action routines of two-point interactors are:

\begin{description}

\pr{:Start-action} -
\begin{programinlist}
(lambda (an-interactor first-points))
\end{programinlist}
The \pr{first-points} is a list of the initial box or 2 points for the object
(the form is determined by the \pr{:line-p} parameter).  If
\pr{:abort-if-too-small} is non-NIL, then \pr{first-points} will be NIL.
Otherwise, the width and height of the object will be the \pr{:min-width} and
\pr{:min-height} or 0 if there are no minimums.
Note that \pr{:running-action} is not
called on this first point; it will not be called
until the mouse moves to a new point.



\pr{:Running-action} -
\begin{programinlist}
(lambda (an-interactor new-points))
\end{programinlist}
\pr{New-points} may be NIL if \pr{:abort-if-too-small} and
the size is too small.



\pr{:Outside-action} -
\begin{programinlist}
(lambda (an-interactor outside-control))
\end{programinlist}
\pr{Outside-control} is set with the value of \pr{:outside}.



\pr{:Back-inside-action} -
\begin{programinlist}
(lambda (an-interactor outside-control new-inside-points))
\end{programinlist}
Note that the running-action procedure is not called on the point passed to this
procedure.  \pr{New-inside-points} may be NIL if \pr{:abort-if-too-small}
is non-NIL.



\pr{:Stop-action} -
\begin{programinlist}
(lambda (an-interactor final-points))
\end{programinlist}
\pr{:Running-action} was not
necessarily called on the point passed to this procedure.
\pr{Final-points} may be NIL if \pr{:abort-if-too-small} is non-NIL.



\pr{:Abort-action} -
\begin{programinlist}
(lambda (an-interactor))
\end{programinlist}


\end{description}

\subsection{Angle Action Routines}
\indexsecondary{Primary=`Action Routines', Secondary=`Angle'}
\index{Angle action routines}
\label{anglecustomaction}

In addition to the standard measure of the angle, the
procedures below also provide an incremental measurement of the difference
between the current and last values.  This might be used if you just want
to have the user give circular gestures to have something rotated.  Then,
you would just want to know the angle differences.  An example of this is in
\pr{demo-angle.lisp}.

The parameters to the action routines of angle interactors are:

\begin{description}

\pr{:Start-action} -
\begin{programinlist}
(lambda (an-interactor object-being-rotated first-angle))
\end{programinlist}
The
\pr{first-angle} is the angle from directly to the right of the
\pr{:center-of-rotation} that the mouse presses.  This angle is in
radians.  The \pr{object-being-rotated} is the actual object
to move, not the feedback object.  Note that \pr{:running-action} is not
called on \pr{first-angle}; it will not be called
until the mouse moves to a new angle.



\pr{:Running-action} -
\begin{programinlist}
(lambda (an-interactor object-being-rotated new-angle angle-delta))
\end{programinlist}
The \pr{object-being-rotated} is the actual object
to move, not the feedback object.  \pr{Angle-delta} is the difference
between the current angle and the last angle.  It will either be positive
or negative, with positive being counter-clockwise.  Note that it is always
ambiguous which way the mouse is rotating from sampled points, and the
system does not {\it yet} implement any hysteresis, so if the user rotates
the mouse swiftly (or too close around the center point), the delta may
oscillate between positive and negative values, since it will guess wrong
about which way the user is going.  {\it In the future, this could be fixed
by keeping a history of the last few points and assuming the user is going
in the same direction as previously.}



\pr{:Outside-action} -
\begin{programinlist}
(lambda (an-interactor outside-control object-being-rotated))
\end{programinlist}
The \pr{object-being-rotated} is the actual object
to move, not the feedback object.  \pr{Outside-control} is set with the
value of \pr{:outside}.



\pr{:Back-inside-action} -
\begin{programinlist}
(lambda (an-interactor outside-control object-being-rotated new-angle))
\end{programinlist}
The \pr{object-being-rotated} is the actual object
to move, not the feedback object.  Note that
the running-action procedure is not called on the point passed to this
procedure.  There is no \pr{angle-delta} since it would be zero if
\pr{:outside-control} was \pr{NIL} and it would probably be inaccurate for
\pr{:last} anyway.



\pr{:Stop-action} -
\begin{programinlist}
(lambda (an-interactor object-being-rotated final-angle angle-delta))
\end{programinlist}
The \pr{object-being-rotated} is the actual object
to move, not the feedback object.  \pr{:Running-action} was not
necessarily called on the angle passed to this procedure.  \pr{Angle-delta}
is the difference from the last call to \pr{:running-action}.



\pr{:Abort-action} -
\begin{programinlist}
(lambda (an-interactor object-being-rotated))
\end{programinlist}
The \pr{object-being-rotated} is the actual object
to move, not the feedback object.


\end{description}


\subsection{Text Action Routines}
\indexsecondary{Primary=`Action Routines', Secondary=`Text'}
\index{Text action routines}
\label{textcustomaction}

The parameters to the action routines of text interactors are:

\begin{description}

\pr{:Start-action} -
\begin{programinlist}
(lambda (an-interactor new-obj-over start-event))
\end{programinlist}
\pr{New-Obj-over} is the
object to edit, either \pr{:obj-to-change} if it is supplied, or if
\pr{:obj-to-change} is NIL, then the object returned from \pr{:start-where}.
The definition of \pr{event}s is in section \ref{events}.



\pr{:Running-action} -
\begin{programinlist}
(lambda (an-interactor obj-over event))
\end{programinlist}



\pr{:Outside-action} -
\begin{programinlist}
(lambda (an-interactor obj-over))
\end{programinlist}
Often, \pr{:running-where} will be T so that this is never called.



\pr{:Back-Inside-action} -
\begin{programinlist}
(lambda (an-interactor obj-over event))
\end{programinlist}



\pr{:Stop-action} -
\begin{programinlist}
(lambda (an-interactor obj-over stop-event))
\end{programinlist}



\pr{:Abort-action} -
\begin{programinlist}
(lambda (an-interactor obj-over abort-event))
\end{programinlist}


\end{description}



\subsection{Gesture Action Routines}
\indexsecondary{Primary=`action routines', Secondary=`Gesture'}
\index{gesture action routines}
\label{gestcustomaction}

The parameters to the action routines of gesture interactors are:

\begin{description}

\pr{:Start-action} -
\begin{programinlist}
(lambda (an-interactor object-under-mouse point))
\end{programinlist}
The \pr{point} is the first point of the gesture.




\pr{:Running-action} -
\begin{programinlist}
(lambda (an-interactor new-obj-over point))
\end{programinlist}



\pr{:Outside-action} -
\begin{programinlist}
(lambda (an-interactor prev-obj-over))
\end{programinlist}
This beeps and erases the trace if \pr{show-trace} is non-NIL. It also
sets \pr{:went-outside} to T.



\pr{:Back-inside-action} -
\begin{programinlist}
(lambda (an-interactor new-obj-over))
\end{programinlist}
This currently does nothing.



\pr{:Stop-action} -
\begin{programinlist}
(lambda (an-interactor final-obj-over point))
\end{programinlist}
\pr{:Running-action} was not necessarily called on the point passed to this
procedure, so it is added to \pr{*points*}.  This procedure calls
\pr{gest-classify} with the points in the trace, \pr{*points*}, and
the classifier given by \pr{:classifier}.




\pr{:Abort-action} -
\begin{programinlist}
(lambda (an-interactor))
\end{programinlist}
This erases the trace if \pr{:show-trace} is non-NIL and
\pr{:went-outside} is NIL.


\end{description}


\subsection{Animation Action Routines}

The \pr{animator-interactor} does not use these action slots.  All of the
work is done by the function supplied in the \pr{:timer-handler} slot.




\string{titstring=`\value{chapter} Debugging'}
\chapter{Debugging}
\index{Debugging}

There are a number of useful functions that help the programmer debug
interactor code.  Since these are most useful in conjunction with the tools
that help debug KR structures and Opal graphical objects, all of these are
described in a separate Garnet Debugging Manual.

In summary, the functions provided include:

\begin{itemize}
\item \index{PS}
Interactors are KR objects so they can be printed using
\pr{kr:ps} and \pr{hemlock-schemas}.

\item \index{Trace-Inter}
The \pr{Inter:Trace-Inter} routine is
useful for turning on and off tracing output that tells what interactors
are running.  Type \pr{(describe 'inter:trace-inter)} for a description.
{\it This function is only available when the \pr{garnet-debug} compiling
switch is on (the default).}

\item \index{Ident}
\pr{(garnet-debug:ident)} will tell the name of the next event
(keyboard key or mouse button) you hit.

\item \index{Look-inter}
\pr{(garnet-debug:look-inter \&optional parameter)} describes the active
interactors, or a particular interactor, or the interactors that affect a
particular graphic object.

\item \index{Print-Inter-Levels}
\pr{(inter:Print-Inter-Levels)}
will print the names of all of the active interactors in all priority levels.

\item \index{Print-Inter-Windows}\index{Windows (debugging function)}
\pr{(inter:Print-Inter-Windows)} will print the names of all the interactor
windows, and \pr{(garnet-debug:Windows)} will print all Opal and Interactor
windows.

\item \index{clean-up}
Destroying the interactor windows will normally get rid of interactors.
You can use \pr{(opal:clean-up :opal)} to delete all interactor windows.

\item 
If for some reason, an interactor is not
deleted (for example, because it is not attached to a window), then
\index{Reset-Inter-Levels}
\begin{programexample}
inter:Reset-Inter-Levels \&optional {\it level}\value{function}
\end{programexample}
will remove {\it all} the existing interactors by simply resetting the queues (it
does not destroy the existing interactors, but they will never be
executed).  If a level is specified, then only interactors on that level
are destroyed.  If level is NIL (the default), then all levels are reset.
This procedure should not be used in applications\dashonly for debugging.
It is pretty drastic.

\end{itemize}

% \begin{comment}
% ====================================================
% \string{titstring=`\value{chapter} Issues'}
% \chapter{Issues}
% 
% How handle sliding out of a menu and having a sub-menu appear?
% 
% Should we increase the number of interactors and decrease the number of
% parameters to each?
% 
% What other specific interactors are needed?
% 
% Need trill button also?
% 
% Can we eliminate some of these interactors if we make interactors simpler?
% For example, use a timer interactor and a regular button interactor
% together to make a trill.
% 
% Can Menu and Button interactors be combined?
% 
% How to make the feedback object have the same properties as the real
% object, e.g., with respect to constraints on movement (grids, only in X,
% etc.).  Want to be able to get them dynamically from the object and/or from
% the particular interactor or parameters to the interactor (e.g., when press
% here, only move in X)  Solutions: extra parameters w/filters passed to the
% interactors, have a `move-object' function (like create-function in twop)
% that is passed either the feedback-obj or the real object (but this is
% similar to the running-action and stop-action), somehow copy the
% constraints into the feedback object (may require creating a new object
% and/or new constraint objects), or leave as is so user defines own
% running-action and stop-action procedures.
% ====================================================
% \end{comment}
\chapter*{References}
\bibliography
